% =============================================================================
% Titel:		MMQM - Mitschrieb
% Erstellt:	SS 10
% Dozent:	Prof. Dr. M. Griesemer
% Autor:	Jan-Cornelius Molnar
% =============================================================================
\documentclass[%
	paper=a5,%
	fleqn,%
% ===================================================
%  10pt aktivieren, falls *nicht* die Lucida Schrift verwendet wird.
% =================================================== 
	10pt,%
	DIV=18,%
	BCOR=0mm,
%	headings=openleft,
	titlepage=false,%
	twoside=true]%
	{scrbook}

% =============================================================================
% 					Benötigte Pakete
% =============================================================================

% ===================================================
%  Option nolucida aktivieren, falls *nicht* die Lucida Schrift verwendet wird.
% =================================================== 
\usepackage%[nolucida]%
				    {janmcommon}
\usepackage{janmscript}
\usepackage{fancyhdr}
\usepackage{float}
\restylefloat{figure}
\usepackage{marginnote}
\usepackage{makeidx}

\makeindex

% =============================================================================
% 					Theorem-Style
% =============================================================================
% Theorem Umgebungen *MIT* Numerierung
\theoremstyle{graymarginwithblueheader}
\theorembodyfont{\itshape}
\theoremseparator{}
\theoremsymbol{}

\newtheorem{prop}{Satz}[chapter]
\newtheorem{thm}[prop]{Theorem}
\newtheorem{lem}[prop]{Lemma}
\newtheorem{defn}[prop]{Definition}
\newtheorem{cor}[prop]{Korollar}

\theoremstyle{graymarginwithyellowheader}
\theorembodyfont{\normalfont}
\theoremseparator{}
\theoremsymbol{}

\newtheorem{bsp}[prop]{Bsp}

\theoremstyle{graymarginwithitblackheader}
\theorembodyfont{\normalfont}
\theoremseparator{}
\theoremsymbol{}

\newtheorem{bem}[prop]{Bemerkung.}

% Theorem Umgebungen *OHNE* Numerierung
\theoremstyle{graymarginwithblueheadern}
\theorembodyfont{\itshape}
\theoremseparator{}
\theoremsymbol{}

\renewtheorem{defn*}{Definition}
\renewtheorem{prop*}{Satz}
\renewtheorem{lem*}{Lemma}
\renewtheorem{cor*}{Korollar}

\theoremstyle{graymarginwithyellowheadern}
\theorembodyfont{\normalfont}
\theoremseparator{}
\theoremsymbol{}

\renewtheorem{bsp*}{Bsp}

\theoremstyle{graymarginwithitblackheadern}
\theorembodyfont{\normalfont}
\theoremseparator{}
\theoremsymbol{}

\renewtheorem{bem*}{Bemerkung.}

% =============================================================================
% 					Überschriften-Style
% =============================================================================
\renewcommand\thesection{\arabic{chapter}-\Alph{section}}
\renewcommand\thesubsection{{\small\ensuremath{\blacksquare}}}
\renewcommand\thebsp{\arabic{bsp}}

\setkomafont{chapter}{\normalfont\bfseries\Huge\color{darkblue}}
\setkomafont{section}{\normalfont\bfseries\Large\color{darkblue}}
\setkomafont{subsection}{\normalfont\bfseries\color{darkblue}}

% =============================================================================
% 					PSTricks-Standards
% =============================================================================
\psset{linecolor=gdarkgray}
\psset{tickcolor=gdarkgray}
\psset{fillcolor=glightgray}

% =============================================================================
% 					Header/Footer-Style
% =============================================================================
% Select page style
\pagestyle{fancyplain}

% Reset header & footer
\fancyhf{}

% Reset chapter & sectionmark
\renewcommand{\chaptermark}[1]{\markboth{\textsc{#1}}{}}
\renewcommand{\sectionmark}[1]{\markright{\textsc{#1}}{}}

% Clear headrule
\renewcommand{\headrule}{}

% =============================================================================
% 					Listen-Style
% =============================================================================
\newenvironment{bemenum}%
	{\begin{enumerate}[label=\textsc{\alph{*}}.,leftmargin=17pt]}{\end{enumerate}}
\newenvironment{defnenum}%
	{\begin{enumerate}[label={\rmfamily(\alph{*})}]}{\end{enumerate}}
\newenvironment{propenum}%
	{\begin{enumerate}[label=\arabic{*})]}{\end{enumerate}}
\newenvironment{equivenum}%
	{\begin{enumerate}[label=(\roman{*})]}{\end{enumerate}}
\newenvironment{bspenum}%
	{\begin{enumerate}[label=\alph{*}.),leftmargin=17pt]}{\end{enumerate}}
\newenvironment{proofenumarabicbr}%
	{\begin{enumerate}[label=(\arabic{*}),leftmargin=17pt]}{\end{enumerate}}
\newenvironment{proofenumroman}%
	{\begin{enumerate}[label=(\roman{*}),leftmargin=17pt]}{\end{enumerate}}
\newenvironment{proofenum}%
	{\begin{enumerate}[label=\arabic{*}),leftmargin=17pt]}{\end{enumerate}}
\newenvironment{proofenuma}%
	{\begin{enumerate}[label=\alph{*}.),leftmargin=2pt]}{\end{enumerate}}

% =============================================================================
% 					Eigene Operatoren
% =============================================================================
\renewcommand{\labelenumi}{{\normalfont(\alph{enumi})}}

\renewcommand{\AA}{\mathcal{A}}

\renewcommand{\Id}{\mathrm{Id}}
\DeclareMathOperator{\codim}{codim}
\DeclareMathOperator{\ran}{ran}
\DeclareMathOperator{\plim}{p-lim}
\DeclareMathOperator{\slim}{s-lim}

\newcommand{\pto}{\overset{p}{\to}}

\newcommand{\ocirc}[1]{\overset{\circ}{#1}}
\renewcommand{\bar}[1]{\overline{#1}}

%  \setlength{\oddsidemargin}{-51pt}
%  \setlength{\evensidemargin}{-35pt}
%  \setlength{\textwidth}{379pt}

%\addtokomafont{partentrypagenumber}{\sf\bfseries\Huge}
%
%\setkomafont{chapterentrypagenumber}{\sf}
%\setkomafont{sectionentrypagenumber}{\sf}
%\setkomafont{partentry}{\sf\Huge}
%\setkomafont{disposition}{\sf\Huge}
%\setkomafont{minisec}{\sf\Huge}


%\setcounter{tocdepth}{2}

% =============================================================================
% 					Document-Body
% =============================================================================
\begin{document}

% Titel
\begin{titlepage}
\vspace*{2mm}
\noindent\bfseries\color{gdarkgray}
Prof. Dr. M. Griesemer, Universität Stuttgart

\begin{center}
\vspace*{10mm}
{\noindent\huge\bfseries\color{darkblue} Mathematische Methoden\\
\vspace{2mm}
der Quantenmechanik}

\vspace*{10mm}
Vorlesungsmitschrieb von Jan-Cornelius Molnar

\vspace*{4mm}

Stuttgart, Sommersemester 2010
\end{center}

\vspace*{\fill}

\begin{flushright}
\small
Version: \today\ \thistime
\vspace*{5mm}

Für Hinweise auf Druckfehler und Kommentare jeder Art bin ich dankbar.
Viel Spaß!\footnote{\color{gdarkgray}
Prof. Dr. M. Griesemer, Universität Stuttgart,
\href{mailto:marcel.griesemer@mathematik.uni-stuttgart.de}{marcel.griesemer@mathematik.uni-stuttgart.de};\\
Jan-Cornelius Molnar,
\href{mailto:jan.molnar@studentpartners.de}{jan.molnar@studentpartners.de}}
\end{flushright}
\end{titlepage}


%\renewcommand{\normalfont}{\sf}

% Inhaltsverzeichnis
% {
% \sf
% \renewcommand{\normalfont}{}
% \tableofcontents
% }
\tableofcontents

\fancyhead[RO]{\footnotesize\color{gdarkgray}%
	\marginnote{$\Big|$\;\textbf{\thesection}}\rightmark}
\fancyhead[LE]{\footnotesize\color{gdarkgray}%
	\marginnote{\;\textbf{\thechapter}$\Big|$}\leftmark}

% Zweiseitig
\fancyfoot[LE]{\footnotesize\color{gdarkgray}%
 	\thepage}%
 \fancyfoot[RO]{\footnotesize\color{gdarkgray}%
 	\thepage}%
 \fancyfoot[RE,LO]{\tiny\color{gdarkgray}\today\; \thistime}

% Inhalt
\chapter{Eigenschaften von Wellenfunktionen}

\begin{lem}[Lemma von Riemann-Lebesgue]
\label{prop:1.1}
\index{Lemma!von Riemann-Lebesgue}
Sei $u\in L^1(\R^n)$. Dann ist $\hat{u}\in L^\infty(\R^n)\cap C(\R^n)$,
$\norm{\hat{u}}_\infty \le (2\pi)^{-n/2}\norm{u}_1$ und
\begin{align*}
\hat{u}(p) \to 0,\qquad \abs{p}\to\infty.\fishhere
\end{align*}
\end{lem}
\begin{proof}
\textit{Beschränktheit}. 
Sei $u\in L^1(\R^n)$. Dann ist
\begin{align*}
&\hat{u}(p) = (2\pi)^{-n/2}\int e^{-ipx}u(x)\dx,\\
\Rightarrow\; &
\abs{\hat{u}(p)} \le (2\pi)^{-n/2}
\int\abs{u(x)}\dx = (2\pi)^{-n/2}\norm{u}_1.
\end{align*}
\textit{Stetigkeit}. Anwendung des Satz von Lebesgue ergibt für $\abs{h}\to 0$
\begin{align*}
\hat{u}(p+h)-\hat{u}(p) = (2\pi)^{-n/2}
\int e^{-ipx}\underbrace{\left(e^{-ihx} - 1\right)}_{\to 0}u(x)\dx \to 0,
\end{align*}
denn $|$Integrand$| \le 2\abs{u}\in L^1(\R^n)$.

Sei $\ep > 0$. Da $\SS(\R^n)$ dicht in $L^1(\R^n)$ liegt, gibt es ein $v\in
\SS(\R^n)$, so dass $\norm{u-v}_1 < \frac{\ep}{2}$. Da
$\hat{v}\in\SS(\R^n)$ gibt es außerdem ein $R>0$ mit
\begin{align*}
\abs{\hat{v}(p)} < \frac{\ep}{2},\qquad \abs{p} \ge R.
\end{align*}
Also gilt für $\abs{p}\ge R$,
\begin{align*}
\abs{\hat{u}(p)} \le \abs{\hat{u}(p)-\hat{v}(p)} + \abs{\hat{v}(p)}
\le (2\pi)^{-n/2}\norm{u-v}_1 + \abs{\hat{v}(p)} < \ep.\qedhere
\end{align*}
\end{proof}

\begin{cor}
\label{prop:1.2}
Sei $u\in L^1(\R^n)$ und $\abs{x}^ku\in L^1(\R^n)$. Dann ist $\hat{u}\in
C^k(\R^n)$ und für $\abs{\alpha}\le k$ gilt
\begin{align*}
\partial^\alpha \hat{u}(p) = \widehat{(-ix)^\alpha u}(p) \to 0,\qquad
\abs{p}\to \infty.\fishhere
\end{align*}
\end{cor}
\begin{proof}
\textit{Induktion über $\abs{\alpha}$}. Der Fall $\abs{\alpha}=0$ ist klar.

Sei $j\in\setd{1,..,n}$ und $e_j = (0,\ldots,1,\ldots,0)$ der $j$-te
Basisvektor. Dann ist
\begin{align*}
\frac{1}{h}\left(\hat{u}(p+he_j) - \hat{u}(p) \right)
= (2\pi)^{-n/2}\int e^{-ipx}\underbrace{\frac{1}{h}\left(e^{-ihx_j}-1
\right)}_{\to -ix_j}u(x)\dx,
\end{align*}
wobei $|$Integrand$| \le 2\abs{u}\abs{x_j}\in L^1(\R^n)$ und daher nach Lemma
\ref{prop:1.1} und dem Satz von Lebesgue
\begin{align*}
\partial_j \hat{u}(p) = \widehat{(-ix_j)u}(p) \to 0,\qquad \abs{p}\to\infty.
\end{align*}
Damit ist der Induktionsschrit von $\abs{\alpha} \le k-1$ nach $\abs{\alpha}
\le k$ beweisen. (Induktionsannahme auf $\partial^{\alpha-e_j} u$
anwenden).\qedhere
\end{proof}

\begin{bem*}[Folgerung.]
Falls $u\in L^2(\R^n)$, $\hat{u}\in L^2(\R^n)$ und $\abs{p}^k\hat{u}\in
L^1(\R^n)$, dann ist $u\in C^k(\R^n)$ und es gilt
\begin{align*}
\partial^\alpha u = \FF^{-1}(ip)^\alpha \FF u.\maphere
\end{align*}
\end{bem*}

\begin{bem*}
Insbesondere gilt diese Identität für alle $u\in\SS(\R^n)$. Die rechte Seite
ist aber auch dann sinnvoll (wohldefiniert), wenn $u\in L^2(\R^n)$ und
$\abs{p}^{\abs{\alpha}}\hat{u}\in L^2(\R^n)$.\maphere
\end{bem*}

Dies motiviert die folgende

\begin{defn*}
Für jedes $s\ge 0$ definieren wir einen Innenproduktraum\index{Sobolevraum}
\begin{align*}
&H^s(\R^n) := \setdef{u\in L^2(\R^n)}{\hat{u}(p)\abs{p}^s \in L^2(\R^n)},\\
&\lin{u,v}_s := \int_{\R^n}
\overline{\hat{u}(p)}\hat{v}(p)(1+p^2)^s\ddp.
\end{align*}
$H^s(\R^m)$ heißt \emph{Sobolevraum}.\fishhere
\end{defn*}

Aus der Plancherel-Gleichung folgt unmittelbar, dass $H^0(\R^n)=L^2(\R^n)$ und
weiterhin gilt
\begin{align*}
s> t\ge 0\Rightarrow H^s(\R^n)\subset H^t(\R^n)\subset L^2(\R^n).
\end{align*}

\begin{prop}
\label{prop:1.3}
Für jedes $s\ge 0$ ist $H^s(\R^n)$ ein Hilbertraum und $C_0^\infty(\R^n)$ ist
dicht in $H^s(\R^n)$.\fishhere
\end{prop}
\begin{proof}
Einen Beweis findet man z.B. in \cite{Fun07}.\qedhere
\end{proof}

\begin{thm}[Theorem (Sobolev-Lemma)]
\label{prop:1.4}
\index{Lemma!Sobolev-}
Sei $u\in H^s(\R^n)$ und $k\in\N_0$ mit $k< s- n/2$. Dann gelten
\begin{propenum}
\item\label{prop:1.4:1} $u\in C^k(\R^n)$ und $\partial^\alpha u(x) =
\FF^{-1}((ip)^\alpha \hat{u})(x)\to 0$ für $\abs{x}\to\infty$, $\abs{\alpha}\le
k$.
\item\label{prop:1.4:2} $\norm{\partial^\alpha u}_\infty =
\sup\limits_{x\in\R^n} \abs{\partial^\alpha u(x)} \le C_{s,k,n}\norm{u}_s$ für
$\abs{\alpha}\le k$.\fishhere
\end{propenum}
\end{thm}
\begin{proof}
Sei $u\in H^s$ und $s>n/2+k$.
``\ref{prop:1.4:1}'': Wir wenden die Folgerung aus Korollar \ref{prop:1.2} auf
$u$ an.
\begin{align*}
&\int_{\R^n} \abs{\hat{u}(p)}\ddp,
\int_{\R^n} \abs{p}^k\abs{\hat{u}(p)}\ddp
\le
\int_{\R^n} (1+\abs{p}^2)^{k/2}\abs{\hat{u}(p)}\ddp\\
&\quad=
\int_{\R^n} \abs{\hat{u}(p)}(1+\abs{p}^2)^{s/2}(1+\abs{p}^2)^{k/2-s/2}\ddp\\
&\quad\overset{\text{CSB}}{\le}
\norm{u}_s \left(\int_{\R^n}
(1+\abs{p}^2)^{k-s}\ddp\right)^{1/2}.
\end{align*}
Das Integral ist endlich für $s > n/2+k$, denn
\begin{align*}
\int_{\R^n}
(1+\abs{p}^2)^{k-s}\ddp
= \abs{\S^{n-1}}
\int_{0}^\infty
\frac{1}{(1+t^2)^{s-k}}t^{n-1}\dt
\end{align*}
wobei $2(s-k) - (n-1) > 1 \Leftrightarrow 2(s-k) > n\Leftrightarrow s-k > n/2$.
Mit Korollar \ref{prop:1.2} folgt nun \ref{prop:1.4:1} und außerdem:

``\ref{prop:1.4:2}'': Da
$\partial^\alpha u(x) = (2\pi)^{-n/2} \int_{\R^n} e^{ipx}(ip)^\alpha
\hat{u}(p)\ddp$ ist folglich
\begin{align*}
\abs{\partial^\alpha u(x)} 
&\le (2\pi)^{-n/2} \int_{\R^n} \abs{p^\alpha}
\abs{\hat{u}(p)}\ddp
\le C_{s,p,k}\norm{u}_s.\qedhere
\end{align*}
\end{proof}

\begin{bem*}[Folgerungen.]
\begin{bemenum}
\item $\bigcap_{s\ge 0} H^s(\R^n)\subset C^\infty(\R^n)$.
\item Der maximale Definitionsbereich des Laplace-Operators im $\R^3$ ist
$H^2(\R^3)$. Für $\ph\in H^2(\R^3)$ gilt folglich $\ph\in C(\R^n)$ und
$\ph(x)\to 0$ für $\abs{x}\to\infty$.

Ist $\ph\in H^2(\R)$, so gilt sogar $\ph\in C^1(\R)$ und $\ph,\ph'\to 0$ für
$\abs{x}\to\infty$.

Ist $\ph\in H^1(\R)$, d.h. $\ph\in L^2(\R)$ und die ``kinetische Energie'' ist
endlich
\begin{align*}
\int_{\R^n} \abs{p}^2\abs{\hat{\ph}(p)}^2\ddp < \infty,
\end{align*}
so ist $\ph\in C(\R)$ und $\ph(x)\to 0$ für $\abs{x}\to\infty$.
\item Sei $V\in L^2(\R^n)$, $\ph\in H^2(\R^n)$ und $n\le 3$. Nach dem
Sobolev-Lemma ist dann $\ph\in L^\infty(\R^n)$ und somit
\begin{align*}
\int_{\R^n} \abs{V\ph(x)}^2\dx \le
\underbrace{\sup_{x\in\R^n} \abs{\ph(x)}^2}_{<\infty}
\underbrace{\int\abs{V(x)}^2\dx}_{<\infty} <
\infty,
\end{align*}
d.h. $V\ph\in L^2(\R^n)$.

Dies gilt auch für $V\in L^2(\R^n) \oplus L^\infty(\R^n)$, d.h. $V =
V_2+V_\infty$, denn
\begin{align*}
V\ph = \underbrace{V_2\ph}_{\in L^2} + \underbrace{V_\infty\ph}_{\in L^2}.
\end{align*}
Das Coulomb-Potential $V(x) = -\frac{1}{\abs{x}}$ ist genau von dieser Form:
\begin{align*}
-\frac{1}{\abs{x}} = -\underbrace{\frac{1}{\abs{x}}\chi_{\abs{x}\le 1}}_{\in
L^2} - \underbrace{\frac{1}{\abs{x}}\chi_{\abs{x}>1}}_{\in L^\infty} \in
L^2(\R^3)\oplus L^\infty(\R^3).
\end{align*}
Somit ist $-\frac{1}{\abs{x}}\ph\in L^2(\R^3)$  für alle $\ph\in H^2(\R^3)$.
\maphere
\end{bemenum}
\end{bem*}

Zur Behandlung des Wasserstoffatoms benötigen wir jedoch noch weitere
Ergebnisse.

\begin{lem}
\label{prop:1.5}
Sei $V\in L^2(\R^n)$ und $\ep > 0$. Dann gibt es eine Zerlegung des Potentials
\begin{align*}
V = V_2 + V_\infty,\qquad V_2\in L^2(\R^n),\quad V_\infty\in L^\infty(\R^n)
\end{align*}
mit $\norm{V_2}_2 < \ep$.\fishhere
\end{lem}
\begin{proof}
Sei $\chi_N$ die charakteristische Funktion der Menge
\begin{align*}
\setdef{x\in\R^n}{\abs{V(x)}\le N},
\end{align*}
dann gilt $V = V(1-\chi_N) + V\chi_N$, wobei $V\chi_N \in L^\infty(\R^n)$ und
\begin{align*}
\norm{V(1-\chi_N)}^2 = \int_{\R^n}\abs{V(x)}^2\underbrace{(1-\chi_N(x))}_{\to
0}\dx \to 0,\quad N\to
\infty
\end{align*}
für fast alle $x$ nach dem Satz von Lebesgue, denn
\begin{align*}
\abs{V(x)}^2(1-\chi_N) \le \abs{V(x)}^2 \in L^1(\R^n).
\end{align*}
Wähle nun $N$ so groß, dass $\norm{V(1-\chi_N)} < \ep$ und setzte $V_2 :=
V(1-\chi_N)$ und $V_\infty := V\chi_N$.\qedhere
\end{proof}

\begin{thm}
\label{prop:1.6}
Sei $V\in L^2(\R^n)$ mit $n\le 3$. Für jedes $\ep > 0$ gibt es ein
$C_\ep\in\R$, so dass
\begin{align*}
\norm{V\ph} \le \ep \norm{\Delta \ph} + C_\ep\norm{\ph},\qquad
\forall \ph\in H^2(\R^n).\fishhere
\end{align*}
\end{thm}
\begin{proof}
Sei $V\in L^2(\R^n)$ und $\ep > 0$ und
\begin{align*}
V = V_2 + V_\infty 
\end{align*}
die Zerlegung nach Lemma \ref{prop:1.5}. Dann gilt für $\ph\in H^2(\R^n)$
\begin{align*}
\norm{V\ph} &= \norm{V_2\ph + V_\infty \ph} \le
\norm{V_2\ph} + \norm{V_\infty\ph}
\le
\norm{V_2}\norm{\ph}_\infty + \norm{V_\infty}_\infty\norm{\ph}\\
&< \ep \norm{\ph}_\infty + \norm{V_\infty}_\infty \norm{\ph}.
\end{align*}
Wobei eine leichte Anwendung des Sobolev-Lemmas zeigt
\begin{align*}
\norm{\ph}_\infty \le c\norm{\ph}_{H^2}
\le C\left(\norm{\Delta \ph} + \norm{\ph}\right).
\end{align*}
Somit folgt
\begin{align*}
\norm{V\ph} &\le \ep\left(C\norm{\Delta \ph}  + C\norm{\ph}\right)
+ \norm{V_\infty}_\infty\norm{\ph}\\
&= \ep C\norm{\Delta \ph} + (\ep C + \norm{V_\infty}_\infty)\norm{\ph}.\qedhere
\end{align*}
\end{proof}

\begin{defn*}
Für $u\in H^s(\R^n)$ und $\alpha\in\N_0^n$ mit $\abs{\alpha} \le s$ definiert
man nun\index{schwache Ableitung}
\begin{align*}
\partial^\alpha u:= \FF^{-1}((ip)^\alpha \hat{u}).
\end{align*}
$\partial^\alpha u$ heißt \emph{schwache Ableitung} von $u$.\fishhere
\end{defn*}

Wenn $\abs{\alpha} < s-n/2$, stimmt nach Theorem \ref{prop:1.4} die schwache
Ableitung mit der partiellen Ableitung  $\partial^\alpha$ überein.

\begin{lem}
\label{prop:1.7}
Sei $\alpha\in\N_0^n$ mit $\abs{\alpha}\le s$. Für $u\in H^s(\R^n)$ gilt
$\partial^\alpha u \in H^{s-\abs{\alpha}}(\R^n)$ und
\begin{align*}
\norm{\partial^\alpha u}_{s-\abs{\alpha}} \le \norm{u}_s.\fishhere
\end{align*}
\end{lem}
\begin{proof}
Blatt 1.\qedhere
\end{proof}

$\partial^\alpha : H^s(\R^n)\to H^{s-\abs{\alpha}}(\R^n)$ ist somit eine
beschränkte lineare Abbildung.

\begin{lem}
\label{prop:1.8}
Sei $f\in C^m(\R^n)$ beschränkt mit beschränkten partiellen Ableitungen bis zu
Ordnung $m$. Dann ist $f\ph\in H^m(\R^n)$ für alle $\ph\in H^m(\R^n)$,
$\norm{f\ph}_m \le C_f\norm{\ph}_m$ und die partiellen Ableitungen von $f\ph$
lassen sich mit der Leibniz-Regel berechnen
\begin{align*}
\partial^\alpha (f\ph) = \sum_{\beta\le \alpha}\partial^\beta
f\partial^{\alpha-\beta} \ph.\fishhere
\end{align*}
\end{lem}
\begin{proof}
Wir setzten für diesen Beweis
\begin{align*}
\norm{\ph}_m := \sum_{\abs{\alpha}\le m} \norm{\partial^\alpha \ph}_{L^2}.
\end{align*}
Diese Norm ist äquivalent zur $\norm{\cdot}_s$-Norm.

Sei zuerst $\ph\in C_0^\infty(\R^n)$. Dann ist $f\ph \in C_0^m(\R^n)\subset
H^m(\R^n)$ und
\begin{align*}
\partial^\alpha(f\ph) = \sum_{\beta\le \alpha} \binom{\alpha}{\beta}
\partial^\beta f\partial^{\alpha-\beta} \ph.
\end{align*}
Daraus folgt
\begin{align*}
\norm{\partial^\alpha (f\ph)}_{L^2} \le
\sum_{\beta\le \alpha} \binom{\alpha}{\beta}
\underbrace{\norm{\partial^\beta
f}_{L^2}}_{<\infty}\underbrace{\norm{\partial^{\alpha-\beta}\ph}_{L^2}}_{\le
\norm{\ph}_m} \le C_{\alpha,f}\norm{\ph}_m
\end{align*}
und somit
\begin{align*}
\norm{f\ph}_m = \sum_{\abs{\alpha}\le m}\norm{\partial^\alpha (f\ph)}_{L^2} \le
\sum_{\abs{\alpha}\le m} C_{\alpha,f}\norm{\ph}_m \overset{*}{\le} C_f
\norm{\ph_m}.
\end{align*}

Da $C_0^\infty(\R^n)$ dicht in $H^m(\R^n)$, existiert zu jedem $\ph\in
H^m(\R^n)$ eine Folge $(\ph_n)\in C_0^\infty(\R^n)$ mit $\norm{\ph_n-\ph}_m \to
0$ für $n\to\infty$. Mit (*) folgt nun
\begin{align*}
\norm{f(\ph_k-\ph_l)} \le C_f\norm{\ph_k-\ph_l}_m \to 0,\qquad k,l\to\infty,
\end{align*}
also ist $(f\ph_k)$ eine Cauchyfolge in $H^m(\R^n)$. Da $H^m(\R^n)$
vollständig, existiert ein $\psi\in H^m(\R^n)$ mit
\begin{align*}
\norm{f\ph_k - \psi}_m \to 0.
\end{align*}
Andererseits gilt $\ph_k \to \ph$ in $L^2(\R^n)$ und $f\in L^\infty(\R^n)$ also
\begin{align*}
f\ph_k \to f\ph,\qquad \text{ in } L^2(\R^n).
\end{align*}
Folglich ist $\psi = f\ph$ in $L^2(\R^n)$ und damit insbesondere in
$H^m(\R^n)$, d.h.
\begin{align*}
f\ph_k \to f\ph,\qquad \text{in } H^m(\R^n).
\end{align*}
Somit lässt sich (*) auf $H^m(\R^n)$ übetragen.

Die Produktregel folgt nun aus
\begin{align*}
\partial^\alpha (f\cdot \ph_n) = \sum_{\beta\le \alpha} \partial^\beta f
\partial^{\alpha-\beta}\ph_n
\end{align*}
im Limes für $n\to\infty$, wegen der Stetigkeit der Multiplikation mit $f$ und
der Differentiation (Lemma \ref{prop:1.7}).\qedhere
\end{proof}

\begin{thm}
\label{prop:1.9}
Sei $V:\R^n\to\C$ messbar, $E\in\C$ und $\ph\in H^2(\R^n)$ mit
\begin{align*}
-\Delta \ph + V\ph = E \ph.
\end{align*}
Ist $\Omega$ offen und $V\big|_\Omega \in C^m(\Omega)$ dann ist
\begin{align*}
\ph\big|_\Omega \in C^k(\Omega),\qquad \text{für }k< m+2-n/2.\fishhere
\end{align*}
\end{thm}
\begin{proof}
Zu jedem Punkt $x\in\Omega$ existiert ein $\gamma\in C_0^\infty(\R^n)$ mit
$\supp(\gamma)\subset\Omega$ und $\gamma = 1$ in einer Umgebung von $x$.

Es genügt also zu zeigen, dass $\gamma\ph\in C^k(\R^n)$ für alle $\gamma\in
C_0^\infty(\Omega)$. Nach dem Sobolev-Lemma genügt es dazu, zu zeigen, dass
\begin{align*}
\gamma\ph\in H^{m+2}(\R^n),\qquad \gamma\in C_0^\infty(\Omega).
\end{align*}
\textit{Induktion über $m$}. ``$m=0$'': Wegen $\ph\in H^2(\R^n)$ ist
$\gamma\ph\in H^2(\R^n)$ für alle $\gamma\in C_0^\infty(\Omega)$.

Wir nehmen an, dass
$\gamma\ph\in H^{m+1}(\R^n)$ für alle $\gamma\in C_0^\infty(\Omega)$ und zeigen,
dass $\gamma\ph\in H^{m+2}(\R^n)$ für alle $\gamma\in C_0^\infty(\Omega)$. Es
gilt
\begin{align*}
\Delta(\gamma\ph) &= (\Delta \gamma)\ph + 2\nabla\gamma\nabla \ph +
\gamma\Delta\ph\\
&= (\Delta\gamma)\ph + 2\div((\nabla\gamma)\ph) - 2(\Delta\gamma)\gamma +
\gamma\Delta\ph\\
&= -(\Delta\gamma)\ph + 2\div((\nabla\gamma)\ph) + (V-E)\gamma\ph
\end{align*}
Nach Induktionsannahme gilt
\begin{align*}
(\Delta\gamma)\ph,\;(\nabla\gamma)\ph,\;\gamma\ph\in H^{m+1}(\R^n).
\end{align*}
Nach Annahme über $V$ und nach Satz \ref{prop:1.8} ist $(V-E)\gamma\ph\in
H^m(\R^n)$ und auch $\div((\nabla\gamma)\ph)\in H^m(\R^n)$. Also auch
$\Delta(\gamma\ph)\in H^m(\R^n)$. Daraus folgt, dass $\gamma\ph\in
H^{m+2}(\R^n)$, denn
\begin{align*}
&\int_{\R^n} \abs{\widehat{\gamma\ph}(p)}^2 (1+p^2)^{m+2}\ddp
= \int_{\R^n}
\abs{\widehat{\gamma\ph}(p)}^2(1+p^2){m}\underbrace{(1+p^2)^2}_{\le 2 +
2(p^2)^2}\ddp\\
&\qquad\le
2\norm{\gamma\ph}_m^2 + 2\int_{\R^n}
\abs{p^2\widehat{\gamma\ph}(p)}^2(1+p^2)^m\ddp\\
&\qquad < \infty.\qedhere
\end{align*}
\end{proof}
\begin{cor}
\label{prop:1.10}
Sei $V:\R\to\C$ messbar, $E\in \C$ und $\ph\in H^2(\R^n)$ mit
\begin{align*}
-\ph'' + V\ph= E\ph. 
\end{align*}
Ist $I\subset\R$ ein offenes Intervall und $V\big|_I\in C^m(I)$, dann ist
$\ph\big|_I \in C^{m+2}(I)$.\fishhere
\end{cor}
\begin{proof}
Nach Theorem \ref{prop:1.9} ist $\ph\big|_I\in C^{m+1}(I)$. Wegen
\begin{align*}
\ph'' = (V-E)\ph \in C^m(I)
\end{align*}
ist $\ph\in C^{m+1}(I)$. Details für $m=0$ siehe Übungsaufgabe 2.4.\qedhere
\end{proof}



\chapter{Exponentieller Abfall von Eigenfunktionen}

Sei $\ph\in L^2(\R)$ eine $C^2$-Lösung der Schrödingergleichung
\begin{align*}
-\ph''  + V\ph = E\ph,
\end{align*}
wobei $V(x)>E$ für $x> R$.

%%%%%%%%%%%%%%%%%%%%%%%%%%%%%%%%%%%%%%%%%%%%%%%%%%%%%%%%---Bild 1---%%%%%%%%%%%%%%%%%%%%%%%%%%%%%%%%%%%%%%%%%%%%%%%%%%%%%%%%%%%%%%%%%%%
%%%%%%%%%%%%%%%%%%%%%%%%%%%%%%%%%%%%%%%%%%%%%%%%%%%%%%%%%%%%%%%%%%%%%%%%%%%%%%%%%%%%%%%%%%%%%%%%%%%%%%%%%%%%%%%%%%%%%%%%%%%%%%%%%%%%%%%
\begin{figure}[!htpb]
\centering
\begin{pspicture}(-1.5,-3.5)(8,2.5) 
 \psline{->}(-1,0)(8.5,0)
 \psline{->}(0,-3)(0,1.8)
 \psline[linestyle=dashed](4.1933,-0.7)(4.1933,1.8)
 \psline[linecolor=accentc](0,-0.4047)(6,-0.4047)

 \psplot[linewidth=1.2pt,plotpoints=200,linecolor=accent,algebraic=true]{0.47}{8}%
 	{(-22.5*x^2+11.25*x^3)*2.71828^(-1.5*x)+0.02}
 \psplot[linewidth=1.2pt,linecolor=accent,algebraic=true]{1.6}{8}% 
 	{-4.25*(2.71828^(-.143*x^2))-0.06}
 	
 \rput(-0.19,-0.38){\color{accentc}$E$}
 \rput(1.9,-2.956){\color{darkgray}$V$}
 \rput(2.1,1){\color{darkgray}$\ph$}
 \rput(6,-0.9){\color{darkgray}$V>E$}
 \rput(6.2,1.43){\color{darkgray}exponentieller Abfall}
 \pscurve[linecolor=gray](5.8,1.2)(5.35,0.8)(5.9,0.75)(5.6,0.4)
\end{pspicture}
\caption{$-\ph^{\prime\prime}+V\ph=E\ph$ mit $\ph\in L^2(\R)$}
\end{figure}
%%%%%%%%%%%%%%%%%%%%%%%%%%%%%%%%%%%%%%%%%%%%%%%%%%%%%%%%%%%%%%%%%%%%%%%%%%%%%%%%%%%%%%%%%%%%%%%%%%%%%%%%%%%%%%%%%%%%%%%%%%%%%%%%%%%%%%%%
Es gilt
\begin{enumerate}
\item[] $V(x)<E \quad\Rightarrow\quad  \ph^{\prime\prime}(x),\ph(x)$ haben verschiedenes Vorzeichen,
\item[] $V(x)>E \quad\Rightarrow\quad  \ph^{\prime\prime}(x),\ph(x)$ haben gleiches Vorzeichen.
\end{enumerate}
Im Bereich $x>R$ ist $V(x)>E$, also muss $\ph$ dort strikt konvex sein wenn $\ph(x)>0$, und strikt konkav wenn $\ph(x)<0$.
%%%%%%%%%%%%%%%%%%%%%%%%%%%%%%%%%%%%%%%%%%%%%---Bild 2---%%%%%%%%%%%%%%%%%%%%%%%%%%%%%%%%%%%%%%%%%%%%%%%%%%%%%%%%%%%%%%%%%%%%%%%%%%%%%%%
%%%%%%%%%%%%%%%%%%%%%%%%%%%%%%%%%%%%%%%%%%%%%%%%%%%%%%%%%%%%%%%%%%%%%%%%%%%%%%%%%%%%%%%%%%%%%%%%%%%%%%%%%%%%%%%%%%%%%%%%%%%%%%%%%%%%%%%%
\begin{figure}[!htpb]
\centering
\begin{pspicture}(-1,-2)(6.4,3)
 \psline[arrowsize=4pt]{->}(0,-1)(0,2)
 
 \pscurve[linecolor=accentb,linewidth=1.2pt](0,1)(1,0.5)(5,2)
 
 \psplot[linewidth=1.2pt,linecolor=accent,algebraic=true]{0}{5}%
 	{2.71828^(-x)}
 \psplot[linewidth=1.2pt,linecolor=accentb,algebraic=true]{1}{4}%
 	{ln(-0.25*(x-5))}
 \psplot[linewidth=1.2pt,linecolor=accentb,algebraic=true]{0}{1}%
	{-0.78125*x^3+2.3125*x^2-2.53125*x+1}
  \psline[arrowsize=4pt]{->}(-2,0)(6,0)
  
  \rput(-0.3,-0.3){\color{darkgray}$R$}
  \rput(5.6,0.3){\color{darkgray}$\ph\in L^2$}
  \rput(5.7,2.1){\color{darkgray}$\ph\not\in L^2$}
\end{pspicture}
\caption{Möglicher Verlauf von $\ph$ für $x>R$.}
\end{figure}
Wegen $\ph\in L^2(\R)$ kommt nur der (exponentielle) Abfall
$\lim_{|x|\to\infty}\ph(x)=0$ in Frage. Wir beweisen ein analoges Resultat im $\R^n$.

\begin{prop}[IMS-Formel]
\label{prop:2.1}
Sei $H=-\Delta : H^2(\R^n)\to H^2(\R^n)$ aufgefasst als linearer Operator und
$f\in C^{\infty}(\R^n\to\R)$ mit $\partial^{\alpha}f\in L^{\infty}(\R^n)$ für $|\alpha|\leq 2$. Dann gilt
\begin{align*}
   fHf = \frac{1}{2}\left(f^2 H + H f^2\right) + |\nabla f|^2.
\end{align*} 
auf $H^2(\R^n)$, wobei $f$, $f^2$ und $\abs{\nabla f}^2$ als
Multiplikationsoperatoren aufzufassen sind.\fish
\end{prop}

\begin{bem*}
Der Satz gilt unverändert auch für
\begin{align*}
H = -\Delta + V,\qquad V: \R^n\to \C,
\end{align*}
wenn $V: H^2(\R^n)\to L^2(\R^n)$.\map
\end{bem*}

\begin{proof}
Nach Satz \ref{prop:1.8} ist $H^2(\R^n)$ invariant unter $f$ und $f^2$
und somit gilt auf $H^2(\R^n)$,
\begin{align*}
 fHf &= f^2H+f[H,f]\tag{1}\\
 fHf &= Hf^2-[H,f]f\tag{2}
\end{align*}
wobei für alle $\ph \in H^2(\R^n)$ gilt
\begin{align*}
[H,f]\ph &= (Hf-fH)\ph\\
  &= -\Delta(f\ph)+f\Delta\ph\\
  &= (-\Delta f)\ph-2(\nabla f)\cdot\nabla\ph.
\end{align*}
Also ist
\begin{align*}
 [H,f]f\ph &= (-\Delta f)f\ph-2(\nabla f)\cdot\nabla(f\ph)\\
    &= (-\Delta f)f\ph-2f(\nabla f)\cdot\nabla\ph-2|\nabla f|^2\ph\\
    &= f[H,f]\ph-2|\nabla f|^2\ph.\tag{3}
\end{align*}
Nach (3) gilt
\begin{align*}
f[H,f] - [H,f]f = 2\abs{\nabla f}^2.\tag{4} 
\end{align*}
Aus (1), (2) und (4) folgt nun die Behauptung nach Addition von (1) und
(2).\qed
\end{proof}

\begin{defn*}
Sei $V:\R^n\to\R$ messbar mit $V\ph\in L^2(\R^n)$ für alle $\ph\in H^2(\R^n)$.
Wir definieren die \emph{Ionisierungsschwelle}\index{Ionisierungsschwelle}
$\Sigma\le \infty$ von $H=-\Delta+V$ durch
\begin{align*}
&\Sigma = \lim\limits_{R\to\infty} \Sigma_R,\\
&\Sigma_R = \inf\setdef{\lin{\ph,H\ph}}{\ph\in H^2(\R^n),\; \norm{\ph} = 1,\;
\ph(x) = 0\text{ für } \abs{x}< R}.\fish
\end{align*}
\end{defn*}

Anschaulich beschreibt die Ionisierungsschwelle die kleinste
Energie des Atoms mit einem Elektron weniger. In den einfachsten Fällen ist
$\Sigma=\lim\limits_{\abs{x}\to\infty} V(x)$, beim Potential $V=x^2$ ist
beispielsweise $\Sigma=\infty$.

\begin{thm}
\label{prop:2.2}
Sei $V:\R^n\to\R$ messbar mit $V\ph\in C^2(\R^n)$ für alle $\ph\in H^2(\R^n)$.
Ist $\ph\in H^2(\R^n)$ und $H\ph = E\ph$, wobei $E< \Sigma$, dann ist
$e^{\beta\abs{x}}\ph(x)\in L^2(\R^n)$ für alle $\beta\in\R$, so dass
$E+\beta^2< \Sigma$.\fish
\end{thm}
\begin{proof}
Sei $\chi\in C_0^\infty(\R^n\to[0,1])$ mit
\begin{align*}
\chi(x) = 
\begin{cases}
1, & \abs{x}\ge 2,\\
0, & \abs{x} \le 1
\end{cases}
\end{align*}
und sei $\chi_R(x) := \chi(R^{-1}x)$ für $R> 0$.

\begin{figure}[!htpb]
\begin{center}
\begin{pspicture}(-2,-1.5)(8,2)
 \psline{->}(-1.5,0)(7,0)
 \psline{->}(0,-1)(0,2)
 \psplot[linewidth=1.18pt,linecolor=accent,algebraic=true]{2}{4}%
 	{0.1875*(x-2)^5-0.9375*(x-2)^4+1.25*(x-2)^3}
 \psline[linewidth=1.18pt,linecolor=accent](-0.75,0)(2,0)
 \psline[linewidth=1.18pt,linecolor=accent](4,1)(6.5,1)
 \rput(2,-0.3){\color{darkgray}$R$}
 \rput(3.9,-0.3){\color{darkgray}$2R$}
 \rput(-0.2,1){\color{darkgray}$1$}
 \rput(5,1.35){\color{darkgray}$\chi_R$}
 \psline(0,1)(-0.08,1)
 \psline(2,0)(2,-0.08)
 \psline(3.9,0)(3.9,-0.08)
\end{pspicture}
\end{center}
\caption{Zur Abschneidefunktion.}
\end{figure}

Für $0<\ep < 1$ definieren wir
\begin{align*}
f(x) = \frac{\beta\abs{x}}{1+\ep\abs{x}}.
\end{align*}
Dann gilt $\abs{f} \le \ep^{-1}\beta$ und $\abs{\nabla f} \le \beta$
für $\abs{x}> 0$. Sei $G=\chi_R e^f$, dann ist $G\in L^\infty\cap
C^\infty(\R^n)$ und $\partial^\alpha G\in L^\infty(\R^n)$ für $\abs{\alpha}\le
2$.

Aus der IMS-Formel folgt nun
\begin{align*}
\lin{G\ph,(H-E)G\ph} &= \lin{\ph, G(H-E)G\ph}
= \frac{1}{2}\lin{\ph,G^2(H-E)\ph} \\ &+ \frac{1}{2}\lin{\ph,(H-E)G^2\ph}
+ \lin{\ph,\abs{\nabla G}^2\ph}
\end{align*}
Da $(H-E)$ ein symmetrischer Operator auf $L^2$ ist, verschwinden die ersten
beiden Terme. Zum dritten Term betrachte
\begin{align*}
\nabla G &= \nabla \chi_R e^f + \chi_R e^f \nabla f 
=  \nabla \chi_R e^f + G f,\\
\Rightarrow\;\abs{\nabla G} &= \nabla \chi_R^2 e^{2f} + 2\nabla \chi_R \nabla f
e^f G + G^2\abs{\nabla f}^2.
\end{align*}
Die ersten beiden Terme sind in der $L^\infty$-Norm durch eine von
$\ep$ unabhängige Konstante $C_R$  beschränkt, denn $\nabla \chi_R$ hat
kompakten Träger. Es folgt
\begin{align*}
\lin{G\ph,(H-E-\abs{\nabla f}^2)G\ph} \le C_R\norm{\ph}^2,\tag{1}
\end{align*}
und andererseits
\begin{align*}
\lin{G\ph,(H-E-\abs{\nabla f}^2)G\ph} &\ge \lin{G\ph,(H-E-\beta^2)G\ph}
\\ &\ge (\Sigma_R-E-\beta^2)\norm{G\ph}^2.\tag{2}
\end{align*}
Da $\Sigma_R\to\Sigma> E+\beta^2$ existiert ein $R> 0$, so dass $\Sigma_R >
E+\beta^2$. Für dieses $R$ folgt aus (1) und (2), dass
\begin{align*}
\norm{G\ph}^2 \le \frac{C_R}{\Sigma_R-E-\beta^2} \norm{\ph}^2.
\end{align*}
Also gilt
\begin{align*}
\int_{\setd{x\ge 2R}} e^{2\beta\abs{x}}\abs{\ph(x)}^2\dx
&=
\int_{\setd{x\ge 2R}} \lim\limits_{\ep \to 0} e^{2f(x)}\abs{\ph(x)}^2\dx\\
&\overset{\text{mon.konv.}}{=}
\lim\limits_{\ep \to 0}
\int_{\setd{x\ge 2R}}  e^{2f(x)}\abs{\ph(x)}^2\dx\\
&\le
\lim\limits_{\ep\to 0}
\norm{G\ph}
\le
\frac{C_R}{\Sigma_R-E-\beta^2} \norm{\ph}^2
< \infty.
\end{align*}
Andererseits ist auch
\begin{align*}
\int_{\setd{x< 2R}} e^{2\beta\abs{x}}\abs{\ph(x)}^2\dx
\le
e^{4\beta R}\norm{\ph}^2 < \infty.\qed
\end{align*}
\end{proof}

\begin{prop}
\label{prop:2.3}
Sei $V:\R^n\to\R$ messbar mit $V\ph\in C^2(\R^n)$ für alle $\ph\in H^2(\R^n)$
und $\ph\in H^2(\R^n)$ mit $H\ph = E\ph$. Sei
\begin{align*}
\sup\limits_{x\in\R^n} \int_{\abs{y-x}\le 1}
\abs{V_-(x)}^p\dy < \infty,
\end{align*}
für ein $p>\frac{n}{2}$ und $n\ge 2$. Dann gilt für alle $x\in\R^n$
\begin{align*}
\esssup\limits_{y:\abs{y-x}\le \frac{1}{2}} \abs{\ph(y)} \le
C\norm{\ph}_{L^2(B_1(x))}.\fish
\end{align*}
\end{prop}
\begin{proof}
Siehe Übungsblatt 3.\qed
\end{proof}

\begin{cor}
\label{prop:2.4}
Sei $V:\R^n\to\R$ messbar mit $V\ph\in C^2(\R^n)$ für alle $\ph\in H^2(\R^n)$
und $\ph\in H^2(\R^n)$ mit $H\ph = E\ph$. Dann gibt es zu jedem $\beta\in\R$
mit $E+\beta^2<\Sigma$ ein $C_\beta \in\R$, so dass
\begin{align*}
\abs{\ph(x)} \le C_\beta e^{-\beta\abs{x}}\quad \text{f.ü.}\fish
\end{align*}
\end{cor}
\begin{proof}
Für jedes $x\in\R^n$ gilt
\begin{align*}
\esssup\limits_{y: \abs{x-y}\le \frac{1}{2}}
e^{\beta\abs{y}}\ph(y)
&\le
e^{\beta \abs{x}}
e^{\beta/2}
\esssup\limits_{y: \abs{x-y}\le \frac{1}{2}}
\ph(y)
\le
e^{\beta \abs{x}}
e^{\beta/2}
C\norm{\ph}_{L^2(B_1(x))}\\
&\le
C\norm{e^{\beta\abs{\cdot}}\ph}_{L^2(B_1(x))}
e^{3/2\beta}
\le
C\norm{e^{\beta\abs{\cdot}}\ph}
e^{3/2\beta}
< \infty.\qed
\end{align*}
\end{proof}

\begin{bem*}
\begin{bemenum}
\item Mit den Methoden des Beweises von Theorem \ref{prop:2.2} lassen sich für
$n$-Teilchensysteme verbesserte, anisotrope Schranken herleiten.
\item Solche Schranken existieren nicht nur für Schrödinger-Operatoren, sondern
auch für Operatoren die ein Teilchen in einem quantisierten Strahlungsfeld
beschreiben,
\begin{align*}
H = (-i\nabla_x  + A(x))^2 + H_f + V,
\end{align*}
wobei $A(x)$ das quantisierte Vektorpotential darstellt und $H_f$ die
Feldenergie.\map
\end{bemenum}
\end{bem*}
% 
% \noindent
% \emph{Bemerkung.} Nach Satz 8.8 ist $V$ relativ kompakt 
% bezüglich $-\Delta/2$ und somit ist $H$ selbstadjungiert mit $D(H)=H^2(\R^n)$ und
% $\sigma_{\rm ess}(H)=[0,\infty)$ nach Theorem 8.12.
% 
% \begin{bew}
% Sei $\chi\in C^{\infty}(\R^n,[0,1])$ mit
% \[\chi(x)=
% \begin{cases}
%  1 & |x|\geq 2,\\
%  0 & |x|\leq 1.
% \end{cases}\]
% Sei $\chi_R(x)=\chi(x/R)$ und $f(x)=\frac{\beta|x|}{1+\eps|x|}$, $\eps>0, R>0$.
% %%%%%%%%%%%%%%%%%%%%%%%%%%%%---Bild 3---%%%%%%%%%%%%%%%%%%%%%%%%%%%%%%%%%%%%%%%%%%%%%%%%%%%%%%%%%%%%%%%%%%%%%%%%%%%%%%%%%%%%%%%%%%%%%%%%%%%%%
% %%%%%%%%%%%%%%%%%%%%%%%%%%%%%%%%%%%%%%%%%%%%%%%%%%%%%%%%%%%%%%%%%%%%%%%%%%%%%%%%%%%%%%%%%%%%%%%%%%%%%%%%%%%%%%%%%%%%%%%%%%%%%%%%%%%%%%%%%%%%%
% \begin{center}
% \begin{pspicture}(-2,-1.5)(8,2)
%  \psline[linewidth=0.5pt,arrowsize=4pt]{->}(-1.5,0)(7,0)
%  \psline[linewidth=0.5pt,arrowsize=4pt]{->}(0,-1)(0,2)
%  \psplot[linewidth=1.2pt,algebraic=true]{2}{4}{0.1875*(x-2)^5-0.9375*(x-2)^4+1.25*(x-2)^3}
%  \psline[linewidth=1.18pt](-0.75,0)(2,0)
%  \psline[linewidth=1.18pt](4,1)(6.5,1)
%  \rput(2,-0.3){$R$}
%  \rput(3.9,-0.3){$2R$}
%  \rput(-0.2,1){$1$}
%  \rput(5,1.35){$\chi_R$}
%  \psline[linewidth=0.5pt](0,1)(-0.08,1)
%  \psline[linewidth=0.5pt](2,0)(2,-0.08)
%  \psline[linewidth=0.5pt](3.9,0)(3.9,-0.08)
% \end{pspicture}
% \end{center}
% %%%%%%%%%%%%%%%%%%%%%%%%%%%%%%%%%%%%%%%%%%%%%%%%%%%%%%%%%%%%%%%%%%%%%%%%%%%%%%%%%%%%%%%%%%%%%%%%%%%%%%%%%%%%%%%%%%%%%%%%%%%%%%%%%%%%%%%%%%%%%
% Es gilt $|f|\leq\beta/\eps$ und für $x\neq0$
% \[
%   |\nabla f(x)|\leq \beta.
% \]
% Sei $G=\chi_Re^f.$ Dann $G\in C^{\infty}(\R^n)\cap L^{\infty}(\R^n)$
% \begin{eqnarray*}
%  \nabla G &=& \nabla\chi_Re^f+\chi_Re^f\nabla f\\
%    &=& \nabla\chi_Re^f+G\nabla f\quad\in L^{\infty}
% \end{eqnarray*}
% und $\partial_i\partial_jG\in L^{\infty}(\R^n)$ für alle $i,j$. Also nach Satz 2
% \begin{eqnarray*}
%  \sprod{G\ph}{(H-E)G\ph} &=& \sprod{\ph}{|\nabla G|^2\ph}\frac{1}{2}\\
%    &=& \sprod{\ph}{(\nabla\chi_Re^f+G\nabla f)^2\ph}\frac{1}{2}\\
%    &=& \frac{1}{2}\sprod{\ph}{\left(|\nabla\chi_R|^2e^{2f}+2\nabla\chi_R\nabla fe^fG+G^2|\nabla f|^2\right)\ph}
% \end{eqnarray*}
% Somit
% \begin{eqnarray}
%  \sprod{G\ph}{\left(H-E-|\nabla f|^2/2\right)G\ph} &\leq&%
%   \sup_x\left(|\nabla\chi_R|e^{2\beta|x|}+2|\nabla\chi_R|\beta e^{2\beta|x|}\right)\|\ph\|^2 \nonumber\\
%      &=& C_R\|\ph\|^2 \label{p:9.3:1}
% \end{eqnarray}
% wobei $C_R<\infty$. Andererseits
% \begin{eqnarray}
%  \sprod{G\ph}{\left(H-E-|\nabla f|^2/2\right)G\ph} &\geq& \sprod{G\ph}{\left(V-E-\beta^2/2\right)G\ph}\nonumber\\
%   &\geq& \left[\left(\inf_{|x|\geq R}V(x)\right)-E-\beta^2/2\right]\|G\ph\|^2 \label{p:9.3:2}\\
%   &=:& \delta_R\|G\ph\|^2.\nonumber
% \end{eqnarray}
% Nach Voraussetzung existiert $R>0$ hinreichend groß, so dass $\delta_R>0$. Halte dieses $R$ fest. 
% Dann folgt aus \eqref{p:9.3:1} und \eqref{p:9.3:2}
% \[
%   \|G\ph\|^2\leq \frac{C_R}{\delta_R}\|\ph\|^2
% \]
% für alle $\eps>0$. Also
% \begin{eqnarray*}
%  \int e^{2\beta|x|}|\ph(x)|^2\:dx &=& \int\lim_{\eps\to0}e^{2f(x)}|\ph(x)|^2\:dx\\
%    &=& \lim_{\eps\to0}\int e^{2f}|\ph|^2\:dx~\leq~\frac{C_R}{\delta_R}\|\ph\|^2~<~\infty.
% \end{eqnarray*}
% \end{bew}
% 
% \noindent{\it Bemerkung.} Unter den Voraussetzungen des Theorems gibt es eine Konstante $C=C(V)$, 
% so dass für alle $x\in\R^n$
% \[
%   \|\ph\|_{L^{\infty}(B_1(x))}\leq C\|\ph\|_{L^2(B_1(x))}
% \]
% (Agmon: Thm. 5.1), somit folgt aus Theorem 3:
% \[
%   |\ph(x)|\leq C_{\beta}e^{-\beta|x|}
% \]
% für $\beta>0$ mit $E+\beta^2/2<0$.
% 
% \subsection*{Regularität der Eigenfunktionen}
% 
% Nach dem Sobolev-Lemma (F.A 5.2.2) ist $H^2(\R^n)\subset C^k(\R^n)$ falls $k<2-n/2$.
% Also sind für $n\leq3$ Eigenfunktionen stetig.
% 
% \begin{thm}
% Sei $H=-\Delta/2+V$ definiert auf $H^2(\R^n)$, $\ph\in H^2(\R^n)$ und
% $H\ph=E\ph$ für $E\in\C$. Falls $\Omega\subset\R^n$ offen ist und $V\restricted\Omega\in C^{\infty}(\Omega)$
% dann ist $\ph\in C^{\infty}(\Omega)$.
% \end{thm}
% 
% \noindent
% \emph{Bemerkung.} Falls $V\in C^{m}(\Omega)$ und $k<2+m-\frac{n}{2}$ dann 
% $\ph\in C^{k}(\Omega)$ (Reed \& Simon Kap. IX.6)
% 
% \begin{bew}
% Wir zeigen, dass $\gamma\ph\in\bigcap_{m\geq2}H^m(\R^n)$ für alle $\gamma\in C_0^{\infty}(\Omega)$.
% Dann folgt $\ph\in C^{\infty}(\Omega)$, denn $\bigcap_{m}H^m(\R^n)\subset C^{\infty}(\R^n)$ nach dem
% Sobolev-Lemma.
% 
% Da $\ph\in H^2(\R^n)$, ist auch $\gamma\ph\in H^2(\R^n)$ für alle $\gamma\in C_0^{\infty}(\Omega)$.
% Sei $m\geq2$ und sei $\gamma\ph\in H^m(\R^n)$ für alle $\gamma\in C_0^{\infty}(\Omega)$. Dann gilt für
% $\gamma\in C_0^{\infty}(\Omega)$
% \begin{eqnarray}
%  \Delta(\gamma\ph) &=& (\Delta\gamma)\ph+2\nabla\gamma\cdot\nabla\ph+\gamma(\Delta\ph)\nonumber\\
%    &=& (\Delta\gamma)\ph+2\nabla\gamma\cdot\nabla\ph+(V-E)\gamma\ph \label{p:9.4:1}
% \end{eqnarray}
% wobei
% \begin{equation}
%   \nabla\gamma\nabla\ph=\div((\nabla\gamma)\ph)-(\Delta\gamma)\ph \label{p:9.4:2}
% \end{equation}
% Nach Annahme sind $(\Delta\gamma)\ph,(V-E)\gamma\ph$ und $(\nabla\gamma)\ph$ in $H^m(\R^n)$.
% Also ist $\div((\nabla\gamma)\ph)\in H^{m-1}(\R^n)$ nach F.A Lemma 5.3.3. Aus \eqref{p:9.4:1} und \eqref{p:9.4:2}
% folgt nun, dass $\Delta(\gamma\ph)\in H^{m-1}(\R^n)$, d.h.
% \[
%   \int \abs{p^2\widehat{\gamma\ph}(p)}^2(1+p^2)^{m-1}\:dp<\infty
% \]
% und somit
% \[
%   \int \abs{\widehat{\gamma\ph}(p)}^2(1+p^2)^{m+1}\:dp<\infty.
% \]
% Also ist $\gamma\ph\in H^{m+1}(\R^n)$ für alle $\gamma\in C_0^{\infty}(\Omega)$.
% \end{bew}


\chapter{Spektrum und Resolvente}

Wir haben bisher Lösungen der Eigenwertgleichung des Operators
\begin{align*}
H = -\Delta + V : H^2(\R^n)\to H^2(\R^n)
\end{align*}
untersucht aber nicht geklärt, ob diese Lösungen überhaupt existieren. Um dies
zu klären, wollen wir die Spektraltheorie für unbeschränkte Operatoren auf
Hilberträumen entwickeln. Dazu betrachten wir allgemein Operatoren
\begin{align*}
A: D\subset X \to X
\end{align*}
und interessieren uns für deren Spektrum, dessen Approximation und Verhalten
unter Störungen.

\subsection{Grundlegende Definitionen}

%%-------------------------------Einfuehrung-Notationen-Definitionen-----------------
Sei $X$ ein komplexer Banachraum und $D \subset X$ ein linearer
Teilraum. 

\begin{defn*}
\begin{defnenum}
\item
Ein \emph{linearer Operator} in $X$,\index{Operator!linear}
\begin{align*}
A : D \subset X \to X,
\end{align*}
ist eine lineare Abbildung $A : D \to X$. $D$ heißt
\emph{Definitionsbereich} von $A$ (schreibe $D = D_A = D(A)$)
und
\begin{align*}
\ran (A) := AD =\setdef{Ax}{x \in D}
\end{align*}
heißt \emph{Wertebereich} von $A$ (range of $A$).
\item
Man sagt, $A$ sei \emph{dicht definiert}, falls\index{dicht definiert}
$\bar{D}=X$.
\item
$A$ heißt \emph{abgeschlossen}, falls der Graph von
$A$\index{Operator!abgeschlossen}
\begin{align*}
\Gamma_A:=\setdef{(x,y)}{x \in D, y=Ax}
\end{align*}
abgeschlossen ist in $X \oplus X$.\fishhere
\end{defnenum}
\end{defn*}

\begin{lem*}
Der Operator $A$ ist
genau dann abgeschlossen, wenn
\begin{align*}
\begin{rcases}
x_n \in D,\, x_n \to x \in X\\
Ax_n\to y
\end{rcases}
\Rightarrow x \in D(A)\text{ und } Ax=y.\fishhere
\end{align*}
\end{lem*}

\begin{defn*}
Ein zweiter Operator, $B : D(B) \subset X \to X$ heißt
\emph{Erweiterung}\index{Operator!Erweiterung}
von $A$, falls $D(A) \subset D(B)$ und $Ax=Bx$ für alle $x \in D(A)$.
Man schreibt dafür $A\subset B$.
Zwei Operatoren sind gleich, $A=B$, falls $D(A)=D(B)$ und
$Ax = Bx$ für alle $x \in D(A)$.\fishhere
\end{defn*}

Den Identitätsoperator $\Id : X \to X,\, x \mapsto x$ lassen
wir meist weg.
Zum Beispiel:
\begin{align*}
\lambda-A := \lambda\Id - A \text{ für } \lambda \in \C.
\end{align*}

\begin{defn*}
\begin{defnenum}
\item
Die \emph{Resolventenmenge} von $A$ ist die Menge\index{Resolventenmenge}
\begin{align*}
\rho(A) := \setdef{\lambda \in \C}{\atop{(\lambda-A) : D(A) \to X \text{
ist bijektiv und }}{(\lambda-A)^{-1} \text{ beschränkt}}}.
\end{align*}
\item
Die Abbildung
\begin{align*}
R(A): \rho(A) \to \LL(X),\qquad
\lambda  \mapsto  (\lambda-A)^{-1}
\end{align*}
heißt \emph{Resolvente}\index{Resolvente} von $A$. $R_{\lambda}(A)$ ist die
Resolvente von $A$ an der Stelle $\lambda$.
\item
Das Komplement der Resolventenmenge:
\begin{align*}
  \sigma(A) := \C \setminus \rho(A)
\end{align*}
heißt \emph{Spektrum}\index{Spektrum} von $A$. Die Menge
\begin{align*}
\sigma_p(A) := \setdef{\lambda \in \C}{\lambda \text{ ist
Eigenwert von A}}\subset \sigma(A)
\end{align*}
heißt \emph{Punktspektrum} von $A$.\index{Spektrum!Punkt-}\fishhere  
\end{defnenum}
\end{defn*}

\begin{bem*}
Wenn $A$ abgeschlossen und $(A-\lambda):D \to X$ bijektiv ist,
dann ist auch $(A-\lambda)^{-1} : X \to D$ abgeschlossen und somit nach
dem Graphensatz beschränkt (FA~Satz~4.3.5 \cite{Fun07}).
Das heißt, wenn $A$ abgeschlossen ist, dann gilt
\begin{align*}
\rho(A)=\setdef{\lambda \in \C}{(\lambda-A) : D(A) \to X \text{ ist
bijektiv}}.
\end{align*}
Wenn $A$ nicht abgeschlossen ist, dann gilt $\rho(A) = \varnothing$ und
$\sigma(A)=\C$.\maphere
\end{bem*}
%%--------------------------------------------------------------


%%----------------------------Beispiele-------------------------
\begin{bsp*}
\begin{bspenum}
\item
Sei $X = \C^{n}$ und $A \in \LL(X)$, dann gilt: $\sigma(A) = \sigma_p(A) \ne
\varnothing$ (Eigenwerte der Matrix bezüglich einer Basis von X).
\item 
Sei $X = L^2(\R)$, $D = \setdef{f \in L^2(\R)}{\int \abs{xf(x)}\dx <
\infty}$ und
\begin{align*}
Af(x) := xf(x).
\end{align*}
Dann gilt: $\sigma(A) = \R$, $\sigma_p(A) = \varnothing$ und 
$\rho(A) = \C \setminus \R$.
\item
Sei $X = C(I)$, $I=[0,1]$, $\norm{f} = \sup \limits_{x \in I} \abs{f(x)}$, 
und sei
\begin{align*}
A_k : D_k \subset X \to X,\qquad (A_kf)(x) = \frac{d}{dx}f(x)
\end{align*}
wobei $k \in \setd{1,\ldots,4}$ und
\begin{align*}
  D_1 & =  C^1(I)  \\
  D_2 & =  \setdef{f \in D_1}{f(0)=0}  \\
  D_3 & =  \setdef{f \in D_1}{f(0)=f(1)} \qquad \text{(Periodische
  Randbedingung)}\\
  D_4 & =  \setdef{f \in D_1}{f(0)=0=f(1)}\  \quad \text{(Dirichlet
  Randbedingung)}
\end{align*}
Alle $A_k$ sind abgeschlossen und es gilt:
\begin{align*}
  \sigma(A_1) & =  \sigma_p(A_1)  =  \C,& \rho(A_1) &= \varnothing, \\
  \sigma(A_2) & =  \sigma_p(A_2)  =  \varnothing,&  \rho(A_2) &= \C, \\
  \sigma(A_3) & =  \sigma_p(A_3)  =  2 \pi i \Z,  \\
  \sigma(A_4) & =  \C,\ \sigma_p(A_4) = \varnothing,&\qquad
  \rho(A_4) &= \varnothing.
\end{align*}
\begin{proof}
\textit{$A$ ist abgeschlossen}.
Sei $(\ph_n)_{n \in \N} \subset D_1$ mit $\ph_n \to \ph$ und $\ph_n' \to \psi$ in $X$.
Dann gilt $\ph_n \to \ph$ gleichm\"a\ss{}ig, $\ph_n' \to \psi'$ gleichm\"a\ss{}ig.
Also ist $\ph \in C^1(I) = D_1$ und $\ph' = \psi$ (Analysis).
Falls $\ph_n \in D_k$, dann ist auch $\ph \in D_k$ f\"ur $k=2,3,4$.
Also ist $A_k$ abgeschlossen f\"ur $k \in \{1,\ldots ,4\}$.

``\textit{$\sigma(A_1) = \sigma_p(A_1) = \C$}''.
Sei $\lambda \in \C$, dann ist $x \mapsto e^{\lambda x}$ in $D_1$ und
\begin{align*}
\frac{\diffd}{\dx}e^{\lambda x} = \lambda e^{\lambda x}.
\end{align*}
Also ist $\lambda$ Eigenwert und daher
 $\sigma(A_1) = \sigma_p(A_1) = \C$.

``\textit{$\sigma(A_2)=\varnothing$}''.
Wir lösen $(A_2-\lambda) f = g$
für gegebenes $g \in X$ nach $f \in D_2$ auf. Das heißt, wir suchen
$f \in C^1(I)$ mit 
\begin{align*}
f'-\lambda f = g, \qquad f(0) = 0.
\end{align*}
Die eindeutige Lösung dieses Anfangswertproblems ist
\begin{align*}
f(x) = \int_{0}^{x} e^{\lambda(x-t)}g(t)\dt =: (Sg)(x).
\end{align*}
Somit ist $(A_2-\lambda):D_2\to X$ bijektiv mit Rechtsinverser $S$.
$S$ ist außerdem beschränkt, denn
\begin{align*}
\abs{(Sg)(x)} \le \int_0^1 \abs{e^{\lambda(x-t)}g(t)}\dt
\le c\int_0^1 \abs{g(t)}\dt \le c\norm{g}.
\end{align*}
Es bleibt noch zu zeigen, dass $(A_2-\lambda)g = g$ für $g \in D_2$. Sei
also $g \in D_2$, dann
\begin{align*}
S(A_2-\lambda)g(x) 
& = \int_{0}^{x} e^{\lambda(x-t)}(g'(t)-\lambda g(t))\dt \\
& = \left. e^{\lambda(x-t)} g(t) \right|_{0}^{x} 
+ \int_{0}^{x} \lambda e^{\lambda(x-t)}g(t)\dt\\ 
& \phantom{ = \left. e^{\lambda(x-t)} g(t) \right|_{0}^{x}\,}
- \int_{0}^{x} e^{\lambda(x-t)} \lambda g(t)\dt\\
& = g(x).
\end{align*}
Also $S = R_{\lambda}(A_2)$.\bsphere
\end{proof}
\end{bspenum}
\end{bsp*}
%%----------------------Beispielende--------------------------


%%----------------------Satz----------------------------------
\begin{prop}
\label{prop:3.1}
Sei $A: D \subset X \to X$ ein linearer Operator und
seien $\lambda,\mu \in \rho(A)$. Dann gilt:
\begin{propenum}
\item\label{prop:3.1:1} $R_{\lambda}(A)-R_{\mu}(A) = (\mu-\lambda)
R_{\lambda}(A)R_{\mu}(A)$,
\item\label{prop:3.1:2} $R_{\lambda}(A) R_{\mu}(A) = R_{\mu}(A) R_{\lambda}(A)$,
\item\label{prop:3.1:3} $R_{\lambda}(A) A \subset A R_{\lambda}(A)$,
\item\label{prop:3.1:4} $A R_{\lambda}(A) = -\Id + R_{\lambda}(A)$.\fishhere
\end{propenum}
\end{prop}

\begin{proof}
\begin{proofenum}
``\ref{prop:3.1:1}'':
Mit $\lambda,\mu \in \rho(A)$ gilt,
\begin{align*}
R_{\lambda}(A)-R_{\mu}(A) & = (\lambda-A)^{-1} - (\mu-A)^{-1}\\
& = (\lambda-A)^{-1} \left[(\mu-A)-(\lambda-A)\right] (\mu-A)^{-1}\\
& = (\lambda-A)^{-1} (\mu-\lambda) (\mu-A)^{-1}\\
& = (\mu-\lambda) R_{\lambda}(A)R_{\mu}(A)
\end{align*}

``\ref{prop:3.1:2}'':
folgt aus \ref{prop:3.1:1}.

``\ref{prop:3.1:3}, \ref{prop:3.1:4}'':
Für $x \in D$ gilt
\begin{align*}
R_{\lambda}(A) Ax & =
(\lambda-A)^{-1}Ax  = (\lambda-A)^{-1}(A-\lambda) + \lambda(\lambda-A)^{-1}\\
&= (-\Id+\lambda R_\lambda(A))x. 
\end{align*}
Außerdem gilt für alle $x \in X$,
\begin{align*}
A R_{\lambda}(A)x & = A(\lambda-A)^{-1}x
= (A-\lambda)(\lambda-A)^{-1}x + \lambda R_\lambda(A)x\\
&= (-\Id + R_\lambda(A))x. 
\end{align*}
Daraus folgen \ref{prop:3.1:3} und \ref{prop:3.1:4}.\qedhere
\end{proofenum}
\end{proof}
%%------------------------------------------------------------

%%-------------------------Theorem------------------------------
\begin{thm}
\label{prop:3.2}
Sei $A : D \subset X \to X$
ein linearer Operator. Dann ist die Resolventenmenge $\rho(A)$ offen,
das Spektrum $\sigma(A)$ abgeschlossen und die Resolvente $z \mapsto
(A-z)^{-1}$ ist analytisch auf $\rho(A)$.

Falls $z_0 \in \rho(A)$, dann ist
\begin{align*}
B(z_0,\norm{R_{z_0}(A)}^{-1}) \subset \rho(A)
\end{align*}
und für alle $z$ aus dieser Kreisscheibe gilt
\begin{align*}
R_z(A) = \sum_{n=0}^{\infty} (-1)^nR_{z_0}(A)^{n+1}(z-z_0)^n.
\end{align*}
Insbesondere gilt
$
\norm{R_{z_0}(A)} \geq \frac{1}{\dist(z_0,\sigma(A))}.\fishhere
$
\end{thm}

\begin{proof}
Sei $z_0 \in \rho(A)$ und $z \in \C$. Dann gilt
\begin{align*}
z- A& = (z_0-A) +  (z-z_0) = (\Id + (z-z_0)R_{z_0}(A))(z_0-A),
\end{align*}
wobei $(z_0-A) : D \to X$ bijektiv mit $(z_0-A)^{-1} \in
\LL(X)$. Falls $\abs{z-z_0} \cdot \norm{R_{z_0}(A)} < 1$, dann ist auch
\begin{align*}
\Id + (z-z_0) R_{z_0}(A) : X \to X\tag{1}
\end{align*}
bijektiv mit stetiger Inversen (FA~Thm~3.3.6 \cite{Fun07}).

Also ist $z\in\rho(A)$, falls $\abs{z-z_0} < \norm{R_{z_0}(A)}^{-1}$. Die
Inverse von (1) ist dann gegeben durch die Neumannreihe
\begin{align*}
\sum_{n\ge 0} (-1)^n R_{z_0}(A)^n(z-z_0)^n
\end{align*}
und folglich
\begin{align*}
\sum_{n\ge 0} (-1)^n R_{z_0}(A)^{n+1}(z-z_0)^n.
\end{align*}

Wenn $z\in\sigma(A)$, dann ist zwingend $\abs{z-z_0}\norm{R_{z_0}(A)} \ge 1$,
d.h. $\abs{z-z_0}\ge \norm{R_{z_0}(A)}^{-1}$ und folglich
\begin{align*}
\dist(z_0,\sigma(A)) = 
\inf\setdef{\abs{z-z_0}}{z\in\sigma(A)} \ge
\frac{1}{\norm{R_{z_0}(A)}}.\qedhere
\end{align*}
\end{proof}

%%---------------------------Korollar------------------------------------------
\begin{cor}
\label{prop:3.4}
Sei $A \in \LL(X)$. Dann ist $\sigma(A)$ kompakt und nichtleer.\fishhere
\end{cor}
\begin{proof}
\textit{$\sigma(A)$ ist kompakt}.
Sei $z\in \C$ mit $\abs{z}\ge \norm{A}$, dann ist $(z-A) = z(\Id-z^{-1}A)$ und
$\norm{z^{-1}A} < 1$. Somit besitzt $(z-A)$ eine beschränkte Inverse
(Neumannreihe)
\begin{align*}
(z-A)^{-1} = \frac{1}{z}\sum_{n=0}^\infty \frac{1}{z^n}A^n
\end{align*}
und daher ist $z\in\rho(A)$. Insbesondere gilt
\begin{align*}
\norm{(z-A)^{-1}} \le \frac{1}{\abs{z}}\sum_{n\ge0}
\frac{1}{\abs{z}^n}\norm{A}^n
= \frac{1}{\abs{z}}\frac{1}{1-\abs{z}^{-1}\norm{A}} = \frac{1}{\abs{z}-\norm{A}}
\end{align*}
und $\sigma(A)\subset\setdef{z\in\C}{\abs{z}\le \norm{A}}$.

\textit{$\sigma(A)$ ist nicht leer}.
Angenommen $\sigma(A)= \varnothing$, dann ist $F_{l,x} : z\mapsto l(R_z(A)x)$
für alle $x\in X$ und $l\in X^*$
eine ganze Funktion und
\begin{align*}
\abs{F_{l,x}(z)} \le \norm{l}\norm{R_z(A)}\norm{x} \le
\norm{l}\norm{x}\frac{1}{\abs{z}-\norm{A}}\to 0,\qquad \abs{z}\to
\infty,\tag{*}
\end{align*}
also auch beschränkt.
Nach dem Satz von Liouville (Funktionentheorie) ist $F_{l,x}$ konsant und wegen
(*) gilt sogar $F_{l,x}\equiv 0$. Somit ist
$l(R_z(A)x)\equiv 0$ für alle $x\in X$ und $l\in X^*$ und daher ist $R_z(A)=0$
für alle $z\in\C$ im Widerspruch zu $(z-A)R_z(A) = \Id$.\qedhere
\end{proof}

Der Beweis lässt sich mit Funktionentheorie über Funktionen $f: \C\to X$
mit $X$ einem komplexen Banachraum deutlich abkürzen. Die Spektraltheorie
befasst sich im Wesentlichen mit solchen Abbildungen.

\begin{bem*}
Für einen unbeschränkten Operator $A:D\subset X\to X$ ist sowohl
$\sigma(A)=\varnothing$ als auch $\sigma(A) = \C$ möglich.\maphere
\end{bem*}

% %%--------------Einf\"uhrung--Notationen---Definitionen--------------------
Im Folgenden wollen wir untersuchen, ob wir einen nichtabgeschlossenen Operator
durch Vergrößerung seines Definitionsbereichs abschließen können.

\begin{defn*}
Ein Operator $A : D \subset X \to X$ heißt
\emph{abschließbar}\index{Operator!abschließbar}, falls ein abgeschlossener
Operator $B \supset A$ existiert.
Die kleinste abgeschlossene Erweiterung eines abschließbaren
Operators $A$ heißt \emph{Abschluss}\index{Operator!Abschluss} von $A$ und wird
mit $\bar{A}$ bezeichnet.
\end{defn*}

Die Existenz von $\bar{A}$ folgt aus folgendem

\begin{prop}
\label{prop:3.5}
Sei $A : D(A) \subset X \to X$ ein linearer Operator.
Dann sind äquivalent:
\begin{equivenum}
\item\label{prop:3.5:1} $A$ ist abschließbar.
\item\label{prop:3.5:2} $\overline{\Gamma}_A$ ist der Graph eines linearen
Operators.
\item\label{prop:3.5:3} Aus $x_n \to 0$ und $Ax_n \to y$ folgt $y = 0$.\fishhere
\end{equivenum}
\end{prop}
\begin{proof}
``\ref{prop:3.5:1}$\Rightarrow$\ref{prop:3.5:3}'': Sei $A\subset B$, $B$
abschließbar und $x_n$ Folge in $D$ mit $x_n\to 0$ und $Ax_n\to y$. Dann gilt
$Bx_n \to y$ und da $B$ abgeschlossen ist $y = \lim\limits_{n\to \infty} Bx_n =
B0 = 0$.

``\ref{prop:3.5:3}$\Rightarrow$\ref{prop:3.5:2}'': Seien $(x,y_1),(x,y_2)\in
\overline{\Gamma_A}$. Dann ist $(0,y_1-y_2) = (x,y_1)-(x,y_2)\in
\overline{\Gamma_A}$. Also existiert eine Folge $(x_n,Ax_n)$ in
$\overline{\Gamma_A}$ mit $x_n\to 0$ und $Ax_n\to y_1-y_2$. Nach
\ref{prop:3.5:3} gilt $y_1=y_2$. Wir können somit einen Operator $B:D(B)\subset
X\to X$ definieren durch
\begin{align*}
x\in D(B) \Leftrightarrow \text{Es existiert }y\in Y\text{ mit }(x,y)\in
\overline{\Gamma_A}.
\end{align*}
und $Bx=y$ wenn $(x,y)\in \overline{\Gamma_A}$. Nach dem eben gezeigten, ist $B$
wohldefiniert, linear, denn $\overline{\Gamma_A}$ ist ein linearer Raum, und
$\Gamma_B = \overline{\Gamma_A}$.

``\ref{prop:3.5:2}$\Rightarrow$\ref{prop:3.5:1}'': Aus
$\Gamma_B=\overline{\Gamma_A}$ folgt $B\supset A$ und $B$ ist abgeschlossen, da
$\Gamma_B$ abgeschlossen. Somit ist $A$ abschließbar.\qedhere
\end{proof}


\begin{bem*}
Nach Satz~\ref{prop:3.5} können wir einen linearen Operator
$\bar{A}$ definieren durch $\Gamma_{\bar{A}} = \overline{\Gamma}_A$.
$\bar{A}$ ist abgeschlossen und f\"ur jeden abgeschlossenen Operator
$B \supset A$ gilt $B \supset \bar{A}$
(folgt aus $A \subset B \Leftrightarrow \Gamma_A \subset \Gamma_B$).
Offenbar gilt:
\begin{align*}
x \in D(\bar{A})
\quad\Leftrightarrow\quad
\begin{cases}
\text{Es existiert eine Folge }(x_n)_{n \in \N} \text{ in } D(A)\\
\text{mit } x_n \to x \text{ und } (Ax_n)_{n \in \N}
\text{ ist eine Cauchyfolge.}
\end{cases}
\end{align*}
Es ist dann $\bar{A}x = \lim \limits_{n \to \infty} Ax_n$.\maphere
\end{bem*}
%%---------------------------------------------------------------------
\begin{bsp*}
Sei $H = L^2(\R)$ und $g\in H$ mit $\norm{g}=1$. Der Operator $A
: D \subset H \to H$ definiert durch
\begin{align*}
  D = C_{0}^{\infty}(\R),\qquad Af = f(0)g,
\end{align*}
ist nicht abschließbar.
\begin{proof}
Sei $f\in D$ mit $f(0)=1$ und sei $f_n(x):=f(nx)$. Dann ist
$\|f_n\|=n^{-1/2}\|f\|\to 0$ f\"ur $n\to\infty$ aber
$Af_n = g$ f\"ur alle $n$, wobei $g\neq 0$.\qedhere\bsphere
\end{proof}
\end{bsp*}

\chapter{Symmetrische und selbstadjungierte Operatoren}
%\renewcommand{\Hc}{\mathrm{H}}
\renewcommand{\bar}[1]{\overline{#1}}

\subsection{Der adjungierte Operator}

\begin{defn*}
Sei $\Hc$ ein komplexer Hilbertraum und sei $A : D(A) \subset \Hc \to \Hc$
linear und dicht definiert $(\bar{D} = \Hc)$. Der zu $A$ \emph{adjungierte
Operator}\index{Operator!adjungiert}
$A^{*} : D(A^{*}) \subset \Hc \to \Hc$ ist wie folgt definiert:
Falls zu gegebenem $\ph \in \Hc$ ein $\ph^{*} \in \Hc$ existiert,
so dass
\begin{align*}
\lin{\ph,A\eta} = \lin{\ph^{*},\eta}, \qquad \text{für alle}\ \eta\in D(A),
\end{align*}
dann ist  $\ph \in D(A^{*})$ und $A^{*}\ph := \ph^{*}$.\fish
\end{defn*}

Da $D(A)$ dicht ist in $\Hc$, ist $\ph^{*}$ durch $\ph$ eindeutig bestimmt,
und die Abbildung $\ph \mapsto \ph^{*}$ ist linear.
Nach dem Lemma von Fréchet-Riesz (FA~Thm~3.4.9 \cite{Fun07}) gilt:
%\begin{align*}
\[
   D(A^{*}) = \setdef{\ph\in\Hc}{\eta \mapsto \lin{\ph,A\eta}
\text{ ist stetig auf } D(A)}.
\]
%\end{align*}
\begin{proof}
Aus $\lin{\ph,A\eta} = \lin{\ph^{*},\eta}$ f\"ur alle
$\eta \in D(A)$ folgt unmittelbar die Stetigkeit von $\eta \mapsto
\lin{\ph,A\eta}$.
Falls umgekehrt $\eta \mapsto \lin{\ph,A\eta}$ stetig ist,
dann gibt es wegen $\overline{D(A)} = \Hc$ ein eindeutiges
$F \in \Hc^{*}$ mit $F(\eta) = \lin{\ph,A\eta}$
für $\eta \in D(A)$ (FA~Satz~3.3.8~\cite{Fun07}).
Nach Fréchet-Riesz gilt $F(\eta) = \lin{\ph^{*},\eta}$
für ein $\ph^{*} \in \Hc$.\qed
\end{proof}

\begin{bem*}
Aus $A \subset B$ folgt $A^{*} \supset B^{*}$.

Man beachte, dass $A^*$ nur existiert, wenn $A$ dicht definiert ist.
Damit $(A^*)^*$ existiert muss also $A^*$ wieder dicht definiert sein,
was im Allgemeinen nicht der Fall ist.~\map
\end{bem*}
%%-------------------Beispiel----------------------------------------------

\begin{bsp*}
\begin{bspenum}
\item Sei $\Hc = L^2(\R)$ und $g\in\Hc$ mit $\|g\|=1$. Wir definieren
$A \colon D \subset \Hc \to \Hc$ durch
\begin{align*}
D = C_{0}^{\infty}(\R),\qquad Af = f(0)g.
\end{align*}
Dann ist $D(A^*)=\setdef{\ph\in \Hc}{\ph \bot\ g}$ und $A^*=0$ auf $D(A^*)$.
\begin{proof}
Sei $\lin{\ph,g}=0$. Dann ist $\lin{\ph,Af}=0$ f\"ur alle $f\in
D(A)$. Also $\ph\in D(A^*)$ und $A^*\ph=0$.
Sei  $\lin{\ph,g}\neq 0$ und sei $(f_n)$ eine Folge in $D$ mit
$\norm{f_n}\to 0$ und $f_n(0)=1$. Dann gilt
$\lim\limits_{n\to\infty}\lin{\ph,Af_n} = \lin{\ph,g}\neq 0$. Somit ist
$f\mapsto \lin{\ph,Af}$ nicht stetig, also $\ph \not\in D(A^*)$.\qed
\end{proof}

\item
Sei $\Hc = L^2(\R)$ und $(e_n)_{n \in \N}$ eine Orthonormalbasis
von $L^2(\R)$. Der Operator $A \colon D \subset \Hc \to \Hc$ sei
definiert durch
$D = C_{0}^{\infty}(\R)$ und
%\[
\begin{align*}
Af = \sum_{n=0}^{\infty} f(n) e_n,
\end{align*}
%\]
wobei die Reihe konvergiert wegen $f \in C_{0}^{\infty}(\R)$.
Dann gilt $D(A^{*}) = (0)$.

% \begin{proof}
% Sei $g \in L^2(\R)$ mit $g \ne 0$. Wir zeigen, dass
% $f \mapsto \sprod{g}{Af}$ nicht stetig ist. Daraus folgt
% $g \notin D(A^{*})$. Da $g\neq 0$, gibt ein $n_0 \in \N$ mit
% $\sprod{e_{n_0}}{g} \ne 0$.
% Sei $(f_k)_{k \in \N}$ eine Folge in $D$ mit
% $\norm{f_k} \to 0$,
% \[
%   f_k(n_0) = 1 \quad \text{und} \quad f_k(n) = 0, n \ne n_0.
% \]
% Dann gilt $A f_k = f_k(n_0) e_{n_0} = e_{n_0}$, also
% \[
% \sprod{g}{Af_k} = \sprod{g}{e_{n_0}} \ne 0
% \]
% f\"ur alle $k$.
% Insbesondere $\sprod{g}{Af_k}\not\rightarrow 0$, w\"ahrend
% $f_k \to 0 \ (k \to \infty)$.
% \end{proof}
\begin{proof}
Übung.\qed
\end{proof}
\end{bspenum}
\end{bsp*}


%%--------------------------------------------------------------------
%%-------------Satz----------------------------------------------------
\medskip %Platz
\begin{prop}
\label{prop:4.1}
Sei $\Hc$ ein komplexer Hilbertraum und
$A : D \subset \Hc \to \Hc$ dicht definiert.
Dann gelten
\begin{propenum}
\item $A^{*}$ ist abgeschlossen.
\item
$A$ ist genau dann abschließbar, wenn $D(A^{*})$
dicht ist und dann gilt $\bar{A} = (A^{*})^{*}$.
\item
Falls $A$ abschließbar ist, gilt $\left(\bar{A}\right)^{*} = A^{*}$.
\end{propenum}
\end{prop}
\begin{proof}
\begin{proofenum}
\item Sei $\ph_n\in D(A^*)$ mit $\ph_n\to \ph\in \Hc$ und $A^*\ph_n\to \psi$.
Für alle $\eta\in D(A)$ gilt
\begin{align*}
\lin{\ph,A\eta} = \lim\limits_{n\to\infty}
\lin{\ph_n,A\eta} = \lim\limits_{n\to\infty} \lin{A^*\ph_n,\eta} =
\lin{\psi,\eta}.
\end{align*}
Somit ist $\ph\in D(A^*)$ und $A^*\ph =\psi$.
\item ``$\Leftarrow$'': Sei $D(A^*)\subset \Hc$ dicht und sei $\ph_n\to0$ mit
$A\ph_n\to \psi$. Zu zeigen ist, dass $\psi=0$. Nach Voraussetzung gilt für alle
$\eta\in D(A^*)$,
\begin{align*}
\lin{\eta,\psi} = \lim\limits_{n\to\infty}\lin{\eta,A\ph_n} = 
\lim\limits_{n\to\infty} \lin{A^*\eta,\ph_n} = 0.
\end{align*}
Also ist $\psi\in D(A^*)^\bot = (0)$.

``$\Rightarrow$'': Siehe (Thm. VII.1, \cite{RS95a}).
\item Aus $\bar{A} \supset A$ folgt $(\bar{A})^* \subset A^*$. Es bleibt noch zu
zeigen, dass $(\bar{A})^* \supset A^*$. Sei also $\ph\in D(A^*)$ und $\ph^* =
A^*\ph$, so gilt 
\begin{align*}
\lin{\ph,A\eta} = \lin{\ph^*,\eta}\text{ für alle }\eta\in D(A).\tag{*}
\end{align*}
Ist $\eta\in D(\bar{A})$, dann gibt es eine Folge $(\eta_n)$ in $D(A)$ mit
$\eta_n\to \eta$ und $A\eta_n\to \bar{A}\eta$. Mit (*) folgt somit
\begin{align*}
\lin{\ph,\bar{A}\eta} = \lin{\ph^*,\eta}\text{ für alle }\eta\in D(\bar{A}).
\end{align*}
Folglich ist $\ph\in D((\bar{A})^*)$ und $(\bar{A})^*\ph = \ph^*$. Also ist
$(\bar{A})^*\supset A^*$.\qed
\end{proofenum}
\end{proof}

\begin{defn*}
Der \emph{Kern}\index{Operator!Kern} eines linearen Operators $B : D
\subset \Hc \to \Hc$ ist definiert durch
\begin{align*}
\ker(B) := \setdef{\ph \in D}{B \ph = 0}.\fish
\end{align*}
\end{defn*}

\begin{prop}
\label{prop:4.2}
Sei $A:D(A)\subset \Hc\to \Hc$ dicht definiert. Dann gilt
\begin{align*}
\ker(A^*) = \ran(A)^\bot.
\end{align*}
Insbesondere ist $\ran(A)^\bot\subset D(A^*)$ und aus $\ker(A^*)=(0)$ folgt
$\bar{\ran(A)} = \Hc$.\fish
\end{prop}
\begin{proof}
Es folgt leicht
\begin{align*}
\ph\in \ker(A^*) &\Leftrightarrow
\lin{A^*\ph,\eta} = 0\forall \eta\in D(A)\\
&\Leftrightarrow
\lin{\ph,A\eta} = 0\forall \eta\in D(A)\\
&\Leftrightarrow
\ph\in\ran(A)^\bot.
\end{align*}
Außerdem folgt für $\ker(A^*)=(0)$ nach (FA Kor 3.8, \cite{Fun07}),
\begin{align*}
\Hc = (0)^\bot = \ker(A^*)^\bot
= (\ran(A)^\bot)^\bot = \bar{\ran(A)}.\qed 
\end{align*}
\end{proof}

\subsection{Symmetrie und Selbstadjungiertheit}

\begin{defn*}[Definition / Lemma]
Ein dicht definierter Operator $A:D(A)\subset \Hc\to\Hc$ heißt
\emph{symmetrisch}\index{Operator!symmetrisch}, falls $A\subset A^*$. Der
Operator $A$ heißt \emph{selbstadjungiert}\index{Operator!selbstadjungiert},
falls $A=A^*$.

$A$ ist genau dann symmetrisch, wenn
\begin{align*}
\lin{A\ph,\eta} = \lin{\ph,A\eta},\quad \text{für alle }\ph,\eta \in
D(A).\tag{*}
\end{align*}
$A$ ist genau dann selbstadjungiert, wenn (*) und zusätzlich $D(A)=D(A^*)$
gilt.\fish
\end{defn*}

\begin{bem*}[Bemerkungen.]
\begin{bemenum}
\item Für $A\in\Lc(\Hc)$ sind Symmetrie und Selbstadjungiertheit äquivalent.
\item Ein selbstadjungierter Operator $A$ hat keine symmetrische Erweiterung
$B\supset A$ mit $B\neq A$.
\begin{proof}
Aus $A\subset B\subset B^*$ und $A=A^*$ folgt $A=A^*\supset B^* \supset B\supset
A$ und somit $A=B$.\qed
\end{proof}
\item Falls $A\subset A^*$ und $B$ eine selbstadjungierte Erweiterung von $A$,
dann gilt $A\subset B\subset A^*$.\map
\end{bemenum}
\end{bem*}

\begin{bsp*}
Sei $\Hc=L^2(\R)$, $D=\SS(\R)$ und $A:D\subset \Hc\to \Hc$ definiert durch $f =
-if'$. Dann ist $A$ symmetrisch aber $A\neq A^*$. Es gilt $D(A^*) = H^1(\R)$ und
\begin{align*}
A^*f= \Fc^{-1}p \Fc f,\qquad f\in H^1(\R).
\end{align*}
$A^*$ ist selbstadjungiert, d.h. $(A^*)^*=A^*$.
\begin{proof}
Wir zeigen hier nur die Symmetrie. Die übrigen Eigenschaften weisen wir später
nach. Für alle $f,g\in \SS(\R)$ gilt
\begin{align*}
\lin{f,Ag} &= -i \int_\R \overline{f(x)}g'(x)\dx
= \lim\limits_{R\to\infty} -i \int_{-R}^R \overline{f(x)}g'(x)\dx\\
&= -\overline{f(x)}g(x)\big|_{-R}^R  + i \int_{-R}^R \overline{f'(x)}g(x)\dx\\
&= \int_\R \overline{-if'(x)}g(x)\dx = \lin{Af,g}.
\end{align*}
Somit ist $A\subset A^*$.\qed\boxc
\end{proof}
\end{bsp*}


Wir suchen nun nach handlichen Kriterien, um zu zeigen, dass ein Operator
selbstadjungiert ist. Dabei wird sich folgender Satz als äußert nützlich
erweisen.

\begin{bem*}[Notation.]
$\C_+ := \setd{z\in\C}{\Im z > 0}$, $\C_- = \setdef{z\in\C}{\Im z < 0}$.\map
\end{bem*}

\begin{prop}
\label{prop:4.3}
Sei $A: D(A)\subset \Hc\to \Hc$ symmetrisch.
\begin{propenum}
\item Für alle $\lambda,\mu\in\R$ und alle $\ph\in D(A)$ gilt
\begin{align*}
\norm{(A-\lambda-i\mu)\ph}^2 = \norm{(A-\lambda)\ph}^2 + \mu^2\norm{\ph}^2.
\end{align*}
\item Falls $\ran(A-z) = \Hc$ für ein $z\in\C_\pm$, so ist $\C_\pm\subset
\rho(A)$.\fish
\end{propenum}
\end{prop}
\begin{proof}
\begin{proofenum}
\item Sei $B=A-\lambda$. Dann gilt
\begin{align*}
\norm{(B-i\mu)\ph}^2
&= \lin{(B-i\mu)\ph,(B-i\mu)\ph}\\
&= \lin{B\ph,B\ph} \overline{-i\mu}\lin{\ph,B\ph}
- i\mu\lin{B\ph,\ph} \\
&\qquad\qquad\qquad
+(\overline{-i})(-i)\mu^2\lin{\ph,\ph}\\
&= 
\norm{B\ph}^2+\mu^2\norm{\ph}^2 
\underbrace{+i\mu\lin{\ph,B\ph}
- i\mu\lin{B\ph,\ph}}_{=0}.
\end{align*}
\item Sei $z\in\C_\pm$ und $\ran(A-z) =\Hc$. Nach Teil 1) gilt für
alle $\ph\in D(A)$,
\begin{align*}
\norm{(A-z)\ph} \ge \underbrace{\abs{\Im z}}_{>0}\norm{\ph}.\tag{*}
\end{align*}
D.h. $(A-z) : D(A)\subset \Hc \to \Hc$ ist injektiv , also nach Voraussetzung an
$z$ sogar bijektiv. Mit der Wahl $\ph=(A-z)^{-1}\psi$ folgt aus (*)
\begin{align*}
\norm{(A-z)^{-1}} \le \frac{1}{\abs{\Im z}}.
\end{align*}
Somit ist $z\in\rho(A)$ und nach Theorem \ref{prop:3.2} gilt außerdem
$B(z,\abs{\Im z}) \subset\rho(A)$.

Für jedes $z_1\in B(z,\abs{\Im z})$ folgt aus dem Gezeigten,
dass $B(z_1,\abs{\Im z}) \subset\rho(A)$. Durch Iteration dieses Arguments
schließen wir $\C_\pm\subset \rho(A)$.\qed
\end{proofenum}
\end{proof}

\begin{thm}
\label{prop:4.4}
Sei $A: D(A)\subset \Hc\to \Hc$ symmetrisch. Dann sind äquivalent
\begin{propenum}
\item\label{prop:4.4:1} $A=A^*$.
\item\label{prop:4.4:2} $\sigma(A) \subset \R$.
\item\label{prop:4.4:3} $\ran(A-z_\pm) = \Hc$ für ein $z_+\in \C_+$ und ein
$z_-\in \C_-$.
\item\label{prop:4.4:4} $A$ ist abgeschlossen und $\ker(A^*-z_\pm) = (0)$ für
ein $z_+\in \C_+$ und ein $z_-\in \C_-$.\fish
\end{propenum}
\end{thm}

\begin{bem*}
$\sigma(A)\subset\R$ allein impliziert \textit{nicht}, dass $A$ selbstadjungiert
ist.

Ein einfaches Gegenbeispiel ist $A=\bigl(\begin{smallmatrix}0 & 1\\ 0 &
0\end{smallmatrix}\bigr)$, denn $A$ hat nur den Eigenwert $0\in\R$, ist aber
offensichtlich nicht hermitesch.\map
\end{bem*}

\begin{proof}
``\ref{prop:4.4:1}$\Rightarrow$\ref{prop:4.4:4}'': Aus $A=A^*$ und Satz
\ref{prop:4.1} folgt, dass $A$ abgeschlossen ist. Außerdem gilt
\begin{align*}
\ker (A^*\pm i) = \ker (A\mp i) = (0)
\end{align*}
nach dem eben bewiesenen Satz \ref{prop:4.3}.

``\ref{prop:4.4:4}$\Rightarrow$\ref{prop:4.4:3}'': Aus $\ker(A- z_\pm) = (0)$
und Satz \ref{prop:4.2} folgt $\ran(A-\overline{z_\pm}) \subset \Hc$ ist dicht.
Wir zeigen $\ran(A-\overline{z_\pm}) = \Hc$ mit Hilfe der Abgeschlossenheit von
$A$. Sei $\psi\in \Hc$, so existiert aufgrund der Dichtheit eine Folge $(\ph_n)$
in $D(A)$ mit
\begin{align*}
(A-\overline{z_\pm})\ph_n \to \psi.
\end{align*}
Nach Satz \ref{prop:4.3} gilt
\begin{align*}
\norm{(A-\overline{z_\pm})(\ph_n-\ph_m)} \ge
\abs{\ph_n-\ph_m}\underbrace{\abs{\Im z_\pm}}_{>0}.
\end{align*}
Also ist $(\ph_n)$ Cauchyfolge in $\Hc$. Sei
$\ph=\lim\limits_{n\to\infty}\ph_n$, so folgt, da $A-\overline{z_+}$
abgeschlossen, dass $\ph\in D(A)$ und $(A-\overline{z_\pm})\ph=\psi$.
%Analog zeigt man $\ran(A-\overline{z_-})=\Hc$.

``\ref{prop:4.4:3}$\Rightarrow$\ref{prop:4.4:2}'': Folgt direkt aus Satz
\ref{prop:4.3}.

``\ref{prop:4.4:2}$\Rightarrow$\ref{prop:4.4:1}'': Da $A\subset A^*$ bleibt nur
$D(A^*)\subset D(A)$ zu zeigen. Sei also $\ph\in D(A^*)$.  Da $(A+i): D(A)\to
\Hc$ nach Voraussetzung surjektiv ist, existiert ein $\eta\in D(A)$ mit
$(A^*+i)\ph = (A+i)\eta = (A^*+i)\eta$. Also ist $(A^*+i)(\ph-\eta) = 0$ und
$\ker (A^*+i) = \ran(A-i)^\bot =(0)$, da $i\in\rho(A)$. Das heißt $\ph=\eta \in
D(A)$.\qed
\end{proof}

\subsection{Wesentliche Selbstadjungiertheit}

Sei $A: D(A)\subset \Hc\to\Hc$ ein symmetrischer Operator, so ist $A\subset
A^*$. Da $A^*$ abgeschlossen, ist $A$ abschließbar mit $\bar{A}\subset A^*$.
Falls $\bar{A}$ selbst-adjungiert ist, dann ist $\bar{A}$ die einzige
selbstadjungierte Erweiterung von $A$ (vgl. obige Bemerkung).
Dies motiviert folgende

\begin{defn*}
Ist der Abschluss eines Operators $A:D(A)\subset\Hc\to\Hc$ selbstadjungiert,
dann heißt $A$ \emph{wesentlich selbstadjungiert}\index{Operator!wesentlich
selbstadjungiert}. Sei $A$ abgeschlossen, so heißt jeder Teilraum $D\subset
D(A)$ mit $\bar{A\big|_D} = A$ \emph{definierender Bereich (core)} von
$A$\index{Operator!definierender Bereich}.\fish
\end{defn*}

\begin{thm}
\label{prop:4.5}
Sei $A: D(A)\subset \Hc\to \Hc$ ein symmetrischer Operator. Dann sind
äquivalent:
\begin{propenum}
\item\label{prop:4.5:1} $A$ ist wesentlich selbstadjungiert.
\item\label{prop:4.5:2} $\ran(A\pm i)\subset \Hc$ ist dicht.
\item\label{prop:4.5:3} $\ker(A^*\pm i) = (0)$.\fish
\end{propenum}
\end{thm}
\begin{proof}
``\ref{prop:4.5:1}$\Rightarrow$\ref{prop:4.5:3}'': Nach Voraussetzung
ist $\bar{A}$ selbstadjungiert und $(\bar{A})^* = A^*$ nach Satz
\ref{prop:4.1}. Also ist $\ker(A^*\pm i) = \ker ((\bar{A})^*\pm i) = (0)$ nach
Theorem \ref{prop:4.4}.

``\ref{prop:4.5:3}$\Rightarrow$\ref{prop:4.5:2}'': Folgt direkt aus Satz
\ref{prop:4.2}.

``\ref{prop:4.4:2}$\Rightarrow$\ref{prop:4.4:1}'': Wir zeigen $\ran(\bar{A}\pm
i) = \Hc$, dann folgt die Selbstadjungiertheit aus Theorem \ref{prop:4.4}. Sei
also $\psi\in \Hc$. Da $\bar{\ran(A-i)} = \Hc$ gibt es eine Folge $(\ph_n)$ in
$D(A)$ mit $(A-i)\ph_n \to \psi$. Nach Satz \ref{prop:4.3} ist $(\ph_n)$ eine
Cauchyfolge in $\Hc$. (Siehe Beweis von Theorem \ref{prop:4.4}) Sei
$\ph=\lim\limits_{n\to\infty} \ph_n$, dann gilt $A\ph_n \to \psi + i\ph$. Es
folgt $\ph\in D(\bar{A})$ und $\bar{A}\ph = \psi+i\ph$, d.h. $(\bar{A}-i)\ph =
\psi$.
Analog zeigt man $\ran(\bar{A}+i)=\Hc$.\qed
\end{proof}

Als Anwendung beweisen wir einen wichtigen Störungssatz.

\begin{thm}[Theorem von Kato-Rellich]
\label{prop:4.6}
Sei $A$ wesentlich selbstadjungiert und $B$ symmetrisch mit $D(B)\supset D(A)$.
Gilt weiterhin
\begin{align*}
\norm{B\ph} \le a\norm{A\ph} + b\norm{\ph}
\end{align*}
für alle $\ph\in D(A)$, wobei $a,b\in\R$ und $a<1$, so ist
\begin{align*}
A+B : D(A)\subset \Hc\to \Hc
\end{align*}
wesentlich selbstadjungiert und $D(\bar{A+B}) = D(\bar{A})$.\fish
\end{thm}

Wir werden später sehen, dass $-\Delta$ auf $\SS(\R)$ wesentlich
selbstadjungiert ist und $\abs{x}^{-1}$ die obige Abschätzung erfüllt, folglich
ist der Hamilton des Wasserstoffatoms wesentlich selbstadjungiert.

\begin{bem*}
Ist $A=A^*$ und $B\in\Lc(\Hc)$ symmetrisch, dann ist
\begin{align*}
A+B : D(A)\subset\Hc \to \Hc 
\end{align*}
selbstadjungiert nach Theorem \ref{prop:4.6}. (Wähle dazu $a=0$ und
$b=\norm{B}$). Beschränkte Störungen sind daher weitgehend
uninteressant.\map
\end{bem*}

\begin{proof}
\begin{proofenum}
\item
Wir betrachten zuerst den Fall, dass $A$ selbstadjungiert ist. Nach
Voraussetzung ist auch der Operator
$
A+B:D(A)\subset\Hc\to \Hc
$
symmetrisch. Es genügt daher nach Theorem \ref{prop:4.4} 
zu zeigen, dass $(A+B-i\mu)D(A) =\Hc$ für $\abs{\mu}$ hinreichend groß.

Nach Satz \ref{prop:4.3} gilt
\begin{align*}
\norm{(A-i\mu)\psi}^2 = \norm{A\psi}^2 + \mu^2\norm{\psi}^2,\qquad \psi\in D(A).
\end{align*}
Setzen wir $\psi=(A-i\mu)^{-1}\ph$ für ein $\mu\in\R\setminus\setd{0}$, so folgt
\begin{align*}
\norm{\ph} \ge \norm{A(A-i\mu)^{-1}\ph},\qquad
\norm{\ph} \ge \abs{\mu}\norm{(A-i\mu)^{-1}\ph}\tag{*}
\end{align*}
für alle $\ph\in\Hc$. Sei $\mu\in\R\setminus \setd{0}$, so gilt weiterhin
\begin{align*}
(A+B-i\mu) = (1+B(A-i\mu)^{-1})(A-i\mu),\tag{**}
\end{align*}
wobei $(A-i\mu):D(A)\to\Hc$ bijektiv ist. Ferner gilt nach Voraussetzung und (*)
\begin{align*}
\norm{B(A-i\mu)^{-1}\ph} &\le a \norm{A(A-i\mu)^{-1}\ph} +
b\norm{(A-i\mu)^{-1}\ph}\\
&\le \left(a+\frac{b}{\abs{\mu}}\right)\norm{\ph}.
\end{align*} 
Somit ist $\norm{B(A-i\mu)^{-1}}\le a+\frac{b}{\abs{\mu}} < 1$ für
$\abs{\mu}$ hinreichend groß. Für solche $\mu$ ist
\begin{align*}
(1+B(A-i\mu)^{-1}) : \Hc\to\Hc
\end{align*}
bijektiv (\cite{Fun07}, Thm 3.3.6), folglich ist nach (**) auch $A+B-i\mu$
bijektiv.
\item Sei nun $A$ lediglich wesentlich selbstadjungiert. Sei $\ph\in D(\bar{A})$
und $(\ph_n)_n$ eine Folge in $D(A)$ mit $\ph_n\to \ph$ und $A\ph_n\to
\bar{A}\ph$. Dann ist auch $B\ph_n$ Cauchyfolge, denn
\begin{align*}
\norm{B(\ph_n-\ph_m)} \le a\norm{A(\ph_n-\ph_m)} + b\norm{\ph_n-\ph_m}
\end{align*}
Also ist $\ph\in D(\bar{B})$ und $\ph\in D(\bar{A+B})$, denn $((A+B)\ph_n)$ ist
ebenfalls Cauchyfolge. Außerdem gilt durch Grenzwertübergang
\begin{align*}
\norm{\bar{B}\ph} \le a\norm{\bar{A}\ph} + b\norm{\ph},\qquad \ph\in D(\bar{A}).
\end{align*}
Somit ist $\bar{A}+\bar{B}$ nach dem 1. Teil selbstadjungiert und insbesondere
abgeschlossen. Also gilt
\begin{align*}
\bar{A}+\bar{B}\subset \bar{A+B},
\end{align*}
weil wir eben außerdem gezeigt haben, das $D(\bar{A})\subset D(\bar{A+B})$ folgt
$D(\bar{A+B}) = D(\bar{A})$ und $\bar{A+B} = \bar{A}+\bar{B}$.\qed
\end{proofenum}
\end{proof}

Die (reinen) Zustände eines quantenmechanischen Systems werden durch normierte
Vektoren eines komplexen Hilbertraums $\Hc$ beschrieben. Die Zeitevolution
eines Zustandes, gegeben durch $u\in\Hc$, ist bestimmt durch das
Anfangswertproblem für eine Schrödingergleichung
\begin{align*}
i\frac{\ddd}{\dt}\ph_t = H\ph_t,\qquad \ph_0 = u,\tag{S}
\end{align*}
wobei $H:D\subset\Hc\to\Hc$ im Allgemeinen unbeschränkt ist.

Wir nehmen im Folgenden an, dass $\bar{D}=\Hc$. Eine Lösung von (S) ist eine
differenzierbare\footnote{differenzierbar im Sinne von (iii)} Funktion $\ph:
I\to\Hc$, $I\subset\R$ ein nichtentartetes Intervall mit $0\in I$, so dass
\begin{equivenum}
\item $\ph_t \in D(H)$ für alle $t\in I$.
\item $\ph_0 = u$.
\item $\dfrac{\ddd}{\dt}\ph_t = \lim_{h\to 0} \dfrac{\ph_{t+h}-\ph_t}{h} =
-iH\ph_t$ für alle $t\in I$.
\end{equivenum}
\begin{lem*}
Falls (S) für jedes $u\in D$ eine Lösung $\ph$ besitzt mit $\norm{\ph_t} =
\norm{u}$ für alle $t\in I$, dann ist notwendigerweise $H\subset H^*$. Umgekehrt
folgt aus $H\subset H^*$, dass $\norm{\ph_t}$ erhalten und die Lösung
von (S) eindeutig ist.\fish
\end{lem*}
\begin{proof}
Sei $u\in D$ und $\ph_t$ eine Lösung von (S) mit $\norm{\ph_t} = \norm{u}$. Dann
gilt
\begin{align*}
0 &= \frac{\ddd}{\dt}\norm{\ph_t}^2\big|_{t=0}
= \lin{\dot{\ph}_t,\ph_t}+\lin{\ph_t,\dot{\ph}_t}\big|_{t=0}
=  \lin{-iHu,u} +\lin{u,-iHu}\\ & = i
\left[\underbrace{\lin{Hu,u}-\lin{u,Hu}}_{=0}\right]
\end{align*}
Aus $\lin{Hu,u} = \lin{u,Hu}$ für alle $u\in D$ folgt $H\subset H^*$.

Sind $\ph_t,\psi_t:I\to \Hc$ Lösungen von (S) und ist  $\Phi_t = \ph_t-\psi_t$,
dann gilt
\begin{align*}
i\frac{\ddd}{\dt}\Phi_t = H\Phi_t,\qquad \Phi_0 = 0. 
\end{align*}
Also folgt wegen $H\subset H^*$, dass $\norm{\Phi_t} = \norm{\Phi_0} = 0$ für
alle $t\in I$.\qed
\end{proof}

\begin{prop}
\label{prop:4.7}
Sei $H:D\subset\Hc\to\Hc$ symmetrisch. Falls (S) für jedes $u\in D$ eine globale
Lösung $\ph: \R\to \Hc$ besitzt, ist $H$ wesentlich selbstadjungiert.\fish
\end{prop}
\begin{proof}
Nach Theorem \ref{prop:4.5} genügt es zu zeigen, dass $\ker(H^*\pm i)=(0)$. Sei
also $u\in D$ und $\ph_t$ die Lösung von (S) zu $u$ und für ein $w\in D(H^*)$,
$(H^*+i)w =0$. Dann gilt
\begin{align*}
\frac{\ddd}{\dt}\lin{w,\ph_t}
= \lin{w,-iH\ph_t} = \lin{H^*w,-i\ph_t} = \lin{-iw,-i\ph_t} =
\lin{w,\ph_t}.
\end{align*}
Also gilt $\lin{w,\ph_t} = \lin{w,u}e^t$ und wegen
\begin{align*}
\abs{\lin{w,\ph_t}} \le \norm{w}\norm{\ph_t} = \norm{w}\norm{u}
\end{align*}
ist $\lin{w,u} = 0$ für alle $u\in D$ und da $\bar{D}=H$ folgt $w=0$. Analog
zeigt man, dass $\ker(H^*-i)=(0)$.\qed
\end{proof}

\begin{bem*}[Bemerkungen]
\begin{bemenum}
\item Wir werden sehen, dass $H=H^*$ auch hinrechend ist für die Existenz einer
globalen Lösung
\begin{align*}
\ph: \R\to \Hc
\end{align*}
von (S). Falls $H=H^*\in\Lc(\Hc)$, ist
\begin{align*}
\ph_t = e^{-iHt}\ph_0,\qquad
e^{-iHt} = \sum_{n=1}^\infty \frac{1}{n!} (-iHt)^n\ph_0.
\end{align*}
\item Nicht jeder symmetrische Operator $H$ hat eine selbstadjungierte
Erweiterung. Notwendig und hinreichend für die Existenz einer selbstadjungierten
Erweiterung ist, dass die \emph{Defektindizies}\index{Defektindizes}
\begin{align*}
&h_+ := \dim \ker(H^*-i) = \dim \ran(H^*+i)^\bot,\\
&h_- := \dim \ker(H^*+i) = \dim \ran(H^*-i)^\bot,
\end{align*}
übereinstimmen. (\cite{RS05a}, Kap. X)\map
\end{bemenum}
\end{bem*}


\chapter{Multiplikations- und Schrödingeroperator}

\subsection{Multiplikationsoperatoren}
Ein Schrödingeroperator $-\Delta + V$ wird im einfachsten Fall $(V=0)$ zu
$-\Delta$. Unter Fouriertransformation geht $-\Delta$ über in einen
Multiplikationsoperator.

\begin{defn*}
Eine messbare Funktion $f: \R^n\to \C$ definiert einen
\emph{Multiplikationsoperator}
$
M_f : D_f \subset L^2(\R^n)\to L^2(\R^n)
$
durch
\begin{align*}
D_f := \setdef{\ph\in L^2(\R^n)}{f\ph\in L^2(\R^n)},\quad
(M_f\ph)(x) = f(x)\ph(x).\fishhere
\end{align*}
\end{defn*}
Man sieht leicht ein, dass $D_f\subset L^2(\R^n)$ dicht ist. 

\begin{prop}
\label{prop:5.1}
Das Spektrum von $M_f$ ist der \emph{wesentliche Wertebereich} von $f$. D.h.
\begin{align*}
\sigma(M_f) := \setdef{z\in \C}{\forall \ep >0 : \abs{f^{-1}(B_\ep(z))}>
0},
\end{align*}
wobei $\abs{\cdot}$ das Lebesgue-Maß  bezeichnet. Ist $f$ reellwertig, dann ist
$M_f$ selbstadjungiert.\fishhere
\end{prop}

Ist $f$ stetig, dann entspricht der wesentliche Wertebereich dem Wertebereich.

\begin{proof}
Sei $W_f$ der wesentliche Wertebereich von $f$ und $z\in W_f$. Wir zeigen, dass
$z\in \sigma(M_f)$. Zu jedem $k\in\N$ existiert eine messbare Menge
$E_k\subset\R^n$ mit
\begin{align*}
E_k \subset f^{-1}(B_k(z)),\qquad 0 < \abs{E_k}< \infty.
\end{align*}
Wähle $E_k = f^{-1}(B_{1/k}(z))$, falls $\abs{f^{-1}(B_{1/k}(z))} < \infty$,
sonst wähle eine messbare Teilmenge von $f^{-1}(B_{1/k}(z))$. Sei
\begin{align*}
\ph_k = \frac{1}{\sqrt{\abs{E_k}}}\chi_{E_k}.
\end{align*}
Dann ist $\ph_k \in L^2(\R^n)$, $\norm{\ph_k}=1$ und $f\ph_k \in L^2$, denn
$\abs{f\big|_{E_k}-z} < \frac{1}{k}$ und
\begin{align*}
\norm{(z-M_f)\ph_k} &= \left(\int \abs{(z-f(x))\ph_k(x)}^2\dx \right)^{1/2}\\
&=
 \sqrt{\abs{E_k}}^{-1} \left(\int_{E_k} \abs{z-f(x)}^2\dx\right)^{1/2} 
 \le \frac{1}{k}\to 0.
\end{align*}
Somit ist $z\in \sigma(M_f)$ (vgl. Aufgabe 4.4).

Ist andererseits $z\notin W_f$, dann existiert ein $\ep > 0$, so dass
\begin{align*}
\abs{f^{-1}(B_\ep(z))} = 0.
\end{align*}
Definiere nun
\begin{align*}
g(x) =
\begin{cases}
(z-f(x))^{-1},& \abs{z-f(x)} \ge \ep,\\
0, & \text{sonst}.
\end{cases}
\end{align*}
Dann ist $g\in L^\infty(\R^n)$, also $M_g$ ein beschränkter Operator und
$(z-f(x))g(x) = 1$ fast überall. Also $fg\in L^\infty(\R^n)$ und somit
$M_gL^2(\R^n)\subset D_f$. Es folgt
\begin{align*}
\ph\in L^2(\R^n) \Rightarrow
(z-M_f)M_g\ph = (z-f)g\ph = \ph\text{ f.ü.},\\
\ph\in D_f \Rightarrow M_g(z-M_f)\ph = g(z-f)\ph = \ph\text{ f.ü.}.
\end{align*}
Somit ist $z\in \rho(M_f)$ und $M_g=(z-M_f)^{-1}$.

Falls $f$ reellwertig ist, so ist $M_f$ symmetrisch (Übung) und $\sigma(M_f) =
W_f \subset\R$, also ist $M_f$ nach Theorem \ref{prop:4.4}
selbstadjungiert.\qedhere
\end{proof}

\begin{bsp}
\begin{bspenum}
\item Der Operator $-i\frac{\diffd}{\dx}: H^1(\R)\subset L^2(\R) \to L^2(\R)$
definiert durch
\begin{align*}
-i\frac{\diffd}{\dx} = \FF^{-1}M_f\FF,\qquad f(p)=p,
\end{align*}
ist selbstadjungiert auf $H^1$, da $M_f$ selbstadjungiert. Auf $C_0^\infty(\R)$
ist $-i\frac{\diffd}{\dx}$ wesentlich selbstadjungiert und
$\sigma(-i\frac{\diffd}{\dx}) = \sigma(M_f) = \R$.
\item Der Operator $-\Delta : H^2(\R^n)\subset L^2(\R^n)\to L^2(\R^n)$ definiert
durch
\begin{align*}
-\Delta := \FF^{-1}M_f\FF,\qquad f(p) = p^2 = \sum_{k=1}^n p_k^2
\end{align*}
ist selbstadjungiert und wesentlich selbstadjungiert auf $C_0^\infty(\R^n)$ und
$\sigma(-\Delta) = [0,\infty)$. Der Beweis erfolgt mit Satz
\ref{prop:1.3}, Satz \ref{prop:5.1} und den Resultaten aus den Übungen.\bsphere
\end{bspenum}
\end{bsp}

\subsection{Schrödingeroperatoren}

\begin{defn*}
\emph{Schrödingeroperatoren} sind Operatoren der Form
\begin{align*}
-\Delta + V : D\subset L^2(\R^n)\to L^2(\R^n),\tag{*}
\end{align*}
wobei $V=M_V$ ein Multiplikationsoperator ist.\fishhere
\end{defn*}

\begin{thm}
\label{prop:5.2}
Sei $V \in L^2(\R^n)  + L^\infty(\R^n)$ reellwertig und $n\le 3$. Dann ist
(*) selbstadjungiert auf $D=H^2(\R^n)$ und wesentlich selbstadjungiert auf
$D=C_0^\infty(\R^n)$.\fishhere
\end{thm}
\begin{proof}
Da $V$ reellwertig ist, ist $M_V$ ein symmetrischer Multiplikationsoperator und
$-\Delta$ ist selbstadjungiert auf $H^2(\R^n)$ und wesentlich selbstadjungiert
auf $C_0^\infty(\R^n)$. Nach Theorem \ref{prop:1.6} existiert zu jedem $\ep > 0$
ein $C_\ep\in\R$, so dass
\begin{align*}
\norm{V\ph} \le \ep \norm{\Delta \ph} + C_\ep \norm{\ph},\qquad \ph\in
H^2(\R^n).
\end{align*}
Aus dem Satz von Kato-Rellich folgt schließlich, dass (*) selbstadjungiert auf
$H^2(\R^n)$ und wesentlich selbstadjungiert auf $C_0^\infty(\R^n)$ ist.\qedhere
\end{proof}

Mit denselben Methoden (vgl. \cite{RS05b}) beweist man das folgende
\begin{thm*}[Theorem von Kato (1951)]
Seien $V_k$, $V_{ik}\in L^2(\R^n) + L^\infty(\R^n)$ für $1\le i,k \le N$
reellwertig und $n\le 3$. Dann ist der Operator
\begin{align*}
-\Delta + \sum_{k=1}^N V_k(x_k) + \sum_{i<k} V_{ik}(x_i-x_k)
: D \subset L^2(\R^n)\to L^2(\R^n)
\end{align*}
selbstadjungiert auf $D=H^2(\R^{nN})$ und wesentlich selbstadjungiert auf
$D=C_0^\infty(\R^{nN})$.\fishhere
\end{thm*}

\begin{cor*}
Der Schrödingeroperator $H: D\subset L^2(\R^{3N})\to L^2(\R^{3N})$ gegeben durch
\begin{align*}
H=\sum_{k=1}^N \left(-\frac{\hbar^2}{2m}\Delta_{x_k}-\frac{Ze^2}{\abs{x_k}}
\right) + \sum_{i<k} \frac{e^2}{\abs{x_i-x_k}}
\end{align*}
ist selbstadjungiert auf $D=H^2(\R^{3N})$ und wesentlich selbstadjungiert auf
$D=C_0^\infty(\R^{3N})$.\fishhere
\end{cor*}

\begin{bem*}[Bemerkung zu den Einheiten.]
Der Operator aus dem obigen Korollar ist unitär äquivalent zu
\begin{align*}
2\alpha^2mc^2\left[ \sum_{k=1}^N \left(-\Delta_{x_k} -
\frac{Z}{\abs{x_k}}\right) + \sum_{i<k} \frac{1}{\abs{x_i-x_k}}
\right],
\end{align*}
wobei $\alpha= \frac{e^2}{\hbar c} \approx \frac{1}{137}$ die
Feinstrukturkonstante ist.

Die zugehörige unitäre Transformation ist gegeben durch
\begin{align*}
U: L^2(\R^{3N}) \to L^2(\R^{3N}),\qquad (U\ph)(x) := \lambda^{3N/2}\ph(\lambda
x)
\end{align*}
mit geeignetem $\lambda > 0$. Es gilt
\begin{align*}
UHU^{-1} = \sum_{k=1}^N \left(-\frac{\hbar^2}{2m\lambda^2}\Delta_{x_k} -
\frac{Ze^2}{\lambda \abs{x_k}}\right) + \sum_{i< k}
\frac{e^2}{\lambda\abs{x_i-x_k}}.
\end{align*}
Eine einfache Rechnung zeigt, dass
$\frac{\hbar^2}{2m\lambda^2}=\frac{e^2}{\lambda}$, mit der Wahl
\begin{align*}
\lambda = \frac{\hbar^2}{2me^2} = \frac{1}{2}\text{Bohr-Radius}.
\end{align*}
Der gemeinsame Faktor
\begin{align*}
\frac{\hbar^2}{2m\lambda^2} = \frac{e^2}{\lambda} = 2\alpha^2me^2
\end{align*}
ist die 4-fache Rydberg-Energie $E_{\mathrm{Ryd}} =
\frac{1}{2}\alpha^2mc^2$.\maphere
\end{bem*}

\chapter{Funktionalkalkül und Spektralsatz}

\section{Messbarer Funktionalkalkül}

Sei $\BB$ die Familie der Borel-messbaren Teilmengen von $\R$, d.h. die kleinste
$\sigma$-Algebra, die alle offenen und abgeschlossenen Teilmengen enthält.
Weiterhin bezeichne $B(\R)$ die Familie \textit{beschränkten} Borel-messbaren
Funktionen $f: \R\to\C$.

\begin{bem*}[Notation.]
Ist $(f_n)$ eine Folge in $B(\R)$, dann schreiben wir
\begin{align*}
f_n\pto f,\quad \text{oder}\quad f = \plim\limits_{n\to\infty} f_n,
\end{align*}
falls $f_n(x)\to f(x)$ für jedes $x\in \R$ und die $f_n$ gleichmäßig beschränkt
sind, d.h.
\begin{align*}
\sup_{n\ge 0} \norm{f_n} < \infty.\maphere
\end{align*} 
\end{bem*}

\begin{prop}
\label{prop:6.1}
Sei $\FF\subset B(\R)$ eine Familie von Funktionen mit
\begin{propenum}
\item $C_0(\R) \subset \FF$,
\item Ist $(f_n)$ eine Folge in $\FF$ mit $f_n\pto f$, dann ist $f\in\FF$.
\end{propenum}
Dann ist $\FF=B(\R)$.\fishhere
\end{prop}

Man kann dies auch als Definition von $B(\R)$ sehen.

\begin{proof}
Siehe \cite{Wer07}.\qedhere
\end{proof}

\begin{bem*}
Nach Satz \ref{prop:6.1} ist $C_0(\R)$ dicht in $B(\R)$ bezüglich der Toplogie
der punktweisen Konvergenz. Diese Topologie erfüllt jedoch nicht das erste
Abzählbarkeitsaxiom. D.h. obwohl $C_0(\R)$ in $B(\R)$ dicht ist, lässt sich zu
einem $f\in B(\R)$ im Allgemeinen keine Folge $(f_n)$ in $B(\R)$ finden mit
$f_n\pto f$.\maphere
\end{bem*}

\begin{defn*}
Sei $\LL(\HH)$ die Menge der beschränkten linearen Operatoren $B:\HH\to~\HH$.
Eine Folge $(B_n)$ in $\LL(\HH)$ \emph{konvergiert
stark}\index{Konvergenz!starke}
gegen einen Operator $B\in \LL(\HH)$, in Zeichen
\begin{align*}
B = \slim\limits_{n\to\infty} B_n,
\end{align*}
falls $B\ph =\lim\limits_{n\to\infty} B_n\ph$ für jedes $\ph\in\HH$.\fishhere
\end{defn*}

\begin{thm}[Messbarer Funktionalkalkül]
\label{prop:6.2}
Zu jedem selbstadjungierten Operator $A: D\subset\HH\to~\HH$ gibt es eine
eindeutig bestimmte Abbildung
\begin{align*}
\Phi: B(\R)\to\LL(\HH)
\end{align*}
mit folgenden Eigenschaften:
\begin{propenum}
\item\label{prop:6.2:1} Falls $f_z(x) = (z-x)^{-1}$ mit $z\in\C\setminus \R$,
dann ist
\begin{align*}
\Phi(f_z) = (z-A)^{-1}.
\end{align*}
\item\label{prop:6.2:2} $\Phi$ ist ein *-Homomorphismus, d.h. für alle $f,g\in
B(\R)$ gilt
\begin{equivenum}
\item $\Phi(\alpha f + \beta g) = \alpha \Phi(f) + \beta \Phi(g)$ für
$\alpha,\beta\in\C$,
\item $\Phi(f)\Phi(g) = \Phi(fg)$,
\item $\Phi(\bar{f}) = \Phi(f)^*$.
\end{equivenum}
\item\label{prop:6.2:3} Falls $f_k\pto f$, dann gilt
\begin{align*}
\Phi(f) = \slim\limits_{n\to\infty} \Phi(f_k).
\end{align*}
\end{propenum}
% Diese Eigenschaften bestimmen $\Phi$ eindeutig.
Weiterhin gilt für alle $f\in B(\R)$:
\begin{propenum}
\item[4)]\label{prop:6.2:4} $\norm{\Phi(f)} \le \norm{f}_\infty$.
\item[5)]\label{prop:6.2:5} Ist $f\ge 0$, so ist $\Phi(f)\ge 0$ (d.h.
$\lin{\ph,\Phi(f)\ph} \ge 0$ für alle $\ph\in\HH$).
\end{propenum}
Statt $\Phi(f)$ schreibt man in der Regel $f(A)$.\fishhere
\end{thm}

\begin{bsp*}
\begin{bspenum}
\item Sei $A=M_x$ der Multiplikationsoperator
\begin{align*}
(A\ph)(x) = x\ph(x),\qquad D_A = \setdef{\ph\in L^2(\R)}{x\ph(x)\in\L^2(\R)}.
\end{align*}
Dann ist $A=A^*$ und für jede Funktion $f\in B(\R)$ gilt
\[
\Phi(f) = M_f,\qquad \text{d.h. } [\Phi(f)\ph](x) =f(x)\ph(x),\quad \ph\in
L^2(\R).
\]
\begin{proof}
Es genügt zu zeigen, dass $f\mapsto M_f$ die Eigenschaften
\ref{prop:6.2:1}-\ref{prop:6.2:3} aus Theorem \ref{prop:6.2} besitzt. Dann folgt
aus der Eindeutigkeit des Funktionalkalküls, dass $\Phi(f) = M_f$.\qedhere

% Zunächst ist $f\in B(\R)$ beschränkt, d.h. $D(M_f) = L^2(\R)$, denn für jedes
% $\ph\in L^2(\R)$ ist auch $f\ph\in L^2(\R)$.
% 
%  ``\ref{prop:6.2:1}'': Sei $f_z = (z-x)^{-1}$, dann ist für $\ph\in L^2(\R)$
% \begin{align*}
% [(z-M_{x})M_{f_z}\ph](x) &= [zM_{f_z}\ph - M_{x}M_{f_z}\ph](x)\\
% &= zf_z(x)\ph(x) - x [M_{f_z}\ph](x)\\
% &= zf_z(x)\ph(x) - x f_z(x)\ph(x)\\
% &= (z-x)f_z(x)\ph(x) = \ph(x).
% \end{align*}
% Also ist $M_{f_z} = (z-M_{x})^{-1}$.
% 
% ``\ref{prop:6.2:2}'': Seien $f,g\in B(\R)$, dann ist für $\ph\in L^2(\R)$
% \begin{align*}
% [M_{\alpha f + \beta g}\ph](x)
% &= (\alpha f + \beta g)(x)\ph(x)
% = \alpha f(x)\ph(x) + \beta g(x)\ph(x)\\
% &= \alpha [M_{f}\ph](x) +\beta [M_{g}\ph](x)\\
% [M_{fg}\ph](x) &=
% (fg)(x)\ph(x) = f(x)g(x)\ph(x)
% = f(x)[M_g\ph](x)
% \end{align*}
% Letztlich gilt für $\eta,\ph\in L^2(\R)$, dass
% \begin{align*}
% \lin{\ph, M_{f}\eta}
% =\lin{\ph, f\eta}
% = \lin{\bar{f}\ph,\eta}
% = \lin{M_{\bar{f}}\ph,\eta}.
% \end{align*}
% ``\ref{prop:6.2:3}'': Sei $f_k\pto f$ und $\ph\in L^2(\R)$, dann ist
% \begin{align*}
% (M_{f_k}\ph)(x) = f_k(x)\ph(x) \to f(x)\ph(x)  = [M_f\ph](x)
% \end{align*}
% für $x\in\R$. Mit dem Satz von Lebesgue, folgt somit auch
% $\norm{M_{f_k}\ph-M_{f}\ph}_2\to 0$. Somit ist $\Phi(f) = f(M_x) =
% M_{f}$.\qedhere
\end{proof}
\item Sei $A = -i\frac{\diffd}{\dx}$ in $L^2(\R)$ mit $D_A=H^1(\R)$. Dann ist
$A=A^*$ und
\begin{align*}
f(A) = \FF^{-1}f(p)\FF,
\end{align*}
wobei $f(p)$ für den Multiplikationsoperator $M_f$ steht.
\begin{proof}
Analog zu Beispiel 1.\qedhere
\end{proof}
\item Sei $A=-\Delta$ in $L^2(\R)$ und $D_A=H^2(\R)$. Dann ist $A=A^*$ und
\begin{align*}
f(A) = \FF^{-1}f(p^2)\FF.\bsphere
\end{align*}
\end{bspenum}
\end{bsp*}

Für alles Weitere sei $\HH$ ein Hilbertraum und $A: D\subset \HH\to \HH$ ein
linearer Operator auf $\HH$.

\begin{prop}
\label{prop:6.3}
Sei $A=A^*$ in $\HH$ und $f\in C_0(\R)$, dann gilt
\begin{align*}
f(A) = \slim\limits_{\ep\downarrow 0}
\frac{1}{2\pi i}
\int_\R f(t) \left(\frac{1}{t-i\ep -A} - \frac{1}{t+i\ep-A}\right)\dt.\fishhere
\end{align*}
\end{prop}
\begin{proof}
% Man rechnet leicht nach, dass
% \begin{align*}
% \frac{1}{2\pi i}
% \int_\R f(t) \left(\frac{1}{t-i\ep -x} - \frac{1}{t+i\ep-x}\right)\dt
% \pto f(x),\quad \ep \to 0.\tag{*}
% \end{align*}
% Nach Eigenschaft \ref{prop:6.2:3} des messbaren Funktionalkalküls ist daher
% \begin{align*}
% f(A) = \slim\limits_{\ep\to0}
% \Phi\left(\frac{1}{2\pi i}
% \int_\R f(t) \left(\frac{1}{t-i\ep -x} - \frac{1}{t+i\ep-x}\right)\dt
% \right).
% \end{align*}
% Weiterhin ist $t\pm i\ep\notin \C\setminus\R$ für $\ep >
% 0$ und folglich nach Eigenschaft \ref{prop:6.2:1} und \ref{prop:6.2:2},
% \begin{align*}
% &\frac{1}{2\pi i}
% \int_\R f(t) \left(\frac{1}{t-i\ep -A} - \frac{1}{t+i\ep-A}\right)\dt\\
% &= 
% \frac{1}{2\pi i}
% \int_\R f(t) \left(\Phi((t-i\ep-x)^{-1}) - \Phi((t+i\ep-x)^{-1})\right)\dt\\
% &=
% \frac{1}{2\pi i}
% \int_\R \Phi\left(f(t) \left((t-i\ep-x)^{-1} -
% (t+i\ep-x)^{-1}\right)\right)\dt.
% \end{align*}
% Interpretieren wir das Integral als Riemann-Integral, dann konvergiert die
% Riemannsumme punktweise, d.h. wir können $\Phi$ und $\int$ vertauschen,
% \begin{align*}
% \ldots =
% \Phi\left(\frac{1}{2\pi i}
% \int_\R f(t) \left((t-i\ep-x)^{-1} -
% (t+i\ep-x)^{-1}\right)\dt\right)
% \end{align*}
% und die Behauptung folgt.
Übung.\qedhere
\end{proof}

\section{Spektralmaß}

\begin{prop}
\label{prop:6.4}
Sei $A=A^*$ und für jede Borel-Menge $\Omega\subset\R$ sei
\begin{align*}
P_\Omega := \chi_\Omega(A) = \Phi(\chi_\Omega).
\end{align*}
Dann gelten:
\begin{propenum}
\item\label{prop:6.4:1} $P_\Omega$ ist Orthogonalprojektor, d.h.
$P_\Omega^2=P_\Omega$ und $P_\Omega^*=P_\Omega$.
\item\label{prop:6.4:2} $P_\varnothing = 0$ und $P_\R = \Id$.
\item\label{prop:6.4:3} Falls $\Omega = \bigcup_{k=1}^\infty \Omega_k$ und
$\Omega_i\cap \Omega_k = \varnothing$ für $i\neq k$, dann ist
\begin{align*}
P_\Omega = \slim\limits_{N\to\infty} \sum_{k=1}^N P_{\Omega_k}.
\end{align*}
\item\label{prop:6.4:4} $P_\Omega P_{\Omega'} = P_{\Omega \cap
\Omega'}$.\fishhere
\end{propenum}
\end{prop}
\begin{proof}
\begin{proofenum}
\item Man rechnet direkt nach, dass
\begin{align*}
&P_\Omega P_\Omega = \Phi(\chi_\Omega)\Phi(\chi_\Omega)
= \Phi(\chi_\Omega^2) = P_\Omega,\\
&P_\Omega^* = \Phi(\chi_\Omega)^* = \Phi(\bar{\chi_\Omega}) = P_\Omega.
\end{align*}
\item Eine leichte Übung zeigt, $\Phi(1) = \Id$ und folglich ist
\begin{align*}
P_\R = \Phi(\chi_{\R}) = \Phi(1) = \Id,\\
P_\varnothing = \Phi(\chi_\varnothing) = \Phi(0) = 0.
\end{align*}
\item Es gilt
\begin{align*}
\sum_{k=1}^N P_{\Omega_k} 
= \sum_{k=1}^N \Phi(\chi_{\Omega_k})
= \Phi\left(\sum_{k=1}^N \chi_{\Omega_k} \right)
= \Phi\left(\chi_{\bigcup_{k=1}^N\Omega_k} \right),
\end{align*}
wobei 
\begin{align*}
\chi_{\bigcup_{k=1}^N\Omega_k} \pto \chi_{\Omega}
\end{align*}
also
\begin{align*}
P_\Omega =\Phi(\chi_\Omega) = \slim\limits_{N\to\infty}
\Phi\left(\chi_{\bigcup_{k=1}^N \Omega_k}\right)
= \slim\limits_{N\to\infty}\sum_{k=1}^N P_{\Omega_k}
\end{align*}
\item $P_\Omega P_{\Omega'} = \Phi(\chi_{\Omega}\chi_{\Omega'}) =
\Phi(\chi_{\Omega\cap \Omega'}) = P_{\Omega\cap \Omega'}$.\qedhere
\end{proofenum}
\end{proof}

\begin{defn*}
Jede Abbildung $P: \BB\to \LL(\HH)$ mit den Eigenschaften
\ref{prop:6.4:1}-\ref{prop:6.4:3} aus Satz \ref{prop:6.4} heißt
\emph{projektionswertiges Maß}\index{projektionswertiges Maß}.

Eigenschaft \ref{prop:6.4:4} folgt aus \ref{prop:6.4:1}-\ref{prop:6.4:3} und
außerdem gelten
\begin{align*}
&\Omega_1\subset\Omega_2 &&\Rightarrow
P_{\Omega_1}\le P_{\Omega_2}, && \text{Monotonie}\\
&\Omega = \bigcup_{k=1}^N \Omega_k &&\Rightarrow
P_\Omega \le \sum_{k=1}^N P_{\Omega_k}, &&
\text{Subadditivität}.
\end{align*}
% Für jedes $\ph\in\HH$ ist die Abbildung
% \begin{align*}
% \mu: \BB\to\R,\quad \Omega\mapsto \lin{\ph,P_\Omega \ph}
% \end{align*}
% ein Borel-Maß.

Nach dem messbaren Funktionalkalkül gibt es zu jedem selbstadjungierten Operator
$A$ ein projektionswertiges Maß. Dieses heißt auch
\emph{Spektralmaß}.\index{Spektralmaß}\fishhere
\end{defn*}
\begin{proof}
Siehe \cite{Tes09}.\qedhere
% Habe $P:\BB\to\LL(\HH)$ die Eigenschaften \ref{prop:6.4:1}-\ref{prop:6.4:3}.
% Dann gilt für jedes $A\in\BB$,
% \begin{align*}
% \Id = P_{\R} = P_{\R\setminus A \dcup A}
% = P_{\R\setminus A} + P_A.
% \end{align*}
% Insbesondere ist
% \begin{align*}
% P_AP_{\R\setminus A} = P_A(\Id-P_A) = P_A- P_A = 0.
% \end{align*}
% 
% Wir zeigen, dass für jedes $\ph\in\HH$ die Abbildung $\mu$ ein Borel-Maß ist.
% Monotonie und Subadditivität von $P$ folgen dann sofort.
% 
% Für jedes $\Omega\in\BB$ ist 
% \begin{align*}
% \mu(\Omega) &= \lin{\ph,P_\Omega\ph}
% = \lin{P_\Omega\ph,P_\Omega\ph}
% + \lin{P_{\R\setminus\Omega}\ph,P_\Omega\ph}\\
% &= \lin{P_\Omega\ph,P_\Omega\ph}
% + \underbrace{\lin{\ph,P_{\R\setminus\Omega}P_\Omega\ph}}_{=0}
% = \norm{P_\Omega\ph}^2 \ge 0.
% \end{align*}
% Weiterhin ist
% \begin{align*}
% \mu(\varnothing) = \lin{\ph,P_\varnothing\ph} = 0
% \end{align*} 
% und für eine Familie $(\Omega_i)_{i\ge N}$ mit $\Omega_i\cap
% \Omega_k=\varnothing$, falls $i\neq k$ gilt
% \begin{align*}
% \mu\left(\bigcup_{i\ge 1} \Omega_k\right)
% &= \lin{\ph,P_{\bigcup_{i\ge 1} \Omega_k}\ph}
% = \lin{\ph,\slim_{N\to\infty} \bigcup_{i=1}^N \Omega_k \ph}
% = \lim_{N\to\infty}\sum_{k=1}^N \lin{\ph,P_{\Omega_k}\ph}\\
% &= \sum_{k=1}^\infty \lin{\ph,P_{\Omega_k}\ph}.
% \end{align*}
% 
% Zum Nachweis von ``\ref{prop:6.4:4}''
% bemerken wir dass für $A,B\in \BB$ disjunkt gilt $A\subset\R\setminus B$,
% \begin{align*}
% 0\le P_AP_B \le P_{\R\setminus B}P_B = 0,
% \end{align*}
% und folglich $P_AP_B = 0$. Seien 
% $\Omega,\Omega'\in \BB$ und
% weiterhin
% \begin{align*} 
% &(\Omega\cap \Omega')\dcup A = \Omega,\qquad
% (\Omega\cap \Omega')\dcup B = \Omega',
% \end{align*}
% mit $A,B\in\BB$. Dann sind $A$, $B$ und $\Omega\cap \Omega'$ disjunkt und
% \begin{align*}
% P_\Omega P_{\Omega'}
% &= P_{(\Omega\cap \Omega')\dcup A}P_{(\Omega\cap \Omega')\dcup B}
% = P_{(\Omega\cap \Omega')} + 
% P_{(\Omega\cap \Omega')}P_{B}
% + P_AP_{(\Omega\cap \Omega')}
% + P_AP_B\\
% &= P_{(\Omega\cap \Omega')}.\qedhere
% \end{align*}
\end{proof}


\begin{defn*}
Der Träger $\supp(P)$ eines projektionswertigen Maßes $P$ ist definiert durch
\begin{align*}
x\in \supp(P) \Leftrightarrow P(U)\neq 0 \text{ für jede offene Umgebung
}U\text{ von }P.\fishhere
\end{align*} 
\end{defn*}

\begin{lem}
\label{prop:6.5}
Sei $A=A^*$ mit Spektralmaß $P$, dann gelten
\begin{propenum}
\item $S=\supp(P)$ ist abgeschlossen.
\item\label{lem:6.4*:2} $P_S=\Id$ und $P_{R\setminus S} = 0$.
\item Für jede Funktion $f\in B(\R)\cap C(\R)$ gilt
\begin{align*}
\norm{f(A)} = \sup_{x\in S}\abs{f(x)}.\fishhere
\end{align*}
\end{propenum}
\end{lem}

\begin{proof}
\begin{proofenum}
\item Sei $x\notin S$, dann existiert eine offene Umgebung $U$ von $x$ mit $P_U
= 0$. Also ist auch $y\notin S$ für jedes $y\in U$. Somit ist $\R\setminus S$
offen und $S$ ist abgeschlossen.
\item Sei $K\subset\R\setminus S$. Dann existieren offene Mengen
\begin{align*}
\Omega_1,\ldots,\Omega_N \subset \R
\text{ mit }
K \subset \bigcup_{k=1}^N \Omega_k\text{ und }P_{\Omega_k} = 0. 
\end{align*}
Also ist auch
\begin{align*}
0\le P_{K} \le P_{\bigcup_{k=1}^N \Omega_k} \le \sum_{k=1}^N P_{\Omega_k} = 0, 
\end{align*}
und somit $P_K=0$. Sei nun $K_m = \setdef{x\in\R}{\abs{x}\le m\text{ und
}\dist(x,S)\ge \frac{1}{m}}\subset \R\setminus S$. Dann ist $K_m$ kompakt,
$P_{K_m} = 0$ und
\begin{align*}
\bigcup_{m=1}^\infty K_m = \R\setminus S. 
\end{align*}
Daraus folgt
\begin{align*}
P_{\R\setminus S} \le \sum_{m=1}^\infty P_{K_m} = 0,
\end{align*}
also ist $P_{\R\setminus S} =0$. Es folgt
\begin{align*}
P_S = P_S + P_{\R\setminus S} = P_\R = \Id.
\end{align*}
\item Nach \ref{lem:6.4*:2} gilt
\begin{align*}
\norm{\Phi(f)} = \norm{\Phi(\chi_S)\Phi(f)} = \norm{\Phi(\chi_S f)} \le
\sup_{x\in S}\abs{f(x)}.
\end{align*}
Sei $x\in S$. Es genügt nun zu zeigen, dass $\norm{\phi(f)} \ge \abs{f(x)}$.
Dies folgt trivialerweise, wenn $f(x) = 0$. Für $f(x)\neq 0$ gibt es ein $\ep >
0$ so klein, dass
\begin{align*}
\abs{f(x)} - \ep \ge 0.
\end{align*}
Die Abbildung $x\mapsto \abs{f(x)}$ ist stetig, es gibt daher eine offene
Umgebung $U$ von $x$ mit
\begin{align*}
y\in U \Rightarrow \abs{f(y)} > \abs{f(x)} - \ep \ge 0.
\end{align*}
Da $x\in S$, ist $P_U \neq 0$ und somit existiert ein $\ph\in\HH$ mit
$\norm{\ph}=1$ und $P_U\ph = \ph$. Somit gilt
\begin{align*}
\norm{\Phi(f)\ph}^2 &= \lin{\ph,\Phi(\abs{f}^2)\ph}
= \lin{\ph,\Phi(\abs{f}^2)\Phi(\chi_U)\ph}\\
&= \lin{\ph,\Phi(\chi_U \abs{f}^2)\ph}
\ge (\abs{f(x)}-\ep)^2 \lin{\ph,\Phi(\chi_U)\ph}\\
&= (\abs{f(x)}-\ep)^2 \norm{\ph}. 
\end{align*}
Somit ist $\norm{\Phi(f)} \ge \abs{f(x)}-\ep$ für jedes $\ep > 0$, also gilt
auch $\norm{\Phi(f)} \ge \abs{f(x)}$.\qedhere
\end{proofenum}
\end{proof}

\begin{prop}
\label{prop:6.6}
Sei $A=A^*$ mit Spektralmaß $P$. Dann gelten
\begin{propenum}
\item $\sigma(A) = \supp P$.
\item $\norm{(z-A)^{-1}} = \dist(z,\sigma(A))^{-1}$ für alle
$z\in\rho(A)$.\fishhere
\end{propenum}
\end{prop}
\begin{proof}
Sei $S:=\supp P$, so gilt für $z\in\C\setminus\R$ nach Lemma \ref{prop:6.5},
\begin{align*}
\norm{(z-A)^{-1}} = \norm{\Phi(f_z)} = \sup_{x\in S} \norm{(z-x)^{-1}} =
\dist(z,S)^{-1}.\tag{*}
\end{align*}
Für $\lambda\in\R$ sei $z_n = \lambda + in^{-1}$. Dann folgt
\begin{align*}
\lambda\in\sigma(A) \Leftrightarrow \norm{(z_n-A)^{-1}} \to \infty
\Leftrightarrow \dist(z_n,S) \to 0
\Leftrightarrow \lambda\in \bar{S} = S.
\end{align*}
Also ist $S=\sigma(A)$.

Für $\lambda\in\R\setminus \sigma(A)$ folgt nun die zweite Behauptung aus (*) im
Limes für $z\to \lambda$.\qedhere
\end{proof}

\begin{lem}
\label{prop:6.7}
Für jeden linearen Operator $B: \HH\to \HH$ in einem Prähilbertraum $\HH$ und
alle $\ph,\psi\in \HH$ gilt die Polarisationsidentität
\begin{align*}
\lin{\ph, B\psi} &= \frac{1}{4}\left[\lin{\ph+\psi,B(\ph+\psi)} -
\lin{\ph-\psi,B(\ph-\psi)} \right]\\
&+ \frac{1}{4i}\left[\lin{\ph+i\psi,B(\ph+i\psi)} -
\lin{\ph-i\psi,B(\ph-i\psi)}\right].\fishhere
\end{align*}
\end{lem}
\begin{proof}
Der Beweis ist eine leichte Übung.\qedhere
\end{proof}

\begin{prop}
\label{prop:6.8}
Sei $A=A^*$ mit Spektralmaß $P$ und sei $f\in B(\R)$. Dann gilt für alle
$\ph,\psi\in\HH$,
\begin{align*}
\lin{\ph,f(A)\psi} &= \int f(\lambda) \dmu_{\ph,\psi}(\lambda)\\
&:= 
\frac{1}{4}\left[\int f \dmu_{\ph+\psi} - \int f\dmu_{\ph-\psi} \right]
+ \frac{1}{4i}\left[\int f \dmu_{\ph+i\psi} - \int f\dmu_{\ph-i\psi}
\right],
\end{align*}
wobei $\mu_\ph(\Omega) = \lin{\ph,P(\Omega)\ph}$.\fishhere
\end{prop}

\begin{bem*}
$\mu_{\ph,\psi}:= \lin{\ph,P(\Omega)\psi}$ ist ein komplexes Borelmaß. Eine
Einführung in die allgemeine Integrationstheorie bezüglich komplexer Borelmaße
würde den Rahmen dieser Vorlesung sprengen. Aufgrund der Zerlegung des
Integrals, wie in Satz \ref{prop:6.8} angegeben, ist dies jedoch auch nicht
notwendig, da wir uns auf gewöhnliche Integrale zurückziehen können.\maphere
\end{bem*}

\begin{proof}
Ist $f$ eine elementare Funktion
\begin{align*}
f = \sum_{k=1}^n c_k \chi_{\Omega_k},\qquad \Omega_k\in\BB,
\end{align*}
so gilt
\begin{align*}
\lin{\ph,\Phi(f)\ph} &= \lin{\ph,\sum_{k=1}^n c_k
\Phi\left(\chi_{\Omega_k}\right)\ph}
= \sum_{k=1}^n c_k \lin{\ph,P(\Omega_k)\ph}
= \sum_{k=1}^n c_k \mu_{\ph}(\Omega)\\
&= \int f(\lambda) \dmu_\ph(\lambda).
\end{align*}
Zu jeder Funktion $f\in B(\R)$ gibt es eine Folge $(f_n)$ von elementaren
Funktionen $f_n : \R\to \C$ mit $f_n\pto f$. Somit gilt
\begin{align*}
\lin{\ph,\Phi(f)\ph} = \lim\limits_{n\to\infty} \lin{\ph,\Phi(f_n)\ph}
= \lim\limits_{n\to\infty} \int f_k \dmu_\ph
= \int f\dmu_\ph,
\end{align*}
mit dem Satz von Lebesgue und der Tatsache, dass $\mu_\ph$ ein endliches Maß
und damit jede beschränkte Funktion $\mu_\ph$-integrierbar ist.

Um den Beweis abzuschließen, ist noch die Polarisationsidentität auf
$\lin{\ph,f(A)\psi}$ anzuwenden.\qedhere
\end{proof}

\section{Spektralsatz}

Der messbare Funktionalkalkül $\Phi: B(\R)\to \LL(\HH)$ lässt sich auch auf
unbeschränkte borelmessbare Funktionen $f: \R\to\C$ ausdehnen. Der Operator
$\Phi(f)$ ist dann in der Regel unbeschränkt und definiert durch
\begin{align*}
D(\Phi(f)) &= \Phi\left(\frac{1}{1+\abs{f}} \right)\HH,\\
\Phi(f)\ph &= \Phi\left(\frac{f}{1+\abs{f}} \right)\gamma,\qquad\text{falls }
\ph = \Phi\left(\frac{1}{1+\abs{f}} \right)\gamma.
\end{align*}

\begin{prop}
\label{prop:6.9}
Mit obigen Notationen gilt
\begin{propenum}
\item $\Phi(f)$ ist wohl- und dicht definiert.
\item\label{prop:6.9:2} $D(\Phi(f)) = \setdef{\ph\in\HH}{\int \abs{f}^2\dmu_\ph
< \infty}$ und für alle $\ph,\psi\in D(\Phi(f))$ gilt
\begin{align*}
&\norm{\Phi(f)\ph}^2 = \int \abs{f}^2\dmu_\ph\\
&\lin{\ph,\Phi(f)\psi} = \int f \dmu_{\ph,\psi}
\end{align*}
\item Ist $(f_k)$ eine Folge in $B(\R)$ mit $f_k(x)\to f(x)$ für alle $x\in\R$
und $\abs{f_k}\le \abs{f}$ für alle $k$, dann gilt für alle $\ph\in D(\Phi(f))$,
\begin{align*}
\Phi(f_k)\ph \to \Phi(f)\ph,\qquad k\to\infty.\fishhere
\end{align*}
\end{propenum}
\end{prop}

\begin{bem*}
Alle Integrale in \ref{prop:6.9:2} können durch $\int_{\sigma(A)}$ ersetzt
werden.\maphere
\end{bem*}

\begin{proof}
Siehe \cite{RS95a}, \cite{Wer07} oder \cite[Kapitel 12]{Mmq08}.\qedhere
\end{proof}

\begin{thm}[Spektralsatz]
\label{prop:6.10}
Zu jedem selbstadjungierten Operator $A$ gibt es ein eindeutig bestimmtes
Spektralmaß $P$ mit
\begin{align*}
&\lin{\ph, A\psi} = \int_{\sigma(A)} \lambda \dmu_{\ph,\psi}\tag{1},\\
&\mu_\ph(\Omega) := \lin{\ph,P(\Omega)\ph}\tag{2}
\end{align*}
für alle $\ph,\psi\in D(A)$. Umgekehrt wird zu jedem projektionswertigen Maß $P$
durch (1) und (2) ein selbstadjungierter Operator $A$ definiert.\fishhere
\end{thm}
\begin{proof}
Übung.\qedhere
\end{proof}

\begin{bem*}[Bemerkungen.]
\begin{bemenum}
\item Man schreibt (1) oft in der Form
\begin{align*}
A = \int_{\sigma(A)} \lambda \dP(\lambda).
\end{align*}
Das ist die Verallgemeinerung der Spektralzerlegung
\begin{align*}
A = \sum_{k=1}^m \lambda_k P_k
\end{align*}
eines selbstadjungierten Operators $A\in\LL(\C^n)$ mit Eigenwerten
$\lambda_1,\ldots,\lambda_m$ und zugehörigen Eigenprojektoren $P_1,\ldots,P_m$.

\item
\emph{Spektrale Unterräume:}
Sei $\Omega \in \R$ eine Borelmenge,
$P_{\Omega} = \phi(\chi_{\Omega})$ und $\HH_{\Omega} = P_{\Omega}\HH$.
Dann ist $P_{\Omega}A \subset AP_{\Omega}$, und
\begin{align*}
A\big|_{\HH_{\Omega}} : P_{\Omega} D(A) \subset \HH_{\Omega}
\to \HH_{\Omega}
\end{align*}
ist selbstadjungiert (Übung). Ist $\Omega$ offen, dann
\begin{align*}
\sigma(A)\cap\Omega\subset\sigma(A\big|_{\HH_{\Omega}}) \subset
\sigma(A)\cap\bar{\Omega}.
\end{align*}
Insbesondere ist $A \big|_{\HH_{\Omega}}$
beschränkt, wenn $\Omega$ beschränkt ist.
\begin{proof}
Für $\lambda\in \sigma(A)\cap\Omega$ und alle $\ep>0$ klein genug
gilt $(\lambda-\ep,\lambda+\ep)\subset\Omega$ und
$P_{(\lambda-\ep,\lambda+\ep)}\neq 0$. Also
$\lambda\in\sigma(A\big|_{\HH_{\Omega}})$.

Falls $\lambda \notin\sigma(A) \cap\overline{\Omega} := B$
dann ist $\delta := \dist(\lambda, B) > 0$ und somit
für alle $\ph\in P_{\Omega} D(A)$,
\begin{align*}
\norm{(A - \lambda)\ph)}^2
=\int_{\sigma(A) \cap \Omega}\abs{t-\lambda}^2 \, \dmu_{\ph}(t)
\geq \delta^2 \int_{\sigma(A) \cap \Omega} \dmu_{\ph}(t)
= \delta^2 \lin{\ph,\ph}.
\end{align*}
Also $\lambda\notin \sigma(A\big|_{\HH_{\Omega}})$.\qedhere
\end{proof}

\item
$\lambda \in \R$ ist genau dann ein Eigenwert von $A$,
wenn $P_{\setd{\lambda}} \ne 0$ ist.
Dann ist $P_{\setd{\lambda}}$ der Orthogonalprojektor auf den Eigenraum
zu $\lambda$. (siehe Aufgabenblatt 6)

\item In der Quantenmechanik wird jede beobachtbare Größe (Ort, Implus, Spin,
\ldots) durch einen selbstadjungierten Operator beschrieben. Die normierten
Vektoren $\ph\in \HH$ beschreiben die (reinen) Zustände des Systems. Ist
$\Omega\subset\R$ eine Borelmenge, dann ist
\begin{align*}
\mu_\ph(\Omega) = \lin{\ph,P(\Omega)\ph},\qquad P(\Omega) = \Phi(\chi_\Omega), 
\end{align*}
die Wahrscheinlichkeit bei einer Messung der Größe $A$ einen Wert in $\Omega$ zu
finden. Wegen $\mu_\ph(\R\setminus\sigma(A)) = 0$ können nur Werte in
$\sigma(A)$ gefunden werden.

Der Erwartungswert
\begin{align*}
\lin{\ph,A\ph} = \int_{\sigma(A)} \lambda\dmu_\ph(\lambda),\qquad \ph\in D(A),\;
\norm{\ph} = 1
\end{align*}
kann allerdings jeden Wert zwischen $\inf \sigma(A)$ und $\sup \sigma(A)$
annehmen.

Zum Weiterlesen siehe \cite{Stra} oder \cite{Thi94}.\maphere
\end{bemenum}
\end{bem*}

\begin{defn*}
\newcommand{\disc}{{\mathrm{disc}}}
\newcommand{\ess}{{\mathrm{ess}}}
Das \emph{diskrete} und das \emph{wesentliche} Spektrum eines selbstadjungierten
Operators $A$ sind definiert durch
\begin{align*}
&\sigma_\disc = \setdef{\lambda\in\R}{\lambda\text{ ist isolierter Eigenwert
von $A$ mit endlicher Vielfachheit}}\\
&\sigma_\ess = \sigma(A)\setminus \sigma_\disc.\fishhere
\end{align*}
\end{defn*}

\begin{prop}
\label{prop:6.11}
Sei $A=A^*$ und $P_\Omega(A) = \chi_\Omega(A)$. Dann gelten
\begin{propenum}
\item $\lambda\in\sigma(A) \Leftrightarrow P_{(\lambda-\ep,\lambda+\ep)}(A)\neq
0$ für alle $\ep > 0$,
\item $\lambda\in \sigma_\mathrm{disc}(A) \Leftrightarrow \lambda\in \sigma(A)$
und es gibt ein $\ep > 0$ mit $\dim P_{(\lambda-\ep,\lambda+\ep)}(A)\HH <
\infty$.
\item $\lambda\in\sigma_{\mathrm{ess}}(A) \Leftrightarrow \dim
P_{(\lambda-\ep,\lambda + \ep)}(A)\HH = \infty$ für alle $\ep > 0$.\fishhere
\end{propenum}
\end{prop}

\begin{proof}
\newcommand{\disc}{{\mathrm{disc}}}
\newcommand{\ess}{{\mathrm{ess}}}
\begin{proofenum}
\item
Nach Satz \ref{prop:6.5} gilt
\begin{align*}
\lambda \in \sigma(A) &
\Leftrightarrow P_{\Omega}(A) \ne 0 \text{ für jede offene Menge }
\Omega \ni \lambda.\\
& \Leftrightarrow
P_{(\lambda - \ep, \lambda + \ep)} (A) \ne 0 \text{ für alle }
\ep > 0.
\end{align*}

\item
Nach der Definition des diskreten Spektrums und Aufgabenblatt 6 gilt
\begin{align*}
\lambda \in \sigma_{\disc}(A) \Leftrightarrow
\begin{cases}
\lambda \text{ ist Eigenwert von $A$ mit } \dim P_{\setd{\lambda}} < \infty\\
\text{ und }
B_{\ep}(\lambda) \cap \sigma(A) = \setd{\lambda}
\ \text{für ein}\ \ep > 0.
\end{cases}
\end{align*}
Aus $\lambda \in \sigma_\disc(A)$ folgt also
$P_{(\lambda - \ep, \lambda + \ep)} (A) = P_{\setd{\lambda}}$
und somit $\dim P_{(\lambda - \ep, \lambda + \ep)} \HH < \infty$.

Ist umgekehrt $\lambda \in \sigma(A)$ und
$n = \dim P_{(\lambda - \ep, \lambda + \ep)} \HH < \infty$.
Dann gilt,
\begin{align*}
\sigma(A)\cap(\lambda-\ep,\lambda+\ep)\subset\sigma(A \big|_{
P_{(\lambda - \ep, \lambda + \ep)}(A) \HH})
= \setd{ \lambda_1, \ldots, \lambda_m},\qquad m\leq n.
\end{align*}
Also ist $\lambda$ ein isolierter Eigenwert von $A$ mit endlicher Vielfachheit.

\item
Folgt aus (i) und (ii).\qedhere
\end{proofenum}
\end{proof}

\chapter{Kompakte Operatoren und Stabilität des wesentlichen Spektrums}

\begin{defn*}
Eine Folge $(\ph_n)$ im Hilbertraum $\HH$ konvergiert
\emph{schwach}\index{schwache Konvergenz} gegen $\ph\in \HH$, in Zeichen
\begin{align*}
\ph_n \wto \ph,\qquad n\to \infty,
\end{align*}
wenn für alle $\psi\in\HH$ gilt,
\begin{align*}
\lim\limits_{n\to\infty} \lin{\psi,\ph_n} = \lin{\psi,\ph}.\fishhere
\end{align*}
\end{defn*}

\begin{bem*}[Bemerkungen.]
\begin{bemenum}
\item Jede Orthonormalfolge $(\ph_n)$ konvegiert schwach gegen Null, denn für
alle $\psi\in\HH$ ist
\begin{align*}
\sum_{n\ge 0} \abs{\lin{\psi,\ph_n}}^2 \le \norm{\psi}^2.
\end{align*}
\item Aus $\ph_n\to \ph$ folgt $\ph_n\wto \ph$. Umgekehrt folgt aus
\begin{align*}
\ph_n\wto \ph,\qquad \norm{\ph_n}\to \norm{\ph}
\end{align*}
auch $\ph_n \to\ph$.\maphere
\end{bemenum}
\end{bem*}

\begin{thm}[Theorem von Weyl]
\label{prop:7.1}
Sei $A$ selbstadjungiert. Die Zahl $\lambda\in\R$ ist genau dann in
$\sigma_\mathrm{ess}(A)$, wenn eine \emph{Weylfolge} $(\ph_n)$ in $D(A)$
existiert, d.h. eine Folge mit 
\begin{align*}
\norm{\ph_n} =1,\; \ph_n\wto 0,\quad \text{ und}\quad
\norm{(A-\lambda)\ph_n}\to 0.\fishhere
\end{align*}
\end{thm}

\begin{proof}
``$\Rightarrow$": Sei $\lambda\in\sigma_\mathrm{ess}(A)$ und $\HH_n =
P_{(\lambda-\frac{1}{n},\lambda+\frac{1}{n})}(A)\HH$.
Dann ist $\dim\HH_n = \infty$ für jedes $n\ge 1$ nach Satz \ref{prop:6.11}. Wir
konstruieren rekursiv eine Orthonormalfolge $(\ph_n)$ mit $\ph_n\in\HH_n$.

Sei $\ph_1\in \HH_1$ mit $\norm{\ph_1}=1$. Gegeben $\ph_1,\ldots,\ph_{n-1}$
orthonormalisiert mit $\ph_k\in\HH_k$ für $k=1,\ldots,n-1$, wählen wir
$\ph_n\in\HH_n$ mit $\ph_n\bot \setd{\ph_1,\ldots,\ph_{n-1}}$ und
$\norm{\ph_n}=1$. Es gilt dann $\ph_n\wto 0$ und
\begin{align*}
\norm{(A-\lambda)\ph_n}^2 = 
\int_{\abs{t-\lambda}<\frac{1}{n}} \abs{t-\lambda}^2 \dmu_{\ph_n}(\lambda)
\le \frac{1}{n^2}\norm{\ph_n}^2 \to 0.
\end{align*}

``$\Leftarrow$'': Sei $(\ph_n)$ eine Weylfolge, dann ist $\lambda\in\sigma(A)$.
Wir führen nun die Annahme $\lambda\in\sigma_\mathrm{disc}(A)$ auf einen
Widerspruch. Angenommen $\lambda\in\sigma_\mathrm{disc}(A)$, dann existiert nach
Satz \ref{prop:6.11} ein $\ep > 0$ , so dass
\begin{align*}
\dim(P_{I_\ep}(A)\HH) < \infty,\qquad I_\ep = (\lambda-\ep,\lambda+\ep).  
\end{align*}
Weiterhin gilt $P_{I_\ep}(A) \ph_n \to 0$, denn
\begin{align*}
P_{I_\ep}(A)\ph_n = \sum_{k=1}^N \psi_k\lin{\psi_k,\ph_n}\to 0
\end{align*}
wenn $(\psi_k)_{k=1}^N$ eine ONB von $P_{I_\ep}(A)\HH$ ist. Außerdem gilt
\begin{align*}
\norm{P_{\R\setminus I_\ep}(A)\ph_n}^2 &=
\int_{\R\setminus I_\ep} \dmu_{\ph_n}(t)
<
\int_{\abs{t-\lambda}>\ep} \frac{\abs{t-\lambda}^2}{\ep^2} \dmu_{\ph_n}(t)\\
&\le
\frac{1}{\ep^2} \int \abs{t-\lambda}^2 \dmu_{\ph_n}(t)
= \frac{1}{\ep^2} \norm{(A-\lambda)\ph_n}^2 \to 0,\qquad n\to \infty.
\end{align*}
Also gilt auch
\begin{align*}
1 = \norm{\ph_n}^2 = \norm{P_{I_\ep}(A)\ph_n}^2 + \norm{P_{\R\setminus
I_\ep}(A)\ph_n}^2 \to 0,
\end{align*}
ein Widerspruch.\qedhere
\end{proof}

\begin{defn*}
Ein beschränkter Operator $B\in\LL(\HH)$ heißt
\emph{kompakt}\index{Operator!kompakter}, wenn die Bildfolge $(B\ph_n)$ jeder
beschränkten Folge $(\ph_n)$ eine konvergente Teilfolge besitzt.

$B$ heißt \emph{von endlichem Rang}\index{Operator!von endlichem Rang}, wenn
$\dim (B\HH) < \infty$.\fishhere
\end{defn*}

\begin{prop}
\label{prop:7.2}
\begin{propenum}
\item\label{prop:7.2:1} Ist $B\in\LL(\HH)$ von endlichem Rang, dann ist $B$
kompakt.
\item\label{prop:7.2:2} Sind $B, C\in\LL(\HH)$ kompakt und $\lambda,\mu\in\C$,
dann ist auch $\lambda B+ \mu C$ kompakt.
\item\label{prop:7.2:3} Ist $B$ kompakt und $C\in\LL(\HH)$, dann sind auch
$BC$ und $CB$ kompakt.
\item\label{prop:7.2:4} Ist $B_n$ eine Folge kompakter Operatoren in $\LL(\HH)$
mit $\norm{B_n-B}\to 0$, dann ist auch $B$ kompakt.
\item\label{prop:7.2:5} Ist $B$ kompakt und $\ph_n\wto \ph$, dann gilt
$B\ph_n\to B\ph$.\fishhere
\end{propenum}
\end{prop}

Die kompakten Operatoren bilden also ein abgeschlossenes Ideal der beschränkten
Operatoren und führen schwach konvergente in konvergente Folgen über.

\begin{proof}
``\ref{prop:7.2:1}-\ref{prop:7.2:5}": Übung.\qedhere
\end{proof}

\begin{thm}
\label{prop:7.3}
Sei $A=A^*$ und $B\subset B^*$ mit $D(B)\supset D(A)$. Ist $B(A+i)^{-1}$
kompakt, dann ist $A+B$ selbstadjungiert auf $D(A)$ und
\begin{align*}
\sigma_\mathrm{ess}(A+B) = \sigma_\mathrm{ess}(A).\fishhere
\end{align*}
\end{thm}

Ist $B(A+i)^{-1}$ kompakt, dann heißt $B$ \emph{relativ kompakt} bezüglich $A$.

\begin{proof}
\begin{proofenum}
\item Wir zeigen $\norm{B(A+in)^{-1}} \to 0$.

Wähle eine Folge $(\ph_n)$ in $\HH$ mit $\norm{\ph_n} = 1$ und
\begin{align*}
\norm{B(A+in)^{-1}} &\le \norm{B(A+in)^{-1}\ph_n} + \frac{1}{n}\\
&= \norm{B(A+i)^{-1}(A+i)(A+in)^{-1}\ph_n} + \frac{1}{n}.\tag{*}
\end{align*}
Die Existienz der Folge $(\ph_n)$ folgt direkt aus der Definition der Norm
$\norm{B(A+in)^{-1}}$. Aus dem Spektralsatz folgt, dass
\begin{align*}
\norm{(A+i)(A+in)^{-1}} \le 1
\end{align*}
und für alle $\gamma\in D(A)$ gilt
\begin{align*}
&\abs{\lin{\gamma, (A+i)(A+in)^{-1}\ph_n}}
= \abs{\lin{(A-i)\gamma,(A+in)^{-1}\ph_n}}\\
&\quad \le \norm{(A-i)\gamma}\norm{(A+in)^{-1}\ph_n}
\le \frac{1}{n}\norm{(A-i)\gamma}\to 0.
\end{align*}
Da $D(A)\subset\HH$ dicht liegt, folgt $(A+i)(A+in)^{-1}\ph_n\wto 0$ und damit
gilt nach Voraussetzung
\begin{align*}
B(A+i)^{-1}(A+i)(A+in)^{-1}\ph_n \to 0.
\end{align*}
Mit (*) folgt somit $\norm{B(A+in)^{-1}} \to 0$.
\item \textit{$A+B$ ist selbstadjungiert auf $D(A)$}.

Nach 1) gibt es ein $n\in\N$, so dass
\begin{align*}
a := \norm{B(A+in)^{-1}} < 1.
\end{align*}
Somit gilt für alle $\ph\in D(A)$,
\begin{align*}
\norm{B\ph} = \norm{B(A+in)^{-1}(A+in)\ph} \le a\norm{(A+in)\ph}
\le a \norm{A\ph} + an\norm{\ph}.
\end{align*}
Also ist $A+B$ selbstadjungiert auf $D(A)$ nach dem Theorem von Kato-Rellich 
\ref{prop:4.6}.

\item $\sigma_\mathrm{ess}(A+B) = \sigma_\mathrm{ess}(A)$.

``$\supset$'': Sei $\lambda\in\sigma_\mathrm{ess}(A)$ und $(\ph_n)$ eine
zugehörige Weylfolge. Dann gilt
\begin{align*}
\norm{(A+B-\lambda)\ph_n} \le \underbrace{\norm{(A-\lambda)\ph_n}}_{\to 0} +
\underbrace{\norm{B\ph_n}}_{\overset{!}{\to}0},
\end{align*}
denn $\norm{B\ph_n} = \norm{B(A+i)^{-1}(A+i)\ph_n}$ und
\begin{align*}
(A+i)\ph_n = (A-\lambda)\ph_n + (\lambda+i)\ph_n \wto 0,
\end{align*}
während $B(A+i)^{-1}$ kompakt ist. Somit ist $\lambda\in
\sigma_\mathrm{ess}(A+B)$.

``$\subset$'': Die Umkehrung folgt mit den gleichen Argumenten, denn
\begin{align*}
B(A+B+i)^{-1} = B(A+i)^{-1}(A+i)(A+B+i)^{-1}
\end{align*}
ist kompakt, da $(A+i)(A+B+i)^{-1}$ nach dem Graphensatz beschränkt ist.\qedhere
\end{proofenum}
\end{proof}

\begin{thm}
\label{prop:7.4}
Sei $\HH$ ein separabler Hilbertraum und $B\in\LL(\HH)$ kompakt. Dann existiert
eine Folge $(B_n)$ von Operatoren von endlichem Rang mit $\norm{B_n-B}\to
0$.\fishhere
\end{thm}
\begin{proof}
Da $\HH$ separabel ist, existiert eine abzählbare ONB $(\ph_n)$. Sei $P_N$
definiert durch
\begin{align*}
P_N\ph = \sum_{k=1}^N \ph_k \lin{\ph_k,\ph},
\end{align*}
dann gilt $P_N^* = P_N = P_N^2$ und $P_N\ph \to \ph$ für alle $\ph\in\HH$. Sei
$B_N := BP_N$, dann ist $B_N$ von endlichem Rang und nach Definition der Norm
 $\norm{B_N-B}$ gibt es Vektoren $\gamma_N$ mit $\norm{\gamma_N}=1$ und
\begin{align*}
\norm{B_N-B} \le \norm{(B_N-B)\gamma_N} + \frac{1}{N}.\tag{*}
\end{align*}
Es gilt
\begin{align*}
(1-P_N)\gamma_N \wto 0,\qquad N\to \infty, 
\end{align*}
denn für $\psi\in\HH$ gilt
\begin{align*}
\lin{\psi,(1-P_N)\gamma_N} = \lin{(1-P_N)\psi,\gamma_N}
\le \underbrace{\norm{(1-P_N)\psi}}_{\to 0}\underbrace{\norm{\gamma_N}}_{=1}
\end{align*}
Da $B$ kompakt ist, folgt aus (*), dass $\norm{B_N-B}\to 0$.\qedhere
\end{proof}

Kompakte Operatoren über einem separablen Hilbertraum können also als Abschluss
der Operatoren von endlichem Rang aufgefasst werden.

\begin{defn*}
Ein Operator auf $\LL^2(\R^n)$ heißt
\emph{Hilbert-Schmidt-Operator}\index{Operator!Hilbert-Schmidt-}, falls ein Kern
$K\in\LL^2(\R^n\times\R^n)$ existiert mit
\begin{align*}
(B\ph)(x) = \int_{\R^n} K(x,y)\ph(y)\dy.\tag{**}\fishhere
\end{align*}
\end{defn*}

\begin{lem*}
Jeder Kern definiert via (**) einen beschränkten Operator $B$ mit
\begin{align*}
\norm{B}\le \norm{K}_2.\fishhere
\end{align*}
\end{lem*}
\begin{proof}
Sei also $K\in L^2(\R^n\times \R^n)$ und $B$ via (**) definiert, dann ist
\begin{align*}
\int \abs{B\ph(x)}^2\dx &= 
\int \abs{\int K(x,y)\ph(y)\dy}^2\dy\\
&\le
\int \left(\int \abs{K(x,y)}^2\dy \right)
\left(\int \abs{\ph(y)}^2 \dy \right)\dx\\
&= \norm{K}_2^2\norm{\ph}_2^2.\qedhere
\end{align*}
\end{proof}

Das Tensorprodukt $\ph\otimes \psi\in L^2(\R^n\times\R^n)$ von $\ph,\psi\in
L^2(\R^n)$ ist definiert durch
\begin{align*}
(\ph\otimes\psi)(x,y) := \ph(x)\psi(y). 
\end{align*}
Ist $(\ph_n)$ eine ONB von $L^2(\R^n)$, dann ist
$(\ph_n\otimes\ph_k)_{n,k\in\N}$ eine ONB von $L^2(\R^n\times\R^n)$.

\begin{thm}
\label{prop:7.5}
Jeder Hilbert-Schmidt-Operator auf $L^2(\R^n)$ ist kompakt.\fishhere
\end{thm}
\begin{proof}
Sei $B$ ein Hilbert-Schmidt-Operator mit Integralkern $K\in
L^2(\R^n\times\R^n)$, $(\ph_n)$ eine ONB von $L^2(\R^n)$ und
\begin{align*}
K_N := \sum_{l,j=1}^N \ph_l\otimes\ph_j \lin{\ph_l\otimes \ph_j,K}.
\end{align*}
Der durch $K_N$ definierte Hilbert-Schmidt-Operator $B_N$,
\begin{align*}
(B_N\ph)(x) = \int K_N(x,y)\ph(y)\dy 
\end{align*}
ist von endlichem Rang, denn $B_NL^2(\R^n)=\mathrm{span
}\setd{\ph_1,\ldots,\ph_N}$, und folglich kompakt. Weiterhin ist
\begin{align*}
\norm{B-B_N}\le \norm{K-K_N}_2 \to 0
\end{align*}
also ist auch $B$ kompakt.\qedhere
\end{proof}

\begin{prop}
\label{prop:7.6}
Sind $f,g\in L^2(\R^n)$, dann ist
\begin{align*}
f(x)g(-i\nabla_x) = M_f \FF^{-1} M_g \FF
\end{align*}
in $L^2(\R^n)$ dicht definiert und beschränkt. Die eindeutige beschränkte
Fortsetzung auf $L^2(\R^n)$ ist ein Hilbert-Schmidt-Operator mit Kern
\begin{align*}
(2\pi)^{-n/2} f(x)\check{g}(x-y).\fishhere
\end{align*}
\end{prop}
\begin{proof}
Der Operator ist auf $\SS(\R^n)$ definiert, denn für $\ph\in \SS(\R^n)$ gilt
\begin{align*}
\hat{\ph}\in \SS(\R^n) \subset L^2(\R^n)\cap L^\infty(\R^n)
\end{align*}
also ist $g\hat{\ph}\in L^1(\R^n)\cap L^2(\R^n)$. Somit ist $\hat{\ph} \in
D(M_g)$ und
\begin{align*}
\FF^{-1}M_g\FF\ph = \FF^{-1}(g\hat{\ph}) \in L^\infty(\R^n)\subset D(M_f).
\end{align*}
Für $g,\ph\in \SS(\R^n)$ gilt
\begin{align*}
\FF^{-1}(g\hat{\ph}) = (2\pi)^{-n/2}\check{g}*\ph.
\end{align*}
Zu $g\in L^2(\R^n)$, existiert eine Folge $(g_k)$ in $\SS(\R^n)$ mit
$\norm{g_k-g}_2 \to 0$ und somit ist für $\ph\in\SS(\R^n)$
\begin{align*}
\FF^{-1} M_g \FF \ph &= \FF^{-1} g\hat{\ph}
= L^2\text{-}\lim\limits_{n\to\infty} \FF^{-1}(g_k \hat{\ph})
= L^2\text{-}\lim\limits_{n\to\infty} (2\pi)^{-n/2} \check{g}_k * \ph\\
&= (2\pi)^{-n/2} \check{g} * \ph,
\end{align*}
denn $\norm{(\check{g}_k-\check{g})*\ph}_2 \le
\norm{\check{g}_k-\check{g}}_2\norm{\ph}_1$.\qedhere
\end{proof}

\begin{lem}
\label{prop:7.7}
Sei $V:\R^n\to\R$ messbar und sei eine der folgenden Bedingungen erfüllt
\begin{equivenum}
\item\label{prop:7.7:1} $V\in L^2(\R^n)$,\qquad $n\le 3$,
\item\label{prop:7.7:2} $V\in L^2_\mathrm{loc}(\R^n)$,\qquad $n\le 3$,\quad
$V(x)\to 0$,\quad $\abs{x}\to \infty$,
\item\label{prop:7.7:3} $V\in L^\infty(\R^n)$,\qquad $V(x)\to
0$,\quad$\abs{x}\to \infty$.
\end{equivenum} 
Dann ist $V(-\Delta + 1)^{-1}$ kompakt.\fishhere
\end{lem}

\begin{bem*}
Die Kompaktheit von $V(-\Delta+1)^{-1}$ ist äquivalent zur Kompaktheit der
Abbildung
\begin{align*}
V : H^2(\R^n)\to L^2(\R^n),\qquad \ph\mapsto V\ph.\maphere
\end{align*}
\end{bem*}

\begin{proof}
Sei $g(p) = (p^2+1)^{-1}$, dann ist $g\in L^2(\R^n)$ genau dann, wenn $n\le 3$.

In den Fällen \ref{prop:7.7:2} und \ref{prop:7.7:3} genügt es zu zeigen, dass
\begin{align*}
V_R(-\Delta + 1)^{-1},\qquad V_R(x) = V(x)\chi_{\abs{x}\le R}
\end{align*}
kompakt ist für alle $R>0$, denn
\begin{align*}
\norm{V(-\Delta+1)^{-1}-V_R(-\Delta+1)^{-1}}
&\le \norm{M_{V-V_r}}\underbrace{\norm{(-\Delta+1)^{-1}}}_{\le 1}\\
&\le \norm{V-V_R}_\infty = \sup_{\abs{x}>R} \abs{V(x)}\to 0,\qquad R\to \infty. 
\end{align*}
Da $V_R\in L^2(\R^n)$ ist $V_R(-\Delta + 1)^{-1}$ unter den Vorraussetzungen von
\ref{prop:7.7:2} nach \ref{prop:7.7:1} kompakt.

Im Fall \ref{prop:7.7:3} definieren wir $g_m(p) = g(p)\chi_{\abs{x}\le m}$, dann
ist $g_m\in L^2(\R^n)$ und
\begin{align*}
\norm{g_m-g}_\infty = \sup_{\abs{p}>m} \frac{1}{p^2+1} \to 0.
\end{align*}
Also ist nach obigem Satz $V_Rg_m(-i\nabla_x)$ kompakt und
\begin{align*}
\norm{V_R(-\Delta+1)^{-1} - V_Rg_m(-i\nabla_x)} &\le
\norm{V_R}_\infty \norm{(-\Delta+1)^{-1} - g_m(-i\nabla_x)}\\
&\le \norm{V_R}_\infty \norm{g-g_m}_\infty \to 0.
\end{align*}
Somit ist auch $V_R(-\Delta+1)^{-1}$ kompakt.\qedhere
\end{proof}

\begin{thm}
\label{prop:7.8}
Ist $V:\R^n\to\R$ messbar und $V(-\Delta+1)^{-1}$ kompakt, dann ist
\begin{align*}
-\Delta + V : D\subset L^2(\R^n)\to L^2(\R^n)
\end{align*}
selbstadjungiert auf $D=H^2(\R^n)$, nach unten beschränkt und
$\sigma_\mathrm{ess} = [0,\infty)$.\fishhere
\end{thm}

\begin{bem*}
Ein selbstadjungierter Operator $A$ heißt nach unten beschränkt, wenn
$\sigma(A)$ nach unten beschränkt ist.\maphere
\end{bem*}

\begin{proof}
Wegen $\sigma_\mathrm{ess}(-\Delta) = \sigma(-\Delta) = [0,\infty)$ bleibt nach
den bisherigen Sätzen nur zu zeigen, dass $(-\Delta+ V)$ nach unten beschränkt
ist.

Für $\lambda > 0$ gilt
\begin{align*}
(-\Delta + V + \lambda) = (\Id + V(-\Delta + \lambda)^{-1})(-\Delta
+\lambda)\tag{*}
\end{align*}
wobei $\norm{V(-\Delta+\lambda)^{-1}}\to 0$ für $\lambda\to \infty$, da
$\norm{V(-\Delta+1)^{-1}}$ kompakt ist (vgl. Beweis von
$\norm{B(A+in)^{-1}} \to 0$ zu Satz \ref{prop:7.3}). Also existiert ein
$\lambda_0$, so dass für $\lambda > \lambda_0$
\begin{align*}
\norm{V(-\Delta + \lambda)^{-1}} < 1
\end{align*}
und folglich der Operator $-\Delta+V+\lambda$ nach (*), $H^2(\R^n)\to
L^2(\R^n)$ bijektiv abbildet mit beschränkter Inverser. Also ist
$-\lambda\in\rho(-\Delta+V)$ für $\lambda > \lambda_0$.\qedhere
\end{proof}

%TODO: Illustration des Spektrums

\begin{cor}
\label{prop:7.9}
Erfüllt $V$ die Voraussetzungen von Theorem \ref{prop:7.8} und ist
\begin{align*}
E := \inf\setdef{\lin{\ph,(-\Delta+V)\ph}}{\ph\in H^2(\R^n),\; \norm{\ph} =
1}<0,
\end{align*}
dann ist $E=\min\sigma(-\Delta+V)$ ein isolierter Eigenwert endlicher
Vielfachheit.\fishhere
\end{cor}
\begin{proof}
Sei $H=-\Delta + V$ und $\lambda_0 = \min \sigma(H)$. Aus dem Spektralsatz
wissen wir,
\begin{align*}
\lin{\ph,H\ph} = \int_{\sigma(H)} \lambda\dmu_{\ph}(\lambda)
\ge \norm{\ph}^2\lambda_0.
\end{align*}
Wegen $\lambda_0\in \sigma(H)$  ist 
$P_{(\lambda_0-\ep,\lambda_0+\ep)}(H)\neq 0$
für jedes $\ep > 0$. Für $\ph_\ep \in
P_{(\lambda_0-\ep,\lambda_0+\ep)}(H)\HH$ ist dann
\begin{align*}
\lin{\ph_\ep,H\ph_\ep} = \int_{(\lambda_0-\ep,\lambda_0+\ep)} \lambda
\dmu_{\ph_\ep}(\lambda) \le \norm{\ph_\ep}^2\left(\lambda_0 + \ep\right)
\end{align*}
und folglich ist $E\le \lambda_0$.
Dass $E$ ein isolierter Eigenwert endlicher Vielfachheit ist, folgt aus der
Annahme $E< 0$ und aus Theorem \ref{prop:7.8}.\qedhere
\end{proof}
% 
% \subsection{Vielteilchensysteme}
% 
% Der Hamiltonoperator eines Atoms mit $N$ Elektronen und $Z$ Protonen ist gegeben
% durch,
% \begin{align*}
% H_{N,Z} =
% \sum_{k=1}^N \left(-\Delta_{x_k} - \frac{Z}{\abs{x}}\right) + \sum_{i< k}
% \frac{1}{\abs{x_i-x_k}}.
% \end{align*}
% Man definiert nun
% \begin{align*}
% E_{N,Z} = \inf \sigma(H_{N,Z})
% \end{align*}


\printindex

\begin{thebibliography}{50}
\newcommand{\bibauthor}[1]{\textsc{#1}}
\newcommand{\bibtitle}[1]{\textit{#1}}
\newcommand{\bibpublisher}[1]{\text{#1}}
%
\bibitem[CFKS07]{CFKS07}
	\bibauthor{Cycon H. L., Froese R. G., Kirsch W., Simon B.},
	\bibtitle{Schrödinger Operators}.
	\bibpublisher{2. Auflage, Springer 2007}.
%
\bibitem[Fun07]{Fun07}
	\bibauthor{Griesemer M.},
	\bibtitle{Funktionalanalysis}.
	\bibpublisher{Sommersemester 2007}.
%
\bibitem[HS95]{HS95}
	\bibauthor{Hislop P.D., Sigal I.M.},
	\bibtitle{Introduction to Spectral Theory}.
	\bibpublisher{1. Auflage, Springer 1995}.
%
\bibitem[Kat66]{Kat66}
	\bibauthor{Kato T.},
	\bibtitle{Perturbation Theory for Linear Operators}.
	\bibpublisher{2. Auflage, Springer 1966}.
%
\bibitem[RSa]{RS95a}
	\bibauthor{Reed M., Simon B.},
	\bibtitle{Methods of modern mathematical physics, Band 1}.
	\bibpublisher{Academic Press, 1995}.
%
\bibitem[RSb]{RS05b}
	\bibauthor{Reed M., Simon B.},
	\bibtitle{Methods of modern mathematical physics, Band 2}.
	\bibpublisher{Academic Press, 2005}.
%
\bibitem[Tes09]{Tes09}
	\bibauthor{Teschl G.},
	\bibtitle{Mathematical Methods in Quantum Mechanics}.
	\bibpublisher{Volume 99, Amer. Math. Soc., Providence, 2009}.
%
\bibitem[Thi94]{Thi94}
	\bibauthor{Thirring W.},
	\bibtitle{Lehrbuch der Mathematischen Physik, Band 3}.
	\bibpublisher{2. Auflage, Springer 1994}.
%
\bibitem[Wer07]{Wer07}
	\bibauthor{Werner D.},
	\bibtitle{Funktionalanalysis}.
	\bibpublisher{6. Auflage, Springer 2007}.
%
\bibitem[Yos80]{Yos80}
	\bibauthor{Yosida K.},
	\bibtitle{Functional analysis}.
	\bibpublisher{Springer 1980}.
\end{thebibliography}

% G. Teschl: Mathematical Methods in Quantum Mechanics
% Thirring: Lehrbuch der Mathematischen Physik, Band 3.
% Kato: Perturbation Theory of Linear Operators
% Hislop, Sigal: Introduction to Spectral Theory. With Applications to Schrödinger Operators.
% Cycon, Froese, Kirsch, Simon: Schödinger Operators


\end{document}