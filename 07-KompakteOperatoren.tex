\chapter{Kompakte Operatoren und Stabilität des wesentlichen Spektrums}

\begin{defn*}
Eine Folge $(\ph_n)$ im Hilbertraum $\Hc$ konvergiert
\emph{schwach}\index{schwache Konvergenz} gegen $\ph\in \Hc$, in Zeichen
\begin{align*}
\ph_n \wto \ph,\qquad n\to \infty,
\end{align*}
wenn für alle $\psi\in\Hc$ gilt,
\begin{align*}
\lim\limits_{n\to\infty} \lin{\psi,\ph_n} = \lin{\psi,\ph}.\fish
\end{align*}
\end{defn*}

\begin{bem*}[Bemerkungen.]
\begin{bemenum}
\item Jede Orthonormalfolge $(\ph_n)$ konvegiert schwach gegen Null, denn für
alle $\psi\in\Hc$ ist
\begin{align*}
\sum_{n\ge 0} \abs{\lin{\psi,\ph_n}}^2 \le \norm{\psi}^2.
\end{align*}
\item Aus $\ph_n\to \ph$ folgt $\ph_n\wto \ph$. Umgekehrt folgt aus
\begin{align*}
\ph_n\wto \ph,\qquad \norm{\ph_n}\to \norm{\ph}
\end{align*}
auch $\ph_n \to\ph$.\map
\end{bemenum}
\end{bem*}

\begin{thm}[Theorem von Weyl]
\label{prop:7.1}
Sei $A$ selbstadjungiert. Die Zahl $\lambda\in\R$ ist genau dann in
$\sigma_\mathrm{ess}(A)$, wenn eine \emph{Weylfolge} $(\ph_n)$ in $D(A)$
existiert, d.h. eine Folge mit 
\begin{align*}
\norm{\ph_n} =1,\; \ph_n\wto 0,\quad \text{ und}\quad
\norm{(A-\lambda)\ph_n}\to 0.\fish
\end{align*}
\end{thm}

\begin{proof}
``$\Rightarrow$": Sei $\lambda\in\sigma_\mathrm{ess}(A)$ und $\Hc_n =
P_{(\lambda-\frac{1}{n},\lambda+\frac{1}{n})}(A)\Hc$.
Dann ist $\dim\Hc_n = \infty$ für jedes $n\ge 1$ nach Satz \ref{prop:6.11}. Wir
konstruieren rekursiv eine Orthonormalfolge $(\ph_n)$ mit $\ph_n\in\Hc_n$.

Sei $\ph_1\in \Hc_1$ mit $\norm{\ph_1}=1$. Gegeben $\ph_1,\ldots,\ph_{n-1}$
orthonormalisiert mit $\ph_k\in\Hc_k$ für $k=1,\ldots,n-1$, wählen wir
$\ph_n\in\Hc_n$ mit $\ph_n\bot \setd{\ph_1,\ldots,\ph_{n-1}}$ und
$\norm{\ph_n}=1$. Es gilt dann $\ph_n\wto 0$ und
\begin{align*}
\norm{(A-\lambda)\ph_n}^2 = 
\int_{\abs{t-\lambda}<\frac{1}{n}} \abs{t-\lambda}^2 \dmu_{\ph_n}(\lambda)
\le \frac{1}{n^2}\norm{\ph_n}^2 \to 0.
\end{align*}

``$\Leftarrow$'': Sei $(\ph_n)$ eine Weylfolge, dann ist $\lambda\in\sigma(A)$.
Wir führen nun die Annahme $\lambda\in\sigma_\mathrm{disc}(A)$ auf einen
Widerspruch. Angenommen $\lambda\in\sigma_\mathrm{disc}(A)$, dann existiert nach
Satz \ref{prop:6.11} ein $\ep > 0$ , so dass
\begin{align*}
\dim(P_{I_\ep}(A)\Hc) < \infty,\qquad I_\ep = (\lambda-\ep,\lambda+\ep).  
\end{align*}
Weiterhin gilt $P_{I_\ep}(A) \ph_n \to 0$, denn
\begin{align*}
P_{I_\ep}(A)\ph_n = \sum_{k=1}^N \psi_k\lin{\psi_k,\ph_n}\to 0
\end{align*}
wenn $(\psi_k)_{k=1}^N$ eine ONB von $P_{I_\ep}(A)\Hc$ ist. Außerdem gilt
\begin{align*}
\norm{P_{\R\setminus I_\ep}(A)\ph_n}^2 &=
\int_{\R\setminus I_\ep} \dmu_{\ph_n}(t)
<
\int_{\abs{t-\lambda}>\ep} \frac{\abs{t-\lambda}^2}{\ep^2} \dmu_{\ph_n}(t)\\
&\le
\frac{1}{\ep^2} \int \abs{t-\lambda}^2 \dmu_{\ph_n}(t)
= \frac{1}{\ep^2} \norm{(A-\lambda)\ph_n}^2 \to 0,\qquad n\to \infty.
\end{align*}
Also gilt auch
\begin{align*}
1 = \norm{\ph_n}^2 = \norm{P_{I_\ep}(A)\ph_n}^2 + \norm{P_{\R\setminus
I_\ep}(A)\ph_n}^2 \to 0,
\end{align*}
ein Widerspruch.\qed
\end{proof}

\begin{defn*}
Ein beschränkter Operator $B\in\Lc(\Hc)$ heißt
\emph{kompakt}\index{Operator!kompakter}, wenn die Bildfolge $(B\ph_n)$ jeder
beschränkten Folge $(\ph_n)$ eine konvergente Teilfolge besitzt.

$B$ heißt \emph{von endlichem Rang}\index{Operator!von endlichem Rang}, wenn
$\dim (B\Hc) < \infty$.\fish
\end{defn*}

\begin{prop}
\label{prop:7.2}
\begin{propenum}
\item\label{prop:7.2:1} Ist $B\in\Lc(\Hc)$ von endlichem Rang, dann ist $B$
kompakt.
\item\label{prop:7.2:2} Sind $B, C\in\Lc(\Hc)$ kompakt und $\lambda,\mu\in\C$,
dann ist auch $\lambda B+ \mu C$ kompakt.
\item\label{prop:7.2:3} Ist $B$ kompakt und $C\in\Lc(\Hc)$, dann sind auch
$BC$ und $CB$ kompakt.
\item\label{prop:7.2:4} Ist $B_n$ eine Folge kompakter Operatoren in $\Lc(\Hc)$
mit $\norm{B_n-B}\to 0$, dann ist auch $B$ kompakt.
\item\label{prop:7.2:5} Ist $B$ kompakt und $\ph_n\wto \ph$, dann gilt
$B\ph_n\to B\ph$.\fish
\end{propenum}
\end{prop}

Die kompakten Operatoren bilden also ein abgeschlossenes Ideal der beschränkten
Operatoren und führen schwach konvergente in konvergente Folgen über.

\begin{proof}
``\ref{prop:7.2:1}-\ref{prop:7.2:5}": Übung.\qed
\end{proof}

\begin{thm}
\label{prop:7.3}
Sei $A=A^*$ und $B\subset B^*$ mit $D(B)\supset D(A)$. Ist $B(A+i)^{-1}$
kompakt, dann ist $A+B$ selbstadjungiert auf $D(A)$ und
\begin{align*}
\sigma_\mathrm{ess}(A+B) = \sigma_\mathrm{ess}(A).\fish
\end{align*}
\end{thm}

Ist $B(A+i)^{-1}$ kompakt, dann heißt $B$ \emph{relativ kompakt} bezüglich $A$.

\begin{proof}
\begin{proofenum}
\item Wir zeigen $\norm{B(A+in)^{-1}} \to 0$.

Wähle eine Folge $(\ph_n)$ in $\Hc$ mit $\norm{\ph_n} = 1$ und
\begin{align*}
\norm{B(A+in)^{-1}} &\le \norm{B(A+in)^{-1}\ph_n} + \frac{1}{n}\\
&= \norm{B(A+i)^{-1}(A+i)(A+in)^{-1}\ph_n} + \frac{1}{n}.\tag{*}
\end{align*}
Die Existienz der Folge $(\ph_n)$ folgt direkt aus der Definition der Norm
$\norm{B(A+in)^{-1}}$. Aus dem Spektralsatz folgt, dass
\begin{align*}
\norm{(A+i)(A+in)^{-1}} \le 1
\end{align*}
und für alle $\gamma\in D(A)$ gilt
\begin{align*}
&\abs{\lin{\gamma, (A+i)(A+in)^{-1}\ph_n}}
= \abs{\lin{(A-i)\gamma,(A+in)^{-1}\ph_n}}\\
&\quad \le \norm{(A-i)\gamma}\norm{(A+in)^{-1}\ph_n}
\le \frac{1}{n}\norm{(A-i)\gamma}\to 0.
\end{align*}
Da $D(A)\subset\Hc$ dicht liegt, folgt $(A+i)(A+in)^{-1}\ph_n\wto 0$ und damit
gilt nach Voraussetzung
\begin{align*}
B(A+i)^{-1}(A+i)(A+in)^{-1}\ph_n \to 0.
\end{align*}
Mit (*) folgt somit $\norm{B(A+in)^{-1}} \to 0$.
\item \textit{$A+B$ ist selbstadjungiert auf $D(A)$}.

Nach 1) gibt es ein $n\in\N$, so dass
\begin{align*}
a := \norm{B(A+in)^{-1}} < 1.
\end{align*}
Somit gilt für alle $\ph\in D(A)$,
\begin{align*}
\norm{B\ph} = \norm{B(A+in)^{-1}(A+in)\ph} \le a\norm{(A+in)\ph}
\le a \norm{A\ph} + an\norm{\ph}.
\end{align*}
Also ist $A+B$ selbstadjungiert auf $D(A)$ nach dem Theorem von Kato-Rellich 
\ref{prop:4.6}.

\item $\sigma_\mathrm{ess}(A+B) = \sigma_\mathrm{ess}(A)$.

``$\supset$'': Sei $\lambda\in\sigma_\mathrm{ess}(A)$ und $(\ph_n)$ eine
zugehörige Weylfolge. Dann gilt
\begin{align*}
\norm{(A+B-\lambda)\ph_n} \le \underbrace{\norm{(A-\lambda)\ph_n}}_{\to 0} +
\underbrace{\norm{B\ph_n}}_{\overset{!}{\to}0},
\end{align*}
denn $\norm{B\ph_n} = \norm{B(A+i)^{-1}(A+i)\ph_n}$ und
\begin{align*}
(A+i)\ph_n = (A-\lambda)\ph_n + (\lambda+i)\ph_n \wto 0,
\end{align*}
während $B(A+i)^{-1}$ kompakt ist. Somit ist $\lambda\in
\sigma_\mathrm{ess}(A+B)$.

``$\subset$'': Die Umkehrung folgt mit den gleichen Argumenten, denn
\begin{align*}
B(A+B+i)^{-1} = B(A+i)^{-1}(A+i)(A+B+i)^{-1}
\end{align*}
ist kompakt, da $(A+i)(A+B+i)^{-1}$ nach dem Graphensatz beschränkt ist.\qed
\end{proofenum}
\end{proof}

\begin{thm}
\label{prop:7.4}
Sei $\Hc$ ein separabler Hilbertraum und $B\in\Lc(\Hc)$ kompakt. Dann existiert
eine Folge $(B_n)$ von Operatoren von endlichem Rang mit $\norm{B_n-B}\to
0$.\fish
\end{thm}
\begin{proof}
Da $\Hc$ separabel ist, existiert eine abzählbare ONB $(\ph_n)$. Sei $P_N$
definiert durch
\begin{align*}
P_N\ph = \sum_{k=1}^N \ph_k \lin{\ph_k,\ph},
\end{align*}
dann gilt $P_N^* = P_N = P_N^2$ und $P_N\ph \to \ph$ für alle $\ph\in\Hc$. Sei
$B_N := BP_N$, dann ist $B_N$ von endlichem Rang und nach Definition der Norm
 $\norm{B_N-B}$ gibt es Vektoren $\gamma_N$ mit $\norm{\gamma_N}=1$ und
\begin{align*}
\norm{B_N-B} \le \norm{(B_N-B)\gamma_N} + \frac{1}{N}.\tag{*}
\end{align*}
Es gilt
\begin{align*}
(1-P_N)\gamma_N \wto 0,\qquad N\to \infty, 
\end{align*}
denn für $\psi\in\Hc$ gilt
\begin{align*}
\lin{\psi,(1-P_N)\gamma_N} = \lin{(1-P_N)\psi,\gamma_N}
\le \underbrace{\norm{(1-P_N)\psi}}_{\to 0}\underbrace{\norm{\gamma_N}}_{=1}
\end{align*}
Da $B$ kompakt ist, folgt aus (*), dass $\norm{B_N-B}\to 0$.\qed
\end{proof}

Kompakte Operatoren über einem separablen Hilbertraum können also als Abschluss
der Operatoren von endlichem Rang aufgefasst werden.

\begin{defn*}
Ein Operator auf $\Lc^2(\R^n)$ heißt
\emph{Hilbert-Schmidt-Operator}\index{Operator!Hilbert-Schmidt-}, falls ein Kern
$K\in\Lc^2(\R^n\times\R^n)$ existiert mit
\begin{align*}
(B\ph)(x) = \int_{\R^n} K(x,y)\ph(y)\dy.\tag{**}\fish
\end{align*}
\end{defn*}

\begin{lem*}
Jeder Kern definiert via (**) einen beschränkten Operator $B$ mit
\begin{align*}
\norm{B}\le \norm{K}_2.\fish
\end{align*}
\end{lem*}
\begin{proof}
Sei also $K\in L^2(\R^n\times \R^n)$ und $B$ via (**) definiert, dann ist
\begin{align*}
\int \abs{B\ph(x)}^2\dx &= 
\int \abs{\int K(x,y)\ph(y)\dy}^2\dy\\
&\le
\int \left(\int \abs{K(x,y)}^2\dy \right)
\left(\int \abs{\ph(y)}^2 \dy \right)\dx\\
&= \norm{K}_2^2\norm{\ph}_2^2.\qed
\end{align*}
\end{proof}

Das Tensorprodukt $\ph\otimes \psi\in L^2(\R^n\times\R^n)$ von $\ph,\psi\in
L^2(\R^n)$ ist definiert durch
\begin{align*}
(\ph\otimes\psi)(x,y) := \ph(x)\psi(y). 
\end{align*}
Ist $(\ph_n)$ eine ONB von $L^2(\R^n)$, dann ist
$(\ph_n\otimes\ph_k)_{n,k\in\N}$ eine ONB von $L^2(\R^n\times\R^n)$.

\begin{thm}
\label{prop:7.5}
Jeder Hilbert-Schmidt-Operator auf $L^2(\R^n)$ ist kompakt.\fish
\end{thm}
\begin{proof}
Sei $B$ ein Hilbert-Schmidt-Operator mit Integralkern $K\in
L^2(\R^n\times\R^n)$, $(\ph_n)$ eine ONB von $L^2(\R^n)$ und
\begin{align*}
K_N := \sum_{l,j=1}^N \ph_l\otimes\ph_j \lin{\ph_l\otimes \ph_j,K}.
\end{align*}
Der durch $K_N$ definierte Hilbert-Schmidt-Operator $B_N$,
\begin{align*}
(B_N\ph)(x) = \int K_N(x,y)\ph(y)\dy 
\end{align*}
ist von endlichem Rang, denn $B_NL^2(\R^n)=\mathrm{span
}\setd{\ph_1,\ldots,\ph_N}$, und folglich kompakt. Weiterhin ist
\begin{align*}
\norm{B-B_N}\le \norm{K-K_N}_2 \to 0
\end{align*}
also ist auch $B$ kompakt.\qed
\end{proof}

\begin{prop}
\label{prop:7.6}
Sind $f,g\in L^2(\R^n)$, dann ist
\begin{align*}
f(x)g(-i\nabla_x) = M_f \Fc^{-1} M_g \Fc
\end{align*}
in $L^2(\R^n)$ dicht definiert und beschränkt. Die eindeutige beschränkte
Fortsetzung auf $L^2(\R^n)$ ist ein Hilbert-Schmidt-Operator mit Kern
\begin{align*}
(2\pi)^{-n/2} f(x)\check{g}(x-y).\fish
\end{align*}
\end{prop}
\begin{proof}
Der Operator ist auf $\SS(\R^n)$ definiert, denn für $\ph\in \SS(\R^n)$ gilt
\begin{align*}
\hat{\ph}\in \SS(\R^n) \subset L^2(\R^n)\cap L^\infty(\R^n)
\end{align*}
also ist $g\hat{\ph}\in L^1(\R^n)\cap L^2(\R^n)$. Somit ist $\hat{\ph} \in
D(M_g)$ und
\begin{align*}
\Fc^{-1}M_g\Fc\ph = \Fc^{-1}(g\hat{\ph}) \in L^\infty(\R^n)\subset D(M_f).
\end{align*}
Für $g,\ph\in \SS(\R^n)$ gilt
\begin{align*}
\Fc^{-1}(g\hat{\ph}) = (2\pi)^{-n/2}\check{g}*\ph.
\end{align*}
Zu $g\in L^2(\R^n)$, existiert eine Folge $(g_k)$ in $\SS(\R^n)$ mit
$\norm{g_k-g}_2 \to 0$ und somit ist für $\ph\in\SS(\R^n)$
\begin{align*}
\Fc^{-1} M_g \Fc \ph &= \Fc^{-1} g\hat{\ph}
= L^2\text{-}\lim\limits_{n\to\infty} \Fc^{-1}(g_k \hat{\ph})
= L^2\text{-}\lim\limits_{n\to\infty} (2\pi)^{-n/2} \check{g}_k * \ph\\
&= (2\pi)^{-n/2} \check{g} * \ph,
\end{align*}
denn $\norm{(\check{g}_k-\check{g})*\ph}_2 \le
\norm{\check{g}_k-\check{g}}_2\norm{\ph}_1$.\qed
\end{proof}

\begin{lem}
\label{prop:7.7}
Sei $V:\R^n\to\R$ messbar und sei eine der folgenden Bedingungen erfüllt
\begin{equivenum}
\item\label{prop:7.7:1} $V\in L^2(\R^n)$,\qquad $n\le 3$,
\item\label{prop:7.7:2} $V\in L^2_\mathrm{loc}(\R^n)$,\qquad $n\le 3$,\quad
$V(x)\to 0$,\quad $\abs{x}\to \infty$,
\item\label{prop:7.7:3} $V\in L^\infty(\R^n)$,\qquad $V(x)\to
0$,\quad$\abs{x}\to \infty$.
\end{equivenum} 
Dann ist $V(-\Delta + 1)^{-1}$ kompakt.\fish
\end{lem}

\begin{bem*}
Die Kompaktheit von $V(-\Delta+1)^{-1}$ ist äquivalent zur Kompaktheit der
Abbildung
\begin{align*}
V : H^2(\R^n)\to L^2(\R^n),\qquad \ph\mapsto V\ph.\map
\end{align*}
\end{bem*}

\begin{proof}
Sei $g(p) = (p^2+1)^{-1}$, dann ist $g\in L^2(\R^n)$ genau dann, wenn $n\le 3$.

In den Fällen \ref{prop:7.7:2} und \ref{prop:7.7:3} genügt es zu zeigen, dass
\begin{align*}
V_R(-\Delta + 1)^{-1},\qquad V_R(x) = V(x)\chi_{\abs{x}\le R}
\end{align*}
kompakt ist für alle $R>0$, denn
\begin{align*}
\norm{V(-\Delta+1)^{-1}-V_R(-\Delta+1)^{-1}}
&\le \norm{M_{V-V_r}}\underbrace{\norm{(-\Delta+1)^{-1}}}_{\le 1}\\
&\le \norm{V-V_R}_\infty = \sup_{\abs{x}>R} \abs{V(x)}\to 0,\qquad R\to \infty. 
\end{align*}
Da $V_R\in L^2(\R^n)$ ist $V_R(-\Delta + 1)^{-1}$ unter den Vorraussetzungen von
\ref{prop:7.7:2} nach \ref{prop:7.7:1} kompakt.

Im Fall \ref{prop:7.7:3} definieren wir $g_m(p) = g(p)\chi_{\abs{x}\le m}$, dann
ist $g_m\in L^2(\R^n)$ und
\begin{align*}
\norm{g_m-g}_\infty = \sup_{\abs{p}>m} \frac{1}{p^2+1} \to 0.
\end{align*}
Also ist nach obigem Satz $V_Rg_m(-i\nabla_x)$ kompakt und
\begin{align*}
\norm{V_R(-\Delta+1)^{-1} - V_Rg_m(-i\nabla_x)} &\le
\norm{V_R}_\infty \norm{(-\Delta+1)^{-1} - g_m(-i\nabla_x)}\\
&\le \norm{V_R}_\infty \norm{g-g_m}_\infty \to 0.
\end{align*}
Somit ist auch $V_R(-\Delta+1)^{-1}$ kompakt.\qed
\end{proof}

\begin{thm}
\label{prop:7.8}
Ist $V:\R^n\to\R$ messbar und $V(-\Delta+1)^{-1}$ kompakt, dann ist
\begin{align*}
-\Delta + V : D\subset L^2(\R^n)\to L^2(\R^n)
\end{align*}
selbstadjungiert auf $D=H^2(\R^n)$, nach unten beschränkt und
$\sigma_\mathrm{ess} = [0,\infty)$.\fish
\end{thm}

\begin{bem*}
Ein selbstadjungierter Operator $A$ heißt nach unten beschränkt, wenn
$\sigma(A)$ nach unten beschränkt ist.\map
\end{bem*}

\begin{proof}
Wegen $\sigma_\mathrm{ess}(-\Delta) = \sigma(-\Delta) = [0,\infty)$ bleibt nach
den bisherigen Sätzen nur zu zeigen, dass $(-\Delta+ V)$ nach unten beschränkt
ist.

Für $\lambda > 0$ gilt
\begin{align*}
(-\Delta + V + \lambda) = (\Id + V(-\Delta + \lambda)^{-1})(-\Delta
+\lambda)\tag{*}
\end{align*}
wobei $\norm{V(-\Delta+\lambda)^{-1}}\to 0$ für $\lambda\to \infty$, da
$\norm{V(-\Delta+1)^{-1}}$ kompakt ist (vgl. Beweis von
$\norm{B(A+in)^{-1}} \to 0$ zu Satz \ref{prop:7.3}). Also existiert ein
$\lambda_0$, so dass für $\lambda > \lambda_0$
\begin{align*}
\norm{V(-\Delta + \lambda)^{-1}} < 1
\end{align*}
und folglich der Operator $-\Delta+V+\lambda$ nach (*), $H^2(\R^n)\to
L^2(\R^n)$ bijektiv abbildet mit beschränkter Inverser. Also ist
$-\lambda\in\rho(-\Delta+V)$ für $\lambda > \lambda_0$.\qed
\end{proof}

%TODO: Illustration des Spektrums

\begin{cor}
\label{prop:7.9}
Erfüllt $V$ die Voraussetzungen von Theorem \ref{prop:7.8} und ist
\begin{align*}
E := \inf\setdef{\lin{\ph,(-\Delta+V)\ph}}{\ph\in H^2(\R^n),\; \norm{\ph} =
1}<0,
\end{align*}
dann ist $E=\min\sigma(-\Delta+V)$ ein isolierter Eigenwert endlicher
Vielfachheit.\fish
\end{cor}
\begin{proof}
Sei $H=-\Delta + V$ und $\lambda_0 = \min \sigma(H)$. Aus dem Spektralsatz
wissen wir,
\begin{align*}
\lin{\ph,H\ph} = \int_{\sigma(H)} \lambda\dmu_{\ph}(\lambda)
\ge \norm{\ph}^2\lambda_0.
\end{align*}
Wegen $\lambda_0\in \sigma(H)$  ist 
$P_{(\lambda_0-\ep,\lambda_0+\ep)}(H)\neq 0$
für jedes $\ep > 0$. Für $\ph_\ep \in
P_{(\lambda_0-\ep,\lambda_0+\ep)}(H)\Hc$ ist dann
\begin{align*}
\lin{\ph_\ep,H\ph_\ep} = \int_{(\lambda_0-\ep,\lambda_0+\ep)} \lambda
\dmu_{\ph_\ep}(\lambda) \le \norm{\ph_\ep}^2\left(\lambda_0 + \ep\right)
\end{align*}
und folglich ist $E\le \lambda_0$.
Dass $E$ ein isolierter Eigenwert endlicher Vielfachheit ist, folgt aus der
Annahme $E< 0$ und aus Theorem \ref{prop:7.8}.\qed
\end{proof}
% 
% \subsection{Vielteilchensysteme}
% 
% Der Hamiltonoperator eines Atoms mit $N$ Elektronen und $Z$ Protonen ist gegeben
% durch,
% \begin{align*}
% H_{N,Z} =
% \sum_{k=1}^N \left(-\Delta_{x_k} - \frac{Z}{\abs{x}}\right) + \sum_{i< k}
% \frac{1}{\abs{x_i-x_k}}.
% \end{align*}
% Man definiert nun
% \begin{align*}
% E_{N,Z} = \inf \sigma(H_{N,Z})
% \end{align*}
