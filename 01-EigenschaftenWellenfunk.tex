\chapter{Eigenschaften von Wellenfunktionen}

\begin{lem}[Lemma von Riemann-Lebesgue]
\label{prop:1.1}
\index{Lemma!von Riemann-Lebesgue}
Sei $u\in L^1(\R^n)$. Dann ist $\hat{u}\in L^\infty(\R^n)\cap C(\R^n)$,
$\norm{\hat{u}}_\infty \le (2\pi)^{-n/2}\norm{u}_1$ und
\begin{align*}
\hat{u}(p) \to 0,\qquad \abs{p}\to\infty.\fishhere
\end{align*}
\end{lem}
\begin{proof}
\textit{Beschränktheit}. 
Sei $u\in L^1(\R^n)$. Dann ist
\begin{align*}
&\hat{u}(p) = (2\pi)^{-n/2}\int e^{-ipx}u(x)\dx,\\
\Rightarrow\; &
\abs{\hat{u}(p)} \le (2\pi)^{-n/2}
\int\abs{u(x)}\dx = (2\pi)^{-n/2}\norm{u}_1.
\end{align*}
\textit{Stetigkeit}. Anwendung des Satz von Lebesgue ergibt für $\abs{h}\to 0$
\begin{align*}
\hat{u}(p+h)-\hat{u}(p) = (2\pi)^{-n/2}
\int e^{-ipx}\underbrace{\left(e^{-ihx} - 1\right)}_{\to 0}u(x)\dx \to 0,
\end{align*}
denn $|$Integrand$| \le 2\abs{u}\in L^1(\R^n)$.

Sei $\ep > 0$. Da $\SS(\R^n)$ dicht in $L^1(\R^n)$ liegt, gibt es ein $v\in
\SS(\R^n)$, so dass $\norm{u-v}_1 < \frac{\ep}{2}$. Da
$\hat{v}\in\SS(\R^n)$ gibt es außerdem ein $R>0$ mit
\begin{align*}
\abs{\hat{v}(p)} < \frac{\ep}{2},\qquad \abs{p} \ge R.
\end{align*}
Also gilt für $\abs{p}\ge R$,
\begin{align*}
\abs{\hat{u}(p)} \le \abs{\hat{u}(p)-\hat{v}(p)} + \abs{\hat{v}(p)}
\le (2\pi)^{-n/2}\norm{u-v}_1 + \abs{\hat{v}(p)} < \ep.\qedhere
\end{align*}
\end{proof}

\begin{cor}
\label{prop:1.2}
Sei $u\in L^1(\R^n)$ und $\abs{x}^ku\in L^1(\R^n)$. Dann ist $\hat{u}\in
C^k(\R^n)$ und für $\abs{\alpha}\le k$ gilt
\begin{align*}
\partial^\alpha \hat{u}(p) = \widehat{(-ix)^\alpha u}(p) \to 0,\qquad
\abs{p}\to \infty.\fishhere
\end{align*}
\end{cor}
\begin{proof}
\textit{Induktion über $\abs{\alpha}$}. Der Fall $\abs{\alpha}=0$ ist klar.

Sei $j\in\setd{1,..,n}$ und $e_j = (0,\ldots,1,\ldots,0)$ der $j$-te
Basisvektor. Dann ist
\begin{align*}
\frac{1}{h}\left(\hat{u}(p+he_j) - \hat{u}(p) \right)
= (2\pi)^{-n/2}\int e^{-ipx}\underbrace{\frac{1}{h}\left(e^{-ihx_j}-1
\right)}_{\to -ix_j}u(x)\dx,
\end{align*}
wobei $|$Integrand$| \le 2\abs{u}\abs{x_j}\in L^1(\R^n)$ und daher nach Lemma
\ref{prop:1.1} und dem Satz von Lebesgue
\begin{align*}
\partial_j \hat{u}(p) = \widehat{(-ix_j)u}(p) \to 0,\qquad \abs{p}\to\infty.
\end{align*}
Damit ist der Induktionsschrit von $\abs{\alpha} \le k-1$ nach $\abs{\alpha}
\le k$ beweisen. (Induktionsannahme auf $\partial^{\alpha-e_j} u$
anwenden).\qedhere
\end{proof}

\begin{bem*}[Folgerung.]
Falls $u\in L^2(\R^n)$, $\hat{u}\in L^2(\R^n)$ und $\abs{p}^k\hat{u}\in
L^1(\R^n)$, dann ist $u\in C^k(\R^n)$ und es gilt
\begin{align*}
\partial^\alpha u = \FF^{-1}(ip)^\alpha \FF u.\maphere
\end{align*}
\end{bem*}

\begin{bem*}
Insbesondere gilt diese Identität für alle $u\in\SS(\R^n)$. Die rechte Seite
ist aber auch dann sinnvoll (wohldefiniert), wenn $u\in L^2(\R^n)$ und
$\abs{p}^{\abs{\alpha}}\hat{u}\in L^2(\R^n)$.\maphere
\end{bem*}

Dies motiviert die folgende

\begin{defn*}
Für jedes $s\ge 0$ definieren wir einen Innenproduktraum\index{Sobolevraum}
\begin{align*}
&H^s(\R^n) := \setdef{u\in L^2(\R^n)}{\hat{u}(p)\abs{p}^s \in L^2(\R^n)},\\
&\lin{u,v}_s := \int_{\R^n}
\overline{\hat{u}(p)}\hat{v}(p)(1+p^2)^s\ddp.
\end{align*}
$H^s(\R^m)$ heißt \emph{Sobolevraum}.\fishhere
\end{defn*}

Aus der Plancherel-Gleichung folgt unmittelbar, dass $H^0(\R^n)=L^2(\R^n)$ und
weiterhin gilt
\begin{align*}
s> t\ge 0\Rightarrow H^s(\R^n)\subset H^t(\R^n)\subset L^2(\R^n).
\end{align*}

\begin{prop}
\label{prop:1.3}
Für jedes $s\ge 0$ ist $H^s(\R^n)$ ein Hilbertraum und $C_0^\infty(\R^n)$ ist
dicht in $H^s(\R^n)$.\fishhere
\end{prop}
\begin{proof}
Einen Beweis findet man z.B. in \cite{Fun07}.\qedhere
\end{proof}

\begin{thm}[Theorem (Sobolev-Lemma)]
\label{prop:1.4}
\index{Lemma!Sobolev-}
Sei $u\in H^s(\R^n)$ und $k\in\N_0$ mit $k< s- n/2$. Dann gelten
\begin{propenum}
\item\label{prop:1.4:1} $u\in C^k(\R^n)$ und $\partial^\alpha u(x) =
\FF^{-1}((ip)^\alpha \hat{u})(x)\to 0$ für $\abs{x}\to\infty$, $\abs{\alpha}\le
k$.
\item\label{prop:1.4:2} $\norm{\partial^\alpha u}_\infty =
\sup\limits_{x\in\R^n} \abs{\partial^\alpha u(x)} \le C_{s,k,n}\norm{u}_s$ für
$\abs{\alpha}\le k$.\fishhere
\end{propenum}
\end{thm}
\begin{proof}
Sei $u\in H^s$ und $s>n/2+k$.
``\ref{prop:1.4:1}'': Wir wenden die Folgerung aus Korollar \ref{prop:1.2} auf
$u$ an.
\begin{align*}
&\int_{\R^n} \abs{\hat{u}(p)}\ddp,
\int_{\R^n} \abs{p}^k\abs{\hat{u}(p)}\ddp
\le
\int_{\R^n} (1+\abs{p}^2)^{k/2}\abs{\hat{u}(p)}\ddp\\
&\quad=
\int_{\R^n} \abs{\hat{u}(p)}(1+\abs{p}^2)^{s/2}(1+\abs{p}^2)^{k/2-s/2}\ddp\\
&\quad\overset{\text{CSB}}{\le}
\norm{u}_s \left(\int_{\R^n}
(1+\abs{p}^2)^{k-s}\ddp\right)^{1/2}.
\end{align*}
Das Integral ist endlich für $s > n/2+k$, denn
\begin{align*}
\int_{\R^n}
(1+\abs{p}^2)^{k-s}\ddp
= \abs{\S^{n-1}}
\int_{0}^\infty
\frac{1}{(1+t^2)^{s-k}}t^{n-1}\dt
\end{align*}
wobei $2(s-k) - (n-1) > 1 \Leftrightarrow 2(s-k) > n\Leftrightarrow s-k > n/2$.
Mit Korollar \ref{prop:1.2} folgt nun \ref{prop:1.4:1} und außerdem:

``\ref{prop:1.4:2}'': Da
$\partial^\alpha u(x) = (2\pi)^{-n/2} \int_{\R^n} e^{ipx}(ip)^\alpha
\hat{u}(p)\ddp$ ist folglich
\begin{align*}
\abs{\partial^\alpha u(x)} 
&\le (2\pi)^{-n/2} \int_{\R^n} \abs{p^\alpha}
\abs{\hat{u}(p)}\ddp
\le C_{s,p,k}\norm{u}_s.\qedhere
\end{align*}
\end{proof}

\begin{bem*}[Folgerungen.]
\begin{bemenum}
\item $\bigcap_{s\ge 0} H^s(\R^n)\subset C^\infty(\R^n)$.
\item Der maximale Definitionsbereich des Laplace-Operators im $\R^3$ ist
$H^2(\R^3)$. Für $\ph\in H^2(\R^3)$ gilt folglich $\ph\in C(\R^n)$ und
$\ph(x)\to 0$ für $\abs{x}\to\infty$.

Ist $\ph\in H^2(\R)$, so gilt sogar $\ph\in C^1(\R)$ und $\ph,\ph'\to 0$ für
$\abs{x}\to\infty$.

Ist $\ph\in H^1(\R)$, d.h. $\ph\in L^2(\R)$ und die ``kinetische Energie'' ist
endlich
\begin{align*}
\int_{\R^n} \abs{p}^2\abs{\hat{\ph}(p)}^2\ddp < \infty,
\end{align*}
so ist $\ph\in C(\R)$ und $\ph(x)\to 0$ für $\abs{x}\to\infty$.
\item Sei $V\in L^2(\R^n)$, $\ph\in H^2(\R^n)$ und $n\le 3$. Nach dem
Sobolev-Lemma ist dann $\ph\in L^\infty(\R^n)$ und somit
\begin{align*}
\int_{\R^n} \abs{V\ph(x)}^2\dx \le
\underbrace{\sup_{x\in\R^n} \abs{\ph(x)}^2}_{<\infty}
\underbrace{\int\abs{V(x)}^2\dx}_{<\infty} <
\infty,
\end{align*}
d.h. $V\ph\in L^2(\R^n)$.

Dies gilt auch für $V\in L^2(\R^n) \oplus L^\infty(\R^n)$, d.h. $V =
V_2+V_\infty$, denn
\begin{align*}
V\ph = \underbrace{V_2\ph}_{\in L^2} + \underbrace{V_\infty\ph}_{\in L^2}.
\end{align*}
Das Coulomb-Potential $V(x) = -\frac{1}{\abs{x}}$ ist genau von dieser Form:
\begin{align*}
-\frac{1}{\abs{x}} = -\underbrace{\frac{1}{\abs{x}}\chi_{\abs{x}\le 1}}_{\in
L^2} - \underbrace{\frac{1}{\abs{x}}\chi_{\abs{x}>1}}_{\in L^\infty} \in
L^2(\R^3)\oplus L^\infty(\R^3).
\end{align*}
Somit ist $-\frac{1}{\abs{x}}\ph\in L^2(\R^3)$  für alle $\ph\in H^2(\R^3)$.
\maphere
\end{bemenum}
\end{bem*}

Zur Behandlung des Wasserstoffatoms benötigen wir jedoch noch weitere
Ergebnisse.

\begin{lem}
\label{prop:1.5}
Sei $V\in L^2(\R^n)$ und $\ep > 0$. Dann gibt es eine Zerlegung des Potentials
\begin{align*}
V = V_2 + V_\infty,\qquad V_2\in L^2(\R^n),\quad V_\infty\in L^\infty(\R^n)
\end{align*}
mit $\norm{V_2}_2 < \ep$.\fishhere
\end{lem}
\begin{proof}
Sei $\chi_N$ die charakteristische Funktion der Menge
\begin{align*}
\setdef{x\in\R^n}{\abs{V(x)}\le N},
\end{align*}
dann gilt $V = V(1-\chi_N) + V\chi_N$, wobei $V\chi_N \in L^\infty(\R^n)$ und
\begin{align*}
\norm{V(1-\chi_N)}^2 = \int_{\R^n}\abs{V(x)}^2\underbrace{(1-\chi_N(x))}_{\to
0}\dx \to 0,\quad N\to
\infty
\end{align*}
für fast alle $x$ nach dem Satz von Lebesgue, denn
\begin{align*}
\abs{V(x)}^2(1-\chi_N) \le \abs{V(x)}^2 \in L^1(\R^n).
\end{align*}
Wähle nun $N$ so groß, dass $\norm{V(1-\chi_N)} < \ep$ und setzte $V_2 :=
V(1-\chi_N)$ und $V_\infty := V\chi_N$.\qedhere
\end{proof}

\begin{thm}
\label{prop:1.6}
Sei $V\in L^2(\R^n)$ mit $n\le 3$. Für jedes $\ep > 0$ gibt es ein
$C_\ep\in\R$, so dass
\begin{align*}
\norm{V\ph} \le \ep \norm{\Delta \ph} + C_\ep\norm{\ph},\qquad
\forall \ph\in H^2(\R^n).\fishhere
\end{align*}
\end{thm}
\begin{proof}
Sei $V\in L^2(\R^n)$ und $\ep > 0$ und
\begin{align*}
V = V_2 + V_\infty 
\end{align*}
die Zerlegung nach Lemma \ref{prop:1.5}. Dann gilt für $\ph\in H^2(\R^n)$
\begin{align*}
\norm{V\ph} &= \norm{V_2\ph + V_\infty \ph} \le
\norm{V_2\ph} + \norm{V_\infty\ph}
\le
\norm{V_2}\norm{\ph}_\infty + \norm{V_\infty}_\infty\norm{\ph}\\
&< \ep \norm{\ph}_\infty + \norm{V_\infty}_\infty \norm{\ph}.
\end{align*}
Wobei eine leichte Anwendung des Sobolev-Lemmas zeigt
\begin{align*}
\norm{\ph}_\infty \le c\norm{\ph}_{H^2}
\le C\left(\norm{\Delta \ph} + \norm{\ph}\right).
\end{align*}
Somit folgt
\begin{align*}
\norm{V\ph} &\le \ep\left(C\norm{\Delta \ph}  + C\norm{\ph}\right)
+ \norm{V_\infty}_\infty\norm{\ph}\\
&= \ep C\norm{\Delta \ph} + (\ep C + \norm{V_\infty}_\infty)\norm{\ph}.\qedhere
\end{align*}
\end{proof}

\begin{defn*}
Für $u\in H^s(\R^n)$ und $\alpha\in\N_0^n$ mit $\abs{\alpha} \le s$ definiert
man nun\index{schwache Ableitung}
\begin{align*}
\partial^\alpha u:= \FF^{-1}((ip)^\alpha \hat{u}).
\end{align*}
$\partial^\alpha u$ heißt \emph{schwache Ableitung} von $u$.\fishhere
\end{defn*}

Wenn $\abs{\alpha} < s-n/2$, stimmt nach Theorem \ref{prop:1.4} die schwache
Ableitung mit der partiellen Ableitung  $\partial^\alpha$ überein.

\begin{lem}
\label{prop:1.7}
Sei $\alpha\in\N_0^n$ mit $\abs{\alpha}\le s$. Für $u\in H^s(\R^n)$ gilt
$\partial^\alpha u \in H^{s-\abs{\alpha}}(\R^n)$ und
\begin{align*}
\norm{\partial^\alpha u}_{s-\abs{\alpha}} \le \norm{u}_s.\fishhere
\end{align*}
\end{lem}
\begin{proof}
Blatt 1.\qedhere
\end{proof}

$\partial^\alpha : H^s(\R^n)\to H^{s-\abs{\alpha}}(\R^n)$ ist somit eine
beschränkte lineare Abbildung.

\begin{lem}
\label{prop:1.8}
Sei $f\in C^m(\R^n)$ beschränkt mit beschränkten partiellen Ableitungen bis zu
Ordnung $m$. Dann ist $f\ph\in H^m(\R^n)$ für alle $\ph\in H^m(\R^n)$,
$\norm{f\ph}_m \le C_f\norm{\ph}_m$ und die partiellen Ableitungen von $f\ph$
lassen sich mit der Leibniz-Regel berechnen
\begin{align*}
\partial^\alpha (f\ph) = \sum_{\beta\le \alpha}\partial^\beta
f\partial^{\alpha-\beta} \ph.\fishhere
\end{align*}
\end{lem}
\begin{proof}
Wir setzten für diesen Beweis
\begin{align*}
\norm{\ph}_m := \sum_{\abs{\alpha}\le m} \norm{\partial^\alpha \ph}_{L^2}.
\end{align*}
Diese Norm ist äquivalent zur $\norm{\cdot}_s$-Norm.

Sei zuerst $\ph\in C_0^\infty(\R^n)$. Dann ist $f\ph \in C_0^m(\R^n)\subset
H^m(\R^n)$ und
\begin{align*}
\partial^\alpha(f\ph) = \sum_{\beta\le \alpha} \binom{\alpha}{\beta}
\partial^\beta f\partial^{\alpha-\beta} \ph.
\end{align*}
Daraus folgt
\begin{align*}
\norm{\partial^\alpha (f\ph)}_{L^2} \le
\sum_{\beta\le \alpha} \binom{\alpha}{\beta}
\underbrace{\norm{\partial^\beta
f}_{L^2}}_{<\infty}\underbrace{\norm{\partial^{\alpha-\beta}\ph}_{L^2}}_{\le
\norm{\ph}_m} \le C_{\alpha,f}\norm{\ph}_m
\end{align*}
und somit
\begin{align*}
\norm{f\ph}_m = \sum_{\abs{\alpha}\le m}\norm{\partial^\alpha (f\ph)}_{L^2} \le
\sum_{\abs{\alpha}\le m} C_{\alpha,f}\norm{\ph}_m \overset{*}{\le} C_f
\norm{\ph_m}.
\end{align*}

Da $C_0^\infty(\R^n)$ dicht in $H^m(\R^n)$, existiert zu jedem $\ph\in
H^m(\R^n)$ eine Folge $(\ph_n)\in C_0^\infty(\R^n)$ mit $\norm{\ph_n-\ph}_m \to
0$ für $n\to\infty$. Mit (*) folgt nun
\begin{align*}
\norm{f(\ph_k-\ph_l)} \le C_f\norm{\ph_k-\ph_l}_m \to 0,\qquad k,l\to\infty,
\end{align*}
also ist $(f\ph_k)$ eine Cauchyfolge in $H^m(\R^n)$. Da $H^m(\R^n)$
vollständig, existiert ein $\psi\in H^m(\R^n)$ mit
\begin{align*}
\norm{f\ph_k - \psi}_m \to 0.
\end{align*}
Andererseits gilt $\ph_k \to \ph$ in $L^2(\R^n)$ und $f\in L^\infty(\R^n)$ also
\begin{align*}
f\ph_k \to f\ph,\qquad \text{ in } L^2(\R^n).
\end{align*}
Folglich ist $\psi = f\ph$ in $L^2(\R^n)$ und damit insbesondere in
$H^m(\R^n)$, d.h.
\begin{align*}
f\ph_k \to f\ph,\qquad \text{in } H^m(\R^n).
\end{align*}
Somit lässt sich (*) auf $H^m(\R^n)$ übetragen.

Die Produktregel folgt nun aus
\begin{align*}
\partial^\alpha (f\cdot \ph_n) = \sum_{\beta\le \alpha} \partial^\beta f
\partial^{\alpha-\beta}\ph_n
\end{align*}
im Limes für $n\to\infty$, wegen der Stetigkeit der Multiplikation mit $f$ und
der Differentiation (Lemma \ref{prop:1.7}).\qedhere
\end{proof}

\begin{thm}
\label{prop:1.9}
Sei $V:\R^n\to\C$ messbar, $E\in\C$ und $\ph\in H^2(\R^n)$ mit
\begin{align*}
-\Delta \ph + V\ph = E \ph.
\end{align*}
Ist $\Omega$ offen und $V\big|_\Omega \in C^m(\Omega)$ dann ist
\begin{align*}
\ph\big|_\Omega \in C^k(\Omega),\qquad \text{für }k< m+2-n/2.\fishhere
\end{align*}
\end{thm}
\begin{proof}
Zu jedem Punkt $x\in\Omega$ existiert ein $\gamma\in C_0^\infty(\R^n)$ mit
$\supp(\gamma)\subset\Omega$ und $\gamma = 1$ in einer Umgebung von $x$.

Es genügt also zu zeigen, dass $\gamma\ph\in C^k(\R^n)$ für alle $\gamma\in
C_0^\infty(\Omega)$. Nach dem Sobolev-Lemma genügt es dazu, zu zeigen, dass
\begin{align*}
\gamma\ph\in H^{m+2}(\R^n),\qquad \gamma\in C_0^\infty(\Omega).
\end{align*}
\textit{Induktion über $m$}. ``$m=0$'': Wegen $\ph\in H^2(\R^n)$ ist
$\gamma\ph\in H^2(\R^n)$ für alle $\gamma\in C_0^\infty(\Omega)$.

Wir nehmen an, dass
$\gamma\ph\in H^{m+1}(\R^n)$ für alle $\gamma\in C_0^\infty(\Omega)$ und zeigen,
dass $\gamma\ph\in H^{m+2}(\R^n)$ für alle $\gamma\in C_0^\infty(\Omega)$. Es
gilt
\begin{align*}
\Delta(\gamma\ph) &= (\Delta \gamma)\ph + 2\nabla\gamma\nabla \ph +
\gamma\Delta\ph\\
&= (\Delta\gamma)\ph + 2\div((\nabla\gamma)\ph) - 2(\Delta\gamma)\gamma +
\gamma\Delta\ph\\
&= -(\Delta\gamma)\ph + 2\div((\nabla\gamma)\ph) + (V-E)\gamma\ph
\end{align*}
Nach Induktionsannahme gilt
\begin{align*}
(\Delta\gamma)\ph,\;(\nabla\gamma)\ph,\;\gamma\ph\in H^{m+1}(\R^n).
\end{align*}
Nach Annahme über $V$ und nach Satz \ref{prop:1.8} ist $(V-E)\gamma\ph\in
H^m(\R^n)$ und auch $\div((\nabla\gamma)\ph)\in H^m(\R^n)$. Also auch
$\Delta(\gamma\ph)\in H^m(\R^n)$. Daraus folgt, dass $\gamma\ph\in
H^{m+2}(\R^n)$, denn
\begin{align*}
&\int_{\R^n} \abs{\widehat{\gamma\ph}(p)}^2 (1+p^2)^{m+2}\ddp
= \int_{\R^n}
\abs{\widehat{\gamma\ph}(p)}^2(1+p^2){m}\underbrace{(1+p^2)^2}_{\le 2 +
2(p^2)^2}\ddp\\
&\qquad\le
2\norm{\gamma\ph}_m^2 + 2\int_{\R^n}
\abs{p^2\widehat{\gamma\ph}(p)}^2(1+p^2)^m\ddp\\
&\qquad < \infty.\qedhere
\end{align*}
\end{proof}
\begin{cor}
\label{prop:1.10}
Sei $V:\R\to\C$ messbar, $E\in \C$ und $\ph\in H^2(\R^n)$ mit
\begin{align*}
-\ph'' + V\ph= E\ph. 
\end{align*}
Ist $I\subset\R$ ein offenes Intervall und $V\big|_I\in C^m(I)$, dann ist
$\ph\big|_I \in C^{m+2}(I)$.\fishhere
\end{cor}
\begin{proof}
Nach Theorem \ref{prop:1.9} ist $\ph\big|_I\in C^{m+1}(I)$. Wegen
\begin{align*}
\ph'' = (V-E)\ph \in C^m(I)
\end{align*}
ist $\ph\in C^{m+1}(I)$. Details für $m=0$ siehe Übungsaufgabe 2.4.\qedhere
\end{proof}

