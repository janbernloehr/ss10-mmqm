\chapter{Spektrum und Resolvente}

Wir haben bisher Lösungen der Eigenwertgleichung des Operators
\begin{align*}
H = -\Delta + V : H^2(\R^n)\to H^2(\R^n)
\end{align*}
untersucht aber nicht geklärt, ob diese Lösungen überhaupt existieren. Um dies
zu klären, wollen wir die Spektraltheorie für unbeschränkte Operatoren auf
Hilberträumen entwickeln. Dazu betrachten wir allgemein Operatoren
\begin{align*}
A: D\subset X \to X
\end{align*}
und interessieren uns für deren Spektrum, dessen Approximation und Verhalten
unter Störungen.

\subsection{Grundlegende Definitionen}

%%-------------------------------Einfuehrung-Notationen-Definitionen-----------------
Sei $X$ ein komplexer Banachraum und $D \subset X$ ein linearer
Teilraum. 

\begin{defn*}
\begin{defnenum}
\item
Ein \emph{linearer Operator} in $X$,\index{Operator!linear}
\begin{align*}
A : D \subset X \to X,
\end{align*}
ist eine lineare Abbildung $A : D \to X$. $D$ heißt
\emph{Definitionsbereich} von $A$ (schreibe $D = D_A = D(A)$)
und
\begin{align*}
\ran (A) := AD =\setdef{Ax}{x \in D}
\end{align*}
heißt \emph{Wertebereich} von $A$ (range of $A$).
\item
Man sagt, $A$ sei \emph{dicht definiert}, falls\index{dicht definiert}
$\bar{D}=X$.
\item
$A$ heißt \emph{abgeschlossen}, falls der Graph von
$A$\index{Operator!abgeschlossen}
\begin{align*}
\Gamma_A:=\setdef{(x,y)}{x \in D, y=Ax}
\end{align*}
abgeschlossen ist in $X \oplus X$.\fishhere
\end{defnenum}
\end{defn*}

\begin{lem*}
Der Operator $A$ ist
genau dann abgeschlossen, wenn
\begin{align*}
\begin{rcases}
x_n \in D,\, x_n \to x \in X\\
Ax_n\to y
\end{rcases}
\Rightarrow x \in D(A)\text{ und } Ax=y.\fishhere
\end{align*}
\end{lem*}

\begin{defn*}
Ein zweiter Operator, $B : D(B) \subset X \to X$ heißt
\emph{Erweiterung}\index{Operator!Erweiterung}
von $A$, falls $D(A) \subset D(B)$ und $Ax=Bx$ für alle $x \in D(A)$.
Man schreibt dafür $A\subset B$.
Zwei Operatoren sind gleich, $A=B$, falls $D(A)=D(B)$ und
$Ax = Bx$ für alle $x \in D(A)$.\fishhere
\end{defn*}

Den Identitätsoperator $\Id : X \to X,\, x \mapsto x$ lassen
wir meist weg.
Zum Beispiel:
\begin{align*}
\lambda-A := \lambda\Id - A \text{ für } \lambda \in \C.
\end{align*}

\begin{defn*}
\begin{defnenum}
\item
Die \emph{Resolventenmenge} von $A$ ist die Menge\index{Resolventenmenge}
\begin{align*}
\rho(A) := \setdef{\lambda \in \C}{\atop{(\lambda-A) : D(A) \to X \text{
ist bijektiv und }}{(\lambda-A)^{-1} \text{ beschränkt}}}.
\end{align*}
\item
Die Abbildung
\begin{align*}
R(A): \rho(A) \to \LL(X),\qquad
\lambda  \mapsto  (\lambda-A)^{-1}
\end{align*}
heißt \emph{Resolvente}\index{Resolvente} von $A$. $R_{\lambda}(A)$ ist die
Resolvente von $A$ an der Stelle $\lambda$.
\item
Das Komplement der Resolventenmenge:
\begin{align*}
  \sigma(A) := \C \setminus \rho(A)
\end{align*}
heißt \emph{Spektrum}\index{Spektrum} von $A$. Die Menge
\begin{align*}
\sigma_p(A) := \setdef{\lambda \in \C}{\lambda \text{ ist
Eigenwert von A}}\subset \sigma(A)
\end{align*}
heißt \emph{Punktspektrum} von $A$.\index{Spektrum!Punkt-}\fishhere  
\end{defnenum}
\end{defn*}

\begin{bem*}
Wenn $A$ abgeschlossen und $(A-\lambda):D \to X$ bijektiv ist,
dann ist auch $(A-\lambda)^{-1} : X \to D$ abgeschlossen und somit nach
dem Graphensatz beschränkt (FA~Satz~4.3.5 \cite{Fun07}).
Das heißt, wenn $A$ abgeschlossen ist, dann gilt
\begin{align*}
\rho(A)=\setdef{\lambda \in \C}{(\lambda-A) : D(A) \to X \text{ ist
bijektiv}}.
\end{align*}
Wenn $A$ nicht abgeschlossen ist, dann gilt $\rho(A) = \varnothing$ und
$\sigma(A)=\C$.\maphere
\end{bem*}
%%--------------------------------------------------------------


%%----------------------------Beispiele-------------------------
\begin{bsp*}
\begin{bspenum}
\item
Sei $X = \C^{n}$ und $A \in \LL(X)$, dann gilt: $\sigma(A) = \sigma_p(A) \ne
\varnothing$ (Eigenwerte der Matrix bezüglich einer Basis von X).
\item 
Sei $X = L^2(\R)$, $D = \setdef{f \in L^2(\R)}{\int \abs{xf(x)}\dx <
\infty}$ und
\begin{align*}
Af(x) := xf(x).
\end{align*}
Dann gilt: $\sigma(A) = \R$, $\sigma_p(A) = \varnothing$ und 
$\rho(A) = \C \setminus \R$.
\item
Sei $X = C(I)$, $I=[0,1]$, $\norm{f} = \sup \limits_{x \in I} \abs{f(x)}$, 
und sei
\begin{align*}
A_k : D_k \subset X \to X,\qquad (A_kf)(x) = \frac{d}{dx}f(x)
\end{align*}
wobei $k \in \setd{1,\ldots,4}$ und
\begin{align*}
  D_1 & =  C^1(I)  \\
  D_2 & =  \setdef{f \in D_1}{f(0)=0}  \\
  D_3 & =  \setdef{f \in D_1}{f(0)=f(1)} \qquad \text{(Periodische
  Randbedingung)}\\
  D_4 & =  \setdef{f \in D_1}{f(0)=0=f(1)}\  \quad \text{(Dirichlet
  Randbedingung)}
\end{align*}
Alle $A_k$ sind abgeschlossen und es gilt:
\begin{align*}
  \sigma(A_1) & =  \sigma_p(A_1)  =  \C,& \rho(A_1) &= \varnothing, \\
  \sigma(A_2) & =  \sigma_p(A_2)  =  \varnothing,&  \rho(A_2) &= \C, \\
  \sigma(A_3) & =  \sigma_p(A_3)  =  2 \pi i \Z,  \\
  \sigma(A_4) & =  \C,\ \sigma_p(A_4) = \varnothing,&\qquad
  \rho(A_4) &= \varnothing.
\end{align*}
\begin{proof}
\textit{$A$ ist abgeschlossen}.
Sei $(\ph_n)_{n \in \N} \subset D_1$ mit $\ph_n \to \ph$ und $\ph_n' \to \psi$ in $X$.
Dann gilt $\ph_n \to \ph$ gleichm\"a\ss{}ig, $\ph_n' \to \psi'$ gleichm\"a\ss{}ig.
Also ist $\ph \in C^1(I) = D_1$ und $\ph' = \psi$ (Analysis).
Falls $\ph_n \in D_k$, dann ist auch $\ph \in D_k$ f\"ur $k=2,3,4$.
Also ist $A_k$ abgeschlossen f\"ur $k \in \{1,\ldots ,4\}$.

``\textit{$\sigma(A_1) = \sigma_p(A_1) = \C$}''.
Sei $\lambda \in \C$, dann ist $x \mapsto e^{\lambda x}$ in $D_1$ und
\begin{align*}
\frac{\diffd}{\dx}e^{\lambda x} = \lambda e^{\lambda x}.
\end{align*}
Also ist $\lambda$ Eigenwert und daher
 $\sigma(A_1) = \sigma_p(A_1) = \C$.

``\textit{$\sigma(A_2)=\varnothing$}''.
Wir lösen $(A_2-\lambda) f = g$
für gegebenes $g \in X$ nach $f \in D_2$ auf. Das heißt, wir suchen
$f \in C^1(I)$ mit 
\begin{align*}
f'-\lambda f = g, \qquad f(0) = 0.
\end{align*}
Die eindeutige Lösung dieses Anfangswertproblems ist
\begin{align*}
f(x) = \int_{0}^{x} e^{\lambda(x-t)}g(t)\dt =: (Sg)(x).
\end{align*}
Somit ist $(A_2-\lambda):D_2\to X$ bijektiv mit Rechtsinverser $S$.
$S$ ist außerdem beschränkt, denn
\begin{align*}
\abs{(Sg)(x)} \le \int_0^1 \abs{e^{\lambda(x-t)}g(t)}\dt
\le c\int_0^1 \abs{g(t)}\dt \le c\norm{g}.
\end{align*}
Es bleibt noch zu zeigen, dass $(A_2-\lambda)g = g$ für $g \in D_2$. Sei
also $g \in D_2$, dann
\begin{align*}
S(A_2-\lambda)g(x) 
& = \int_{0}^{x} e^{\lambda(x-t)}(g'(t)-\lambda g(t))\dt \\
& = \left. e^{\lambda(x-t)} g(t) \right|_{0}^{x} 
+ \int_{0}^{x} \lambda e^{\lambda(x-t)}g(t)\dt\\ 
& \phantom{ = \left. e^{\lambda(x-t)} g(t) \right|_{0}^{x}\,}
- \int_{0}^{x} e^{\lambda(x-t)} \lambda g(t)\dt\\
& = g(x).
\end{align*}
Also $S = R_{\lambda}(A_2)$.\bsphere
\end{proof}
\end{bspenum}
\end{bsp*}
%%----------------------Beispielende--------------------------


%%----------------------Satz----------------------------------
\begin{prop}
\label{prop:3.1}
Sei $A: D \subset X \to X$ ein linearer Operator und
seien $\lambda,\mu \in \rho(A)$. Dann gilt:
\begin{propenum}
\item\label{prop:3.1:1} $R_{\lambda}(A)-R_{\mu}(A) = (\mu-\lambda)
R_{\lambda}(A)R_{\mu}(A)$,
\item\label{prop:3.1:2} $R_{\lambda}(A) R_{\mu}(A) = R_{\mu}(A) R_{\lambda}(A)$,
\item\label{prop:3.1:3} $R_{\lambda}(A) A \subset A R_{\lambda}(A)$,
\item\label{prop:3.1:4} $A R_{\lambda}(A) = -\Id + R_{\lambda}(A)$.\fishhere
\end{propenum}
\end{prop}

\begin{proof}
\begin{proofenum}
``\ref{prop:3.1:1}'':
Mit $\lambda,\mu \in \rho(A)$ gilt,
\begin{align*}
R_{\lambda}(A)-R_{\mu}(A) & = (\lambda-A)^{-1} - (\mu-A)^{-1}\\
& = (\lambda-A)^{-1} \left[(\mu-A)-(\lambda-A)\right] (\mu-A)^{-1}\\
& = (\lambda-A)^{-1} (\mu-\lambda) (\mu-A)^{-1}\\
& = (\mu-\lambda) R_{\lambda}(A)R_{\mu}(A)
\end{align*}

``\ref{prop:3.1:2}'':
folgt aus \ref{prop:3.1:1}.

``\ref{prop:3.1:3}, \ref{prop:3.1:4}'':
Für $x \in D$ gilt
\begin{align*}
R_{\lambda}(A) Ax & =
(\lambda-A)^{-1}Ax  = (\lambda-A)^{-1}(A-\lambda) + \lambda(\lambda-A)^{-1}\\
&= (-\Id+\lambda R_\lambda(A))x. 
\end{align*}
Außerdem gilt für alle $x \in X$,
\begin{align*}
A R_{\lambda}(A)x & = A(\lambda-A)^{-1}x
= (A-\lambda)(\lambda-A)^{-1}x + \lambda R_\lambda(A)x\\
&= (-\Id + R_\lambda(A))x. 
\end{align*}
Daraus folgen \ref{prop:3.1:3} und \ref{prop:3.1:4}.\qedhere
\end{proofenum}
\end{proof}
%%------------------------------------------------------------

%%-------------------------Theorem------------------------------
\begin{thm}
\label{prop:3.2}
Sei $A : D \subset X \to X$
ein linearer Operator. Dann ist die Resolventenmenge $\rho(A)$ offen,
das Spektrum $\sigma(A)$ abgeschlossen und die Resolvente $z \mapsto
(A-z)^{-1}$ ist analytisch auf $\rho(A)$.

Falls $z_0 \in \rho(A)$, dann ist
\begin{align*}
B(z_0,\norm{R_{z_0}(A)}^{-1}) \subset \rho(A)
\end{align*}
und für alle $z$ aus dieser Kreisscheibe gilt
\begin{align*}
R_z(A) = \sum_{n=0}^{\infty} (-1)^nR_{z_0}(A)^{n+1}(z-z_0)^n.
\end{align*}
Insbesondere gilt
$
\norm{R_{z_0}(A)} \geq \frac{1}{\dist(z_0,\sigma(A))}.\fishhere
$
\end{thm}

\begin{proof}
Sei $z_0 \in \rho(A)$ und $z \in \C$. Dann gilt
\begin{align*}
z- A& = (z_0-A) +  (z-z_0) = (\Id + (z-z_0)R_{z_0}(A))(z_0-A),
\end{align*}
wobei $(z_0-A) : D \to X$ bijektiv mit $(z_0-A)^{-1} \in
\LL(X)$. Falls $\abs{z-z_0} \cdot \norm{R_{z_0}(A)} < 1$, dann ist auch
\begin{align*}
\Id + (z-z_0) R_{z_0}(A) : X \to X\tag{1}
\end{align*}
bijektiv mit stetiger Inversen (FA~Thm~3.3.6 \cite{Fun07}).

Also ist $z\in\rho(A)$, falls $\abs{z-z_0} < \norm{R_{z_0}(A)}^{-1}$. Die
Inverse von (1) ist dann gegeben durch die Neumannreihe
\begin{align*}
\sum_{n\ge 0} (-1)^n R_{z_0}(A)^n(z-z_0)^n
\end{align*}
und folglich
\begin{align*}
\sum_{n\ge 0} (-1)^n R_{z_0}(A)^{n+1}(z-z_0)^n.
\end{align*}

Wenn $z\in\sigma(A)$, dann ist zwingend $\abs{z-z_0}\norm{R_{z_0}(A)} \ge 1$,
d.h. $\abs{z-z_0}\ge \norm{R_{z_0}(A)}^{-1}$ und folglich
\begin{align*}
\dist(z_0,\sigma(A)) = 
\inf\setdef{\abs{z-z_0}}{z\in\sigma(A)} \ge
\frac{1}{\norm{R_{z_0}(A)}}.\qedhere
\end{align*}
\end{proof}

%%---------------------------Korollar------------------------------------------
\begin{cor}
\label{prop:3.4}
Sei $A \in \LL(X)$. Dann ist $\sigma(A)$ kompakt und nichtleer.\fishhere
\end{cor}
\begin{proof}
\textit{$\sigma(A)$ ist kompakt}.
Sei $z\in \C$ mit $\abs{z}\ge \norm{A}$, dann ist $(z-A) = z(\Id-z^{-1}A)$ und
$\norm{z^{-1}A} < 1$. Somit besitzt $(z-A)$ eine beschränkte Inverse
(Neumannreihe)
\begin{align*}
(z-A)^{-1} = \frac{1}{z}\sum_{n=0}^\infty \frac{1}{z^n}A^n
\end{align*}
und daher ist $z\in\rho(A)$. Insbesondere gilt
\begin{align*}
\norm{(z-A)^{-1}} \le \frac{1}{\abs{z}}\sum_{n\ge0}
\frac{1}{\abs{z}^n}\norm{A}^n
= \frac{1}{\abs{z}}\frac{1}{1-\abs{z}^{-1}\norm{A}} = \frac{1}{\abs{z}-\norm{A}}
\end{align*}
und $\sigma(A)\subset\setdef{z\in\C}{\abs{z}\le \norm{A}}$.

\textit{$\sigma(A)$ ist nicht leer}.
Angenommen $\sigma(A)= \varnothing$, dann ist $F_{l,x} : z\mapsto l(R_z(A)x)$
für alle $x\in X$ und $l\in X^*$
eine ganze Funktion und
\begin{align*}
\abs{F_{l,x}(z)} \le \norm{l}\norm{R_z(A)}\norm{x} \le
\norm{l}\norm{x}\frac{1}{\abs{z}-\norm{A}}\to 0,\qquad \abs{z}\to
\infty,\tag{*}
\end{align*}
also auch beschränkt.
Nach dem Satz von Liouville (Funktionentheorie) ist $F_{l,x}$ konsant und wegen
(*) gilt sogar $F_{l,x}\equiv 0$. Somit ist
$l(R_z(A)x)\equiv 0$ für alle $x\in X$ und $l\in X^*$ und daher ist $R_z(A)=0$
für alle $z\in\C$ im Widerspruch zu $(z-A)R_z(A) = \Id$.\qedhere
\end{proof}

Der Beweis lässt sich mit Funktionentheorie über Funktionen $f: \C\to X$
mit $X$ einem komplexen Banachraum deutlich abkürzen. Die Spektraltheorie
befasst sich im Wesentlichen mit solchen Abbildungen.

\begin{bem*}
Für einen unbeschränkten Operator $A:D\subset X\to X$ ist sowohl
$\sigma(A)=\varnothing$ als auch $\sigma(A) = \C$ möglich.\maphere
\end{bem*}

% %%--------------Einf\"uhrung--Notationen---Definitionen--------------------
Im Folgenden wollen wir untersuchen, ob wir einen nichtabgeschlossenen Operator
durch Vergrößerung seines Definitionsbereichs abschließen können.

\begin{defn*}
Ein Operator $A : D \subset X \to X$ heißt
\emph{abschließbar}\index{Operator!abschließbar}, falls ein abgeschlossener
Operator $B \supset A$ existiert.
Die kleinste abgeschlossene Erweiterung eines abschließbaren
Operators $A$ heißt \emph{Abschluss}\index{Operator!Abschluss} von $A$ und wird
mit $\bar{A}$ bezeichnet.
\end{defn*}

Die Existenz von $\bar{A}$ folgt aus folgendem

\begin{prop}
\label{prop:3.5}
Sei $A : D(A) \subset X \to X$ ein linearer Operator.
Dann sind äquivalent:
\begin{equivenum}
\item\label{prop:3.5:1} $A$ ist abschließbar.
\item\label{prop:3.5:2} $\overline{\Gamma}_A$ ist der Graph eines linearen
Operators.
\item\label{prop:3.5:3} Aus $x_n \to 0$ und $Ax_n \to y$ folgt $y = 0$.\fishhere
\end{equivenum}
\end{prop}
\begin{proof}
``\ref{prop:3.5:1}$\Rightarrow$\ref{prop:3.5:3}'': Sei $A\subset B$, $B$
abschließbar und $x_n$ Folge in $D$ mit $x_n\to 0$ und $Ax_n\to y$. Dann gilt
$Bx_n \to y$ und da $B$ abgeschlossen ist $y = \lim\limits_{n\to \infty} Bx_n =
B0 = 0$.

``\ref{prop:3.5:3}$\Rightarrow$\ref{prop:3.5:2}'': Seien $(x,y_1),(x,y_2)\in
\overline{\Gamma_A}$. Dann ist $(0,y_1-y_2) = (x,y_1)-(x,y_2)\in
\overline{\Gamma_A}$. Also existiert eine Folge $(x_n,Ax_n)$ in
$\overline{\Gamma_A}$ mit $x_n\to 0$ und $Ax_n\to y_1-y_2$. Nach
\ref{prop:3.5:3} gilt $y_1=y_2$. Wir können somit einen Operator $B:D(B)\subset
X\to X$ definieren durch
\begin{align*}
x\in D(B) \Leftrightarrow \text{Es existiert }y\in Y\text{ mit }(x,y)\in
\overline{\Gamma_A}.
\end{align*}
und $Bx=y$ wenn $(x,y)\in \overline{\Gamma_A}$. Nach dem eben gezeigten, ist $B$
wohldefiniert, linear, denn $\overline{\Gamma_A}$ ist ein linearer Raum, und
$\Gamma_B = \overline{\Gamma_A}$.

``\ref{prop:3.5:2}$\Rightarrow$\ref{prop:3.5:1}'': Aus
$\Gamma_B=\overline{\Gamma_A}$ folgt $B\supset A$ und $B$ ist abgeschlossen, da
$\Gamma_B$ abgeschlossen. Somit ist $A$ abschließbar.\qedhere
\end{proof}


\begin{bem*}
Nach Satz~\ref{prop:3.5} können wir einen linearen Operator
$\bar{A}$ definieren durch $\Gamma_{\bar{A}} = \overline{\Gamma}_A$.
$\bar{A}$ ist abgeschlossen und f\"ur jeden abgeschlossenen Operator
$B \supset A$ gilt $B \supset \bar{A}$
(folgt aus $A \subset B \Leftrightarrow \Gamma_A \subset \Gamma_B$).
Offenbar gilt:
\begin{align*}
x \in D(\bar{A})
\quad\Leftrightarrow\quad
\begin{cases}
\text{Es existiert eine Folge }(x_n)_{n \in \N} \text{ in } D(A)\\
\text{mit } x_n \to x \text{ und } (Ax_n)_{n \in \N}
\text{ ist eine Cauchyfolge.}
\end{cases}
\end{align*}
Es ist dann $\bar{A}x = \lim \limits_{n \to \infty} Ax_n$.\maphere
\end{bem*}
%%---------------------------------------------------------------------
\begin{bsp*}
Sei $H = L^2(\R)$ und $g\in H$ mit $\norm{g}=1$. Der Operator $A
: D \subset H \to H$ definiert durch
\begin{align*}
  D = C_{0}^{\infty}(\R),\qquad Af = f(0)g,
\end{align*}
ist nicht abschließbar.
\begin{proof}
Sei $f\in D$ mit $f(0)=1$ und sei $f_n(x):=f(nx)$. Dann ist
$\|f_n\|=n^{-1/2}\|f\|\to 0$ f\"ur $n\to\infty$ aber
$Af_n = g$ f\"ur alle $n$, wobei $g\neq 0$.\qedhere\bsphere
\end{proof}
\end{bsp*}