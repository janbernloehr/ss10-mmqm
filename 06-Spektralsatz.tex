\chapter{Funktionalkalkül und Spektralsatz}

\section{Messbarer Funktionalkalkül}

Sei $\BB$ die Familie der Borel-messbaren Teilmengen von $\R$, d.h. die kleinste
$\sigma$-Algebra, die alle offenen und abgeschlossenen Teilmengen enthält.
Weiterhin bezeichne $B(\R)$ die Familie \textit{beschränkten} Borel-messbaren
Funktionen $f: \R\to\C$.

\begin{bem*}[Notation.]
Ist $(f_n)$ eine Folge in $B(\R)$, dann schreiben wir
\begin{align*}
f_n\pto f,\quad \text{oder}\quad f = \plim\limits_{n\to\infty} f_n,
\end{align*}
falls $f_n(x)\to f(x)$ für jedes $x\in \R$ und die $f_n$ gleichmäßig beschränkt
sind, d.h.
\begin{align*}
\sup_{n\ge 0} \norm{f_n} < \infty.\maphere
\end{align*} 
\end{bem*}

\begin{prop}
\label{prop:6.1}
Sei $\FF\subset B(\R)$ eine Familie von Funktionen mit
\begin{propenum}
\item $C_0(\R) \subset \FF$,
\item Ist $(f_n)$ eine Folge in $\FF$ mit $f_n\pto f$, dann ist $f\in\FF$.
\end{propenum}
Dann ist $\FF=B(\R)$.\fishhere
\end{prop}

Man kann dies auch als Definition von $B(\R)$ sehen.

\begin{proof}
Siehe \cite{Wer07}.\qedhere
\end{proof}

\begin{bem*}
Nach Satz \ref{prop:6.1} ist $C_0(\R)$ dicht in $B(\R)$ bezüglich der Toplogie
der punktweisen Konvergenz. Diese Topologie erfüllt jedoch nicht das erste
Abzählbarkeitsaxiom. D.h. obwohl $C_0(\R)$ in $B(\R)$ dicht ist, lässt sich zu
einem $f\in B(\R)$ im Allgemeinen keine Folge $(f_n)$ in $B(\R)$ finden mit
$f_n\pto f$.\maphere
\end{bem*}

\begin{defn*}
Sei $\LL(\HH)$ die Menge der beschränkten linearen Operatoren $B:\HH\to~\HH$.
Eine Folge $(B_n)$ in $\LL(\HH)$ \emph{konvergiert
stark}\index{Konvergenz!starke}
gegen einen Operator $B\in \LL(\HH)$, in Zeichen
\begin{align*}
B = \slim\limits_{n\to\infty} B_n,
\end{align*}
falls $B\ph =\lim\limits_{n\to\infty} B_n\ph$ für jedes $\ph\in\HH$.\fishhere
\end{defn*}

\begin{thm}[Messbarer Funktionalkalkül]
\label{prop:6.2}
Zu jedem selbstadjungierten Operator $A: D\subset\HH\to~\HH$ gibt es eine
eindeutig bestimmte Abbildung
\begin{align*}
\Phi: B(\R)\to\LL(\HH)
\end{align*}
mit folgenden Eigenschaften:
\begin{propenum}
\item\label{prop:6.2:1} Falls $f_z(x) = (z-x)^{-1}$ mit $z\in\C\setminus \R$,
dann ist
\begin{align*}
\Phi(f_z) = (z-A)^{-1}.
\end{align*}
\item\label{prop:6.2:2} $\Phi$ ist ein *-Homomorphismus, d.h. für alle $f,g\in
B(\R)$ gilt
\begin{equivenum}
\item $\Phi(\alpha f + \beta g) = \alpha \Phi(f) + \beta \Phi(g)$ für
$\alpha,\beta\in\C$,
\item $\Phi(f)\Phi(g) = \Phi(fg)$,
\item $\Phi(\bar{f}) = \Phi(f)^*$.
\end{equivenum}
\item\label{prop:6.2:3} Falls $f_k\pto f$, dann gilt
\begin{align*}
\Phi(f) = \slim\limits_{n\to\infty} \Phi(f_k).
\end{align*}
\end{propenum}
% Diese Eigenschaften bestimmen $\Phi$ eindeutig.
Weiterhin gilt für alle $f\in B(\R)$:
\begin{propenum}
\item[4)]\label{prop:6.2:4} $\norm{\Phi(f)} \le \norm{f}_\infty$.
\item[5)]\label{prop:6.2:5} Ist $f\ge 0$, so ist $\Phi(f)\ge 0$ (d.h.
$\lin{\ph,\Phi(f)\ph} \ge 0$ für alle $\ph\in\HH$).
\end{propenum}
Statt $\Phi(f)$ schreibt man in der Regel $f(A)$.\fishhere
\end{thm}

\begin{bsp*}
\begin{bspenum}
\item Sei $A=M_x$ der Multiplikationsoperator
\begin{align*}
(A\ph)(x) = x\ph(x),\qquad D_A = \setdef{\ph\in L^2(\R)}{x\ph(x)\in\L^2(\R)}.
\end{align*}
Dann ist $A=A^*$ und für jede Funktion $f\in B(\R)$ gilt
\[
\Phi(f) = M_f,\qquad \text{d.h. } [\Phi(f)\ph](x) =f(x)\ph(x),\quad \ph\in
L^2(\R).
\]
\begin{proof}
Es genügt zu zeigen, dass $f\mapsto M_f$ die Eigenschaften
\ref{prop:6.2:1}-\ref{prop:6.2:3} aus Theorem \ref{prop:6.2} besitzt. Dann folgt
aus der Eindeutigkeit des Funktionalkalküls, dass $\Phi(f) = M_f$.\qedhere

% Zunächst ist $f\in B(\R)$ beschränkt, d.h. $D(M_f) = L^2(\R)$, denn für jedes
% $\ph\in L^2(\R)$ ist auch $f\ph\in L^2(\R)$.
% 
%  ``\ref{prop:6.2:1}'': Sei $f_z = (z-x)^{-1}$, dann ist für $\ph\in L^2(\R)$
% \begin{align*}
% [(z-M_{x})M_{f_z}\ph](x) &= [zM_{f_z}\ph - M_{x}M_{f_z}\ph](x)\\
% &= zf_z(x)\ph(x) - x [M_{f_z}\ph](x)\\
% &= zf_z(x)\ph(x) - x f_z(x)\ph(x)\\
% &= (z-x)f_z(x)\ph(x) = \ph(x).
% \end{align*}
% Also ist $M_{f_z} = (z-M_{x})^{-1}$.
% 
% ``\ref{prop:6.2:2}'': Seien $f,g\in B(\R)$, dann ist für $\ph\in L^2(\R)$
% \begin{align*}
% [M_{\alpha f + \beta g}\ph](x)
% &= (\alpha f + \beta g)(x)\ph(x)
% = \alpha f(x)\ph(x) + \beta g(x)\ph(x)\\
% &= \alpha [M_{f}\ph](x) +\beta [M_{g}\ph](x)\\
% [M_{fg}\ph](x) &=
% (fg)(x)\ph(x) = f(x)g(x)\ph(x)
% = f(x)[M_g\ph](x)
% \end{align*}
% Letztlich gilt für $\eta,\ph\in L^2(\R)$, dass
% \begin{align*}
% \lin{\ph, M_{f}\eta}
% =\lin{\ph, f\eta}
% = \lin{\bar{f}\ph,\eta}
% = \lin{M_{\bar{f}}\ph,\eta}.
% \end{align*}
% ``\ref{prop:6.2:3}'': Sei $f_k\pto f$ und $\ph\in L^2(\R)$, dann ist
% \begin{align*}
% (M_{f_k}\ph)(x) = f_k(x)\ph(x) \to f(x)\ph(x)  = [M_f\ph](x)
% \end{align*}
% für $x\in\R$. Mit dem Satz von Lebesgue, folgt somit auch
% $\norm{M_{f_k}\ph-M_{f}\ph}_2\to 0$. Somit ist $\Phi(f) = f(M_x) =
% M_{f}$.\qedhere
\end{proof}
\item Sei $A = -i\frac{\diffd}{\dx}$ in $L^2(\R)$ mit $D_A=H^1(\R)$. Dann ist
$A=A^*$ und
\begin{align*}
f(A) = \FF^{-1}f(p)\FF,
\end{align*}
wobei $f(p)$ für den Multiplikationsoperator $M_f$ steht.
\begin{proof}
Analog zu Beispiel 1.\qedhere
\end{proof}
\item Sei $A=-\Delta$ in $L^2(\R)$ und $D_A=H^2(\R)$. Dann ist $A=A^*$ und
\begin{align*}
f(A) = \FF^{-1}f(p^2)\FF.\bsphere
\end{align*}
\end{bspenum}
\end{bsp*}

Für alles Weitere sei $\HH$ ein Hilbertraum und $A: D\subset \HH\to \HH$ ein
linearer Operator auf $\HH$.

\begin{prop}
\label{prop:6.3}
Sei $A=A^*$ in $\HH$ und $f\in C_0(\R)$, dann gilt
\begin{align*}
f(A) = \slim\limits_{\ep\downarrow 0}
\frac{1}{2\pi i}
\int_\R f(t) \left(\frac{1}{t-i\ep -A} - \frac{1}{t+i\ep-A}\right)\dt.\fishhere
\end{align*}
\end{prop}
\begin{proof}
% Man rechnet leicht nach, dass
% \begin{align*}
% \frac{1}{2\pi i}
% \int_\R f(t) \left(\frac{1}{t-i\ep -x} - \frac{1}{t+i\ep-x}\right)\dt
% \pto f(x),\quad \ep \to 0.\tag{*}
% \end{align*}
% Nach Eigenschaft \ref{prop:6.2:3} des messbaren Funktionalkalküls ist daher
% \begin{align*}
% f(A) = \slim\limits_{\ep\to0}
% \Phi\left(\frac{1}{2\pi i}
% \int_\R f(t) \left(\frac{1}{t-i\ep -x} - \frac{1}{t+i\ep-x}\right)\dt
% \right).
% \end{align*}
% Weiterhin ist $t\pm i\ep\notin \C\setminus\R$ für $\ep >
% 0$ und folglich nach Eigenschaft \ref{prop:6.2:1} und \ref{prop:6.2:2},
% \begin{align*}
% &\frac{1}{2\pi i}
% \int_\R f(t) \left(\frac{1}{t-i\ep -A} - \frac{1}{t+i\ep-A}\right)\dt\\
% &= 
% \frac{1}{2\pi i}
% \int_\R f(t) \left(\Phi((t-i\ep-x)^{-1}) - \Phi((t+i\ep-x)^{-1})\right)\dt\\
% &=
% \frac{1}{2\pi i}
% \int_\R \Phi\left(f(t) \left((t-i\ep-x)^{-1} -
% (t+i\ep-x)^{-1}\right)\right)\dt.
% \end{align*}
% Interpretieren wir das Integral als Riemann-Integral, dann konvergiert die
% Riemannsumme punktweise, d.h. wir können $\Phi$ und $\int$ vertauschen,
% \begin{align*}
% \ldots =
% \Phi\left(\frac{1}{2\pi i}
% \int_\R f(t) \left((t-i\ep-x)^{-1} -
% (t+i\ep-x)^{-1}\right)\dt\right)
% \end{align*}
% und die Behauptung folgt.
Übung.\qedhere
\end{proof}

\section{Spektralmaß}

\begin{prop}
\label{prop:6.4}
Sei $A=A^*$ und für jede Borel-Menge $\Omega\subset\R$ sei
\begin{align*}
P_\Omega := \chi_\Omega(A) = \Phi(\chi_\Omega).
\end{align*}
Dann gelten:
\begin{propenum}
\item\label{prop:6.4:1} $P_\Omega$ ist Orthogonalprojektor, d.h.
$P_\Omega^2=P_\Omega$ und $P_\Omega^*=P_\Omega$.
\item\label{prop:6.4:2} $P_\varnothing = 0$ und $P_\R = \Id$.
\item\label{prop:6.4:3} Falls $\Omega = \bigcup_{k=1}^\infty \Omega_k$ und
$\Omega_i\cap \Omega_k = \varnothing$ für $i\neq k$, dann ist
\begin{align*}
P_\Omega = \slim\limits_{N\to\infty} \sum_{k=1}^N P_{\Omega_k}.
\end{align*}
\item\label{prop:6.4:4} $P_\Omega P_{\Omega'} = P_{\Omega \cap
\Omega'}$.\fishhere
\end{propenum}
\end{prop}
\begin{proof}
\begin{proofenum}
\item Man rechnet direkt nach, dass
\begin{align*}
&P_\Omega P_\Omega = \Phi(\chi_\Omega)\Phi(\chi_\Omega)
= \Phi(\chi_\Omega^2) = P_\Omega,\\
&P_\Omega^* = \Phi(\chi_\Omega)^* = \Phi(\bar{\chi_\Omega}) = P_\Omega.
\end{align*}
\item Eine leichte Übung zeigt, $\Phi(1) = \Id$ und folglich ist
\begin{align*}
P_\R = \Phi(\chi_{\R}) = \Phi(1) = \Id,\\
P_\varnothing = \Phi(\chi_\varnothing) = \Phi(0) = 0.
\end{align*}
\item Es gilt
\begin{align*}
\sum_{k=1}^N P_{\Omega_k} 
= \sum_{k=1}^N \Phi(\chi_{\Omega_k})
= \Phi\left(\sum_{k=1}^N \chi_{\Omega_k} \right)
= \Phi\left(\chi_{\bigcup_{k=1}^N\Omega_k} \right),
\end{align*}
wobei 
\begin{align*}
\chi_{\bigcup_{k=1}^N\Omega_k} \pto \chi_{\Omega}
\end{align*}
also
\begin{align*}
P_\Omega =\Phi(\chi_\Omega) = \slim\limits_{N\to\infty}
\Phi\left(\chi_{\bigcup_{k=1}^N \Omega_k}\right)
= \slim\limits_{N\to\infty}\sum_{k=1}^N P_{\Omega_k}
\end{align*}
\item $P_\Omega P_{\Omega'} = \Phi(\chi_{\Omega}\chi_{\Omega'}) =
\Phi(\chi_{\Omega\cap \Omega'}) = P_{\Omega\cap \Omega'}$.\qedhere
\end{proofenum}
\end{proof}

\begin{defn*}
Jede Abbildung $P: \BB\to \LL(\HH)$ mit den Eigenschaften
\ref{prop:6.4:1}-\ref{prop:6.4:3} aus Satz \ref{prop:6.4} heißt
\emph{projektionswertiges Maß}\index{projektionswertiges Maß}.

Eigenschaft \ref{prop:6.4:4} folgt aus \ref{prop:6.4:1}-\ref{prop:6.4:3} und
außerdem gelten
\begin{align*}
&\Omega_1\subset\Omega_2 &&\Rightarrow
P_{\Omega_1}\le P_{\Omega_2}, && \text{Monotonie}\\
&\Omega = \bigcup_{k=1}^N \Omega_k &&\Rightarrow
P_\Omega \le \sum_{k=1}^N P_{\Omega_k}, &&
\text{Subadditivität}.
\end{align*}
% Für jedes $\ph\in\HH$ ist die Abbildung
% \begin{align*}
% \mu: \BB\to\R,\quad \Omega\mapsto \lin{\ph,P_\Omega \ph}
% \end{align*}
% ein Borel-Maß.

Nach dem messbaren Funktionalkalkül gibt es zu jedem selbstadjungierten Operator
$A$ ein projektionswertiges Maß. Dieses heißt auch
\emph{Spektralmaß}.\index{Spektralmaß}\fishhere
\end{defn*}
\begin{proof}
Siehe \cite{Tes09}.\qedhere
% Habe $P:\BB\to\LL(\HH)$ die Eigenschaften \ref{prop:6.4:1}-\ref{prop:6.4:3}.
% Dann gilt für jedes $A\in\BB$,
% \begin{align*}
% \Id = P_{\R} = P_{\R\setminus A \dcup A}
% = P_{\R\setminus A} + P_A.
% \end{align*}
% Insbesondere ist
% \begin{align*}
% P_AP_{\R\setminus A} = P_A(\Id-P_A) = P_A- P_A = 0.
% \end{align*}
% 
% Wir zeigen, dass für jedes $\ph\in\HH$ die Abbildung $\mu$ ein Borel-Maß ist.
% Monotonie und Subadditivität von $P$ folgen dann sofort.
% 
% Für jedes $\Omega\in\BB$ ist 
% \begin{align*}
% \mu(\Omega) &= \lin{\ph,P_\Omega\ph}
% = \lin{P_\Omega\ph,P_\Omega\ph}
% + \lin{P_{\R\setminus\Omega}\ph,P_\Omega\ph}\\
% &= \lin{P_\Omega\ph,P_\Omega\ph}
% + \underbrace{\lin{\ph,P_{\R\setminus\Omega}P_\Omega\ph}}_{=0}
% = \norm{P_\Omega\ph}^2 \ge 0.
% \end{align*}
% Weiterhin ist
% \begin{align*}
% \mu(\varnothing) = \lin{\ph,P_\varnothing\ph} = 0
% \end{align*} 
% und für eine Familie $(\Omega_i)_{i\ge N}$ mit $\Omega_i\cap
% \Omega_k=\varnothing$, falls $i\neq k$ gilt
% \begin{align*}
% \mu\left(\bigcup_{i\ge 1} \Omega_k\right)
% &= \lin{\ph,P_{\bigcup_{i\ge 1} \Omega_k}\ph}
% = \lin{\ph,\slim_{N\to\infty} \bigcup_{i=1}^N \Omega_k \ph}
% = \lim_{N\to\infty}\sum_{k=1}^N \lin{\ph,P_{\Omega_k}\ph}\\
% &= \sum_{k=1}^\infty \lin{\ph,P_{\Omega_k}\ph}.
% \end{align*}
% 
% Zum Nachweis von ``\ref{prop:6.4:4}''
% bemerken wir dass für $A,B\in \BB$ disjunkt gilt $A\subset\R\setminus B$,
% \begin{align*}
% 0\le P_AP_B \le P_{\R\setminus B}P_B = 0,
% \end{align*}
% und folglich $P_AP_B = 0$. Seien 
% $\Omega,\Omega'\in \BB$ und
% weiterhin
% \begin{align*} 
% &(\Omega\cap \Omega')\dcup A = \Omega,\qquad
% (\Omega\cap \Omega')\dcup B = \Omega',
% \end{align*}
% mit $A,B\in\BB$. Dann sind $A$, $B$ und $\Omega\cap \Omega'$ disjunkt und
% \begin{align*}
% P_\Omega P_{\Omega'}
% &= P_{(\Omega\cap \Omega')\dcup A}P_{(\Omega\cap \Omega')\dcup B}
% = P_{(\Omega\cap \Omega')} + 
% P_{(\Omega\cap \Omega')}P_{B}
% + P_AP_{(\Omega\cap \Omega')}
% + P_AP_B\\
% &= P_{(\Omega\cap \Omega')}.\qedhere
% \end{align*}
\end{proof}


\begin{defn*}
Der Träger $\supp(P)$ eines projektionswertigen Maßes $P$ ist definiert durch
\begin{align*}
x\in \supp(P) \Leftrightarrow P(U)\neq 0 \text{ für jede offene Umgebung
}U\text{ von }P.\fishhere
\end{align*} 
\end{defn*}

\begin{lem}
\label{prop:6.5}
Sei $A=A^*$ mit Spektralmaß $P$, dann gelten
\begin{propenum}
\item $S=\supp(P)$ ist abgeschlossen.
\item\label{lem:6.4*:2} $P_S=\Id$ und $P_{R\setminus S} = 0$.
\item Für jede Funktion $f\in B(\R)\cap C(\R)$ gilt
\begin{align*}
\norm{f(A)} = \sup_{x\in S}\abs{f(x)}.\fishhere
\end{align*}
\end{propenum}
\end{lem}

\begin{proof}
\begin{proofenum}
\item Sei $x\notin S$, dann existiert eine offene Umgebung $U$ von $x$ mit $P_U
= 0$. Also ist auch $y\notin S$ für jedes $y\in U$. Somit ist $\R\setminus S$
offen und $S$ ist abgeschlossen.
\item Sei $K\subset\R\setminus S$. Dann existieren offene Mengen
\begin{align*}
\Omega_1,\ldots,\Omega_N \subset \R
\text{ mit }
K \subset \bigcup_{k=1}^N \Omega_k\text{ und }P_{\Omega_k} = 0. 
\end{align*}
Also ist auch
\begin{align*}
0\le P_{K} \le P_{\bigcup_{k=1}^N \Omega_k} \le \sum_{k=1}^N P_{\Omega_k} = 0, 
\end{align*}
und somit $P_K=0$. Sei nun $K_m = \setdef{x\in\R}{\abs{x}\le m\text{ und
}\dist(x,S)\ge \frac{1}{m}}\subset \R\setminus S$. Dann ist $K_m$ kompakt,
$P_{K_m} = 0$ und
\begin{align*}
\bigcup_{m=1}^\infty K_m = \R\setminus S. 
\end{align*}
Daraus folgt
\begin{align*}
P_{\R\setminus S} \le \sum_{m=1}^\infty P_{K_m} = 0,
\end{align*}
also ist $P_{\R\setminus S} =0$. Es folgt
\begin{align*}
P_S = P_S + P_{\R\setminus S} = P_\R = \Id.
\end{align*}
\item Nach \ref{lem:6.4*:2} gilt
\begin{align*}
\norm{\Phi(f)} = \norm{\Phi(\chi_S)\Phi(f)} = \norm{\Phi(\chi_S f)} \le
\sup_{x\in S}\abs{f(x)}.
\end{align*}
Sei $x\in S$. Es genügt nun zu zeigen, dass $\norm{\phi(f)} \ge \abs{f(x)}$.
Dies folgt trivialerweise, wenn $f(x) = 0$. Für $f(x)\neq 0$ gibt es ein $\ep >
0$ so klein, dass
\begin{align*}
\abs{f(x)} - \ep \ge 0.
\end{align*}
Die Abbildung $x\mapsto \abs{f(x)}$ ist stetig, es gibt daher eine offene
Umgebung $U$ von $x$ mit
\begin{align*}
y\in U \Rightarrow \abs{f(y)} > \abs{f(x)} - \ep \ge 0.
\end{align*}
Da $x\in S$, ist $P_U \neq 0$ und somit existiert ein $\ph\in\HH$ mit
$\norm{\ph}=1$ und $P_U\ph = \ph$. Somit gilt
\begin{align*}
\norm{\Phi(f)\ph}^2 &= \lin{\ph,\Phi(\abs{f}^2)\ph}
= \lin{\ph,\Phi(\abs{f}^2)\Phi(\chi_U)\ph}\\
&= \lin{\ph,\Phi(\chi_U \abs{f}^2)\ph}
\ge (\abs{f(x)}-\ep)^2 \lin{\ph,\Phi(\chi_U)\ph}\\
&= (\abs{f(x)}-\ep)^2 \norm{\ph}. 
\end{align*}
Somit ist $\norm{\Phi(f)} \ge \abs{f(x)}-\ep$ für jedes $\ep > 0$, also gilt
auch $\norm{\Phi(f)} \ge \abs{f(x)}$.\qedhere
\end{proofenum}
\end{proof}

\begin{prop}
\label{prop:6.6}
Sei $A=A^*$ mit Spektralmaß $P$. Dann gelten
\begin{propenum}
\item $\sigma(A) = \supp P$.
\item $\norm{(z-A)^{-1}} = \dist(z,\sigma(A))^{-1}$ für alle
$z\in\rho(A)$.\fishhere
\end{propenum}
\end{prop}
\begin{proof}
Sei $S:=\supp P$, so gilt für $z\in\C\setminus\R$ nach Lemma \ref{prop:6.5},
\begin{align*}
\norm{(z-A)^{-1}} = \norm{\Phi(f_z)} = \sup_{x\in S} \norm{(z-x)^{-1}} =
\dist(z,S)^{-1}.\tag{*}
\end{align*}
Für $\lambda\in\R$ sei $z_n = \lambda + in^{-1}$. Dann folgt
\begin{align*}
\lambda\in\sigma(A) \Leftrightarrow \norm{(z_n-A)^{-1}} \to \infty
\Leftrightarrow \dist(z_n,S) \to 0
\Leftrightarrow \lambda\in \bar{S} = S.
\end{align*}
Also ist $S=\sigma(A)$.

Für $\lambda\in\R\setminus \sigma(A)$ folgt nun die zweite Behauptung aus (*) im
Limes für $z\to \lambda$.\qedhere
\end{proof}

\begin{lem}
\label{prop:6.7}
Für jeden linearen Operator $B: \HH\to \HH$ in einem Prähilbertraum $\HH$ und
alle $\ph,\psi\in \HH$ gilt die Polarisationsidentität
\begin{align*}
\lin{\ph, B\psi} &= \frac{1}{4}\left[\lin{\ph+\psi,B(\ph+\psi)} -
\lin{\ph-\psi,B(\ph-\psi)} \right]\\
&+ \frac{1}{4i}\left[\lin{\ph+i\psi,B(\ph+i\psi)} -
\lin{\ph-i\psi,B(\ph-i\psi)}\right].\fishhere
\end{align*}
\end{lem}
\begin{proof}
Der Beweis ist eine leichte Übung.\qedhere
\end{proof}

\begin{prop}
\label{prop:6.8}
Sei $A=A^*$ mit Spektralmaß $P$ und sei $f\in B(\R)$. Dann gilt für alle
$\ph,\psi\in\HH$,
\begin{align*}
\lin{\ph,f(A)\psi} &= \int f(\lambda) \dmu_{\ph,\psi}(\lambda)\\
&:= 
\frac{1}{4}\left[\int f \dmu_{\ph+\psi} - \int f\dmu_{\ph-\psi} \right]
+ \frac{1}{4i}\left[\int f \dmu_{\ph+i\psi} - \int f\dmu_{\ph-i\psi}
\right],
\end{align*}
wobei $\mu_\ph(\Omega) = \lin{\ph,P(\Omega)\ph}$.\fishhere
\end{prop}

\begin{bem*}
$\mu_{\ph,\psi}:= \lin{\ph,P(\Omega)\psi}$ ist ein komplexes Borelmaß. Eine
Einführung in die allgemeine Integrationstheorie bezüglich komplexer Borelmaße
würde den Rahmen dieser Vorlesung sprengen. Aufgrund der Zerlegung des
Integrals, wie in Satz \ref{prop:6.8} angegeben, ist dies jedoch auch nicht
notwendig, da wir uns auf gewöhnliche Integrale zurückziehen können.\maphere
\end{bem*}

\begin{proof}
Ist $f$ eine elementare Funktion
\begin{align*}
f = \sum_{k=1}^n c_k \chi_{\Omega_k},\qquad \Omega_k\in\BB,
\end{align*}
so gilt
\begin{align*}
\lin{\ph,\Phi(f)\ph} &= \lin{\ph,\sum_{k=1}^n c_k
\Phi\left(\chi_{\Omega_k}\right)\ph}
= \sum_{k=1}^n c_k \lin{\ph,P(\Omega_k)\ph}
= \sum_{k=1}^n c_k \mu_{\ph}(\Omega)\\
&= \int f(\lambda) \dmu_\ph(\lambda).
\end{align*}
Zu jeder Funktion $f\in B(\R)$ gibt es eine Folge $(f_n)$ von elementaren
Funktionen $f_n : \R\to \C$ mit $f_n\pto f$. Somit gilt
\begin{align*}
\lin{\ph,\Phi(f)\ph} = \lim\limits_{n\to\infty} \lin{\ph,\Phi(f_n)\ph}
= \lim\limits_{n\to\infty} \int f_k \dmu_\ph
= \int f\dmu_\ph,
\end{align*}
mit dem Satz von Lebesgue und der Tatsache, dass $\mu_\ph$ ein endliches Maß
und damit jede beschränkte Funktion $\mu_\ph$-integrierbar ist.

Um den Beweis abzuschließen, ist noch die Polarisationsidentität auf
$\lin{\ph,f(A)\psi}$ anzuwenden.\qedhere
\end{proof}

\section{Spektralsatz}

Der messbare Funktionalkalkül $\Phi: B(\R)\to \LL(\HH)$ lässt sich auch auf
unbeschränkte borelmessbare Funktionen $f: \R\to\C$ ausdehnen. Der Operator
$\Phi(f)$ ist dann in der Regel unbeschränkt und definiert durch
\begin{align*}
D(\Phi(f)) &= \Phi\left(\frac{1}{1+\abs{f}} \right)\HH,\\
\Phi(f)\ph &= \Phi\left(\frac{f}{1+\abs{f}} \right)\gamma,\qquad\text{falls }
\ph = \Phi\left(\frac{1}{1+\abs{f}} \right)\gamma.
\end{align*}

\begin{prop}
\label{prop:6.9}
Mit obigen Notationen gilt
\begin{propenum}
\item $\Phi(f)$ ist wohl- und dicht definiert.
\item\label{prop:6.9:2} $D(\Phi(f)) = \setdef{\ph\in\HH}{\int \abs{f}^2\dmu_\ph
< \infty}$ und für alle $\ph,\psi\in D(\Phi(f))$ gilt
\begin{align*}
&\norm{\Phi(f)\ph}^2 = \int \abs{f}^2\dmu_\ph\\
&\lin{\ph,\Phi(f)\psi} = \int f \dmu_{\ph,\psi}
\end{align*}
\item Ist $(f_k)$ eine Folge in $B(\R)$ mit $f_k(x)\to f(x)$ für alle $x\in\R$
und $\abs{f_k}\le \abs{f}$ für alle $k$, dann gilt für alle $\ph\in D(\Phi(f))$,
\begin{align*}
\Phi(f_k)\ph \to \Phi(f)\ph,\qquad k\to\infty.\fishhere
\end{align*}
\end{propenum}
\end{prop}

\begin{bem*}
Alle Integrale in \ref{prop:6.9:2} können durch $\int_{\sigma(A)}$ ersetzt
werden.\maphere
\end{bem*}

\begin{proof}
Siehe \cite{RS95a}, \cite{Wer07} oder \cite[Kapitel 12]{Mmq08}.\qedhere
\end{proof}

\begin{thm}[Spektralsatz]
\label{prop:6.10}
Zu jedem selbstadjungierten Operator $A$ gibt es ein eindeutig bestimmtes
Spektralmaß $P$ mit
\begin{align*}
&\lin{\ph, A\psi} = \int_{\sigma(A)} \lambda \dmu_{\ph,\psi}\tag{1},\\
&\mu_\ph(\Omega) := \lin{\ph,P(\Omega)\ph}\tag{2}
\end{align*}
für alle $\ph,\psi\in D(A)$. Umgekehrt wird zu jedem projektionswertigen Maß $P$
durch (1) und (2) ein selbstadjungierter Operator $A$ definiert.\fishhere
\end{thm}
\begin{proof}
Übung.\qedhere
\end{proof}

\begin{bem*}[Bemerkungen.]
\begin{bemenum}
\item Man schreibt (1) oft in der Form
\begin{align*}
A = \int_{\sigma(A)} \lambda \dP(\lambda).
\end{align*}
Das ist die Verallgemeinerung der Spektralzerlegung
\begin{align*}
A = \sum_{k=1}^m \lambda_k P_k
\end{align*}
eines selbstadjungierten Operators $A\in\LL(\C^n)$ mit Eigenwerten
$\lambda_1,\ldots,\lambda_m$ und zugehörigen Eigenprojektoren $P_1,\ldots,P_m$.

\item
\emph{Spektrale Unterräume:}
Sei $\Omega \in \R$ eine Borelmenge,
$P_{\Omega} = \phi(\chi_{\Omega})$ und $\HH_{\Omega} = P_{\Omega}\HH$.
Dann ist $P_{\Omega}A \subset AP_{\Omega}$, und
\begin{align*}
A\big|_{\HH_{\Omega}} : P_{\Omega} D(A) \subset \HH_{\Omega}
\to \HH_{\Omega}
\end{align*}
ist selbstadjungiert (Übung). Ist $\Omega$ offen, dann
\begin{align*}
\sigma(A)\cap\Omega\subset\sigma(A\big|_{\HH_{\Omega}}) \subset
\sigma(A)\cap\bar{\Omega}.
\end{align*}
Insbesondere ist $A \big|_{\HH_{\Omega}}$
beschränkt, wenn $\Omega$ beschränkt ist.
\begin{proof}
Für $\lambda\in \sigma(A)\cap\Omega$ und alle $\ep>0$ klein genug
gilt $(\lambda-\ep,\lambda+\ep)\subset\Omega$ und
$P_{(\lambda-\ep,\lambda+\ep)}\neq 0$. Also
$\lambda\in\sigma(A\big|_{\HH_{\Omega}})$.

Falls $\lambda \notin\sigma(A) \cap\overline{\Omega} := B$
dann ist $\delta := \dist(\lambda, B) > 0$ und somit
für alle $\ph\in P_{\Omega} D(A)$,
\begin{align*}
\norm{(A - \lambda)\ph)}^2
=\int_{\sigma(A) \cap \Omega}\abs{t-\lambda}^2 \, \dmu_{\ph}(t)
\geq \delta^2 \int_{\sigma(A) \cap \Omega} \dmu_{\ph}(t)
= \delta^2 \lin{\ph,\ph}.
\end{align*}
Also $\lambda\notin \sigma(A\big|_{\HH_{\Omega}})$.\qedhere
\end{proof}

\item
$\lambda \in \R$ ist genau dann ein Eigenwert von $A$,
wenn $P_{\setd{\lambda}} \ne 0$ ist.
Dann ist $P_{\setd{\lambda}}$ der Orthogonalprojektor auf den Eigenraum
zu $\lambda$. (siehe Aufgabenblatt 6)

\item In der Quantenmechanik wird jede beobachtbare Größe (Ort, Implus, Spin,
\ldots) durch einen selbstadjungierten Operator beschrieben. Die normierten
Vektoren $\ph\in \HH$ beschreiben die (reinen) Zustände des Systems. Ist
$\Omega\subset\R$ eine Borelmenge, dann ist
\begin{align*}
\mu_\ph(\Omega) = \lin{\ph,P(\Omega)\ph},\qquad P(\Omega) = \Phi(\chi_\Omega), 
\end{align*}
die Wahrscheinlichkeit bei einer Messung der Größe $A$ einen Wert in $\Omega$ zu
finden. Wegen $\mu_\ph(\R\setminus\sigma(A)) = 0$ können nur Werte in
$\sigma(A)$ gefunden werden.

Der Erwartungswert
\begin{align*}
\lin{\ph,A\ph} = \int_{\sigma(A)} \lambda\dmu_\ph(\lambda),\qquad \ph\in D(A),\;
\norm{\ph} = 1
\end{align*}
kann allerdings jeden Wert zwischen $\inf \sigma(A)$ und $\sup \sigma(A)$
annehmen.

Zum Weiterlesen siehe \cite{Stra} oder \cite{Thi94}.\maphere
\end{bemenum}
\end{bem*}

\begin{defn*}
\newcommand{\disc}{{\mathrm{disc}}}
\newcommand{\ess}{{\mathrm{ess}}}
Das \emph{diskrete} und das \emph{wesentliche} Spektrum eines selbstadjungierten
Operators $A$ sind definiert durch
\begin{align*}
&\sigma_\disc = \setdef{\lambda\in\R}{\lambda\text{ ist isolierter Eigenwert
von $A$ mit endlicher Vielfachheit}}\\
&\sigma_\ess = \sigma(A)\setminus \sigma_\disc.\fishhere
\end{align*}
\end{defn*}

\begin{prop}
\label{prop:6.11}
Sei $A=A^*$ und $P_\Omega(A) = \chi_\Omega(A)$. Dann gelten
\begin{propenum}
\item $\lambda\in\sigma(A) \Leftrightarrow P_{(\lambda-\ep,\lambda+\ep)}(A)\neq
0$ für alle $\ep > 0$,
\item $\lambda\in \sigma_\mathrm{disc}(A) \Leftrightarrow \lambda\in \sigma(A)$
und es gibt ein $\ep > 0$ mit $\dim P_{(\lambda-\ep,\lambda+\ep)}(A)\HH <
\infty$.
\item $\lambda\in\sigma_{\mathrm{ess}}(A) \Leftrightarrow \dim
P_{(\lambda-\ep,\lambda + \ep)}(A)\HH = \infty$ für alle $\ep > 0$.\fishhere
\end{propenum}
\end{prop}

\begin{proof}
\newcommand{\disc}{{\mathrm{disc}}}
\newcommand{\ess}{{\mathrm{ess}}}
\begin{proofenum}
\item
Nach Satz \ref{prop:6.5} gilt
\begin{align*}
\lambda \in \sigma(A) &
\Leftrightarrow P_{\Omega}(A) \ne 0 \text{ für jede offene Menge }
\Omega \ni \lambda.\\
& \Leftrightarrow
P_{(\lambda - \ep, \lambda + \ep)} (A) \ne 0 \text{ für alle }
\ep > 0.
\end{align*}

\item
Nach der Definition des diskreten Spektrums und Aufgabenblatt 6 gilt
\begin{align*}
\lambda \in \sigma_{\disc}(A) \Leftrightarrow
\begin{cases}
\lambda \text{ ist Eigenwert von $A$ mit } \dim P_{\setd{\lambda}} < \infty\\
\text{ und }
B_{\ep}(\lambda) \cap \sigma(A) = \setd{\lambda}
\ \text{für ein}\ \ep > 0.
\end{cases}
\end{align*}
Aus $\lambda \in \sigma_\disc(A)$ folgt also
$P_{(\lambda - \ep, \lambda + \ep)} (A) = P_{\setd{\lambda}}$
und somit $\dim P_{(\lambda - \ep, \lambda + \ep)} \HH < \infty$.

Ist umgekehrt $\lambda \in \sigma(A)$ und
$n = \dim P_{(\lambda - \ep, \lambda + \ep)} \HH < \infty$.
Dann gilt,
\begin{align*}
\sigma(A)\cap(\lambda-\ep,\lambda+\ep)\subset\sigma(A \big|_{
P_{(\lambda - \ep, \lambda + \ep)}(A) \HH})
= \setd{ \lambda_1, \ldots, \lambda_m},\qquad m\leq n.
\end{align*}
Also ist $\lambda$ ein isolierter Eigenwert von $A$ mit endlicher Vielfachheit.

\item
Folgt aus (i) und (ii).\qedhere
\end{proofenum}
\end{proof}