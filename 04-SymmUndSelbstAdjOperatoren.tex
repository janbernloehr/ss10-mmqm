\chapter{Symmetrische und selbstadjungierte Operatoren}
%\renewcommand{\Hc}{\mathrm{H}}
\renewcommand{\bar}[1]{\overline{#1}}

\subsection{Der adjungierte Operator}

\begin{defn*}
Sei $\Hc$ ein komplexer Hilbertraum und sei $A : D(A) \subset \Hc \to \Hc$
linear und dicht definiert $(\bar{D} = \Hc)$. Der zu $A$ \emph{adjungierte
Operator}\index{Operator!adjungiert}
$A^{*} : D(A^{*}) \subset \Hc \to \Hc$ ist wie folgt definiert:
Falls zu gegebenem $\ph \in \Hc$ ein $\ph^{*} \in \Hc$ existiert,
so dass
\begin{align*}
\lin{\ph,A\eta} = \lin{\ph^{*},\eta}, \qquad \text{für alle}\ \eta\in D(A),
\end{align*}
dann ist  $\ph \in D(A^{*})$ und $A^{*}\ph := \ph^{*}$.\fish
\end{defn*}

Da $D(A)$ dicht ist in $\Hc$, ist $\ph^{*}$ durch $\ph$ eindeutig bestimmt,
und die Abbildung $\ph \mapsto \ph^{*}$ ist linear.
Nach dem Lemma von Fréchet-Riesz (FA~Thm~3.4.9 \cite{Fun07}) gilt:
%\begin{align*}
\[
   D(A^{*}) = \setdef{\ph\in\Hc}{\eta \mapsto \lin{\ph,A\eta}
\text{ ist stetig auf } D(A)}.
\]
%\end{align*}
\begin{proof}
Aus $\lin{\ph,A\eta} = \lin{\ph^{*},\eta}$ f\"ur alle
$\eta \in D(A)$ folgt unmittelbar die Stetigkeit von $\eta \mapsto
\lin{\ph,A\eta}$.
Falls umgekehrt $\eta \mapsto \lin{\ph,A\eta}$ stetig ist,
dann gibt es wegen $\overline{D(A)} = \Hc$ ein eindeutiges
$F \in \Hc^{*}$ mit $F(\eta) = \lin{\ph,A\eta}$
für $\eta \in D(A)$ (FA~Satz~3.3.8~\cite{Fun07}).
Nach Fréchet-Riesz gilt $F(\eta) = \lin{\ph^{*},\eta}$
für ein $\ph^{*} \in \Hc$.\qed
\end{proof}

\begin{bem*}
Aus $A \subset B$ folgt $A^{*} \supset B^{*}$.

Man beachte, dass $A^*$ nur existiert, wenn $A$ dicht definiert ist.
Damit $(A^*)^*$ existiert muss also $A^*$ wieder dicht definiert sein,
was im Allgemeinen nicht der Fall ist.~\map
\end{bem*}
%%-------------------Beispiel----------------------------------------------

\begin{bsp*}
\begin{bspenum}
\item Sei $\Hc = L^2(\R)$ und $g\in\Hc$ mit $\|g\|=1$. Wir definieren
$A \colon D \subset \Hc \to \Hc$ durch
\begin{align*}
D = C_{0}^{\infty}(\R),\qquad Af = f(0)g.
\end{align*}
Dann ist $D(A^*)=\setdef{\ph\in \Hc}{\ph \bot\ g}$ und $A^*=0$ auf $D(A^*)$.
\begin{proof}
Sei $\lin{\ph,g}=0$. Dann ist $\lin{\ph,Af}=0$ f\"ur alle $f\in
D(A)$. Also $\ph\in D(A^*)$ und $A^*\ph=0$.
Sei  $\lin{\ph,g}\neq 0$ und sei $(f_n)$ eine Folge in $D$ mit
$\norm{f_n}\to 0$ und $f_n(0)=1$. Dann gilt
$\lim\limits_{n\to\infty}\lin{\ph,Af_n} = \lin{\ph,g}\neq 0$. Somit ist
$f\mapsto \lin{\ph,Af}$ nicht stetig, also $\ph \not\in D(A^*)$.\qed
\end{proof}

\item
Sei $\Hc = L^2(\R)$ und $(e_n)_{n \in \N}$ eine Orthonormalbasis
von $L^2(\R)$. Der Operator $A \colon D \subset \Hc \to \Hc$ sei
definiert durch
$D = C_{0}^{\infty}(\R)$ und
%\[
\begin{align*}
Af = \sum_{n=0}^{\infty} f(n) e_n,
\end{align*}
%\]
wobei die Reihe konvergiert wegen $f \in C_{0}^{\infty}(\R)$.
Dann gilt $D(A^{*}) = (0)$.

% \begin{proof}
% Sei $g \in L^2(\R)$ mit $g \ne 0$. Wir zeigen, dass
% $f \mapsto \sprod{g}{Af}$ nicht stetig ist. Daraus folgt
% $g \notin D(A^{*})$. Da $g\neq 0$, gibt ein $n_0 \in \N$ mit
% $\sprod{e_{n_0}}{g} \ne 0$.
% Sei $(f_k)_{k \in \N}$ eine Folge in $D$ mit
% $\norm{f_k} \to 0$,
% \[
%   f_k(n_0) = 1 \quad \text{und} \quad f_k(n) = 0, n \ne n_0.
% \]
% Dann gilt $A f_k = f_k(n_0) e_{n_0} = e_{n_0}$, also
% \[
% \sprod{g}{Af_k} = \sprod{g}{e_{n_0}} \ne 0
% \]
% f\"ur alle $k$.
% Insbesondere $\sprod{g}{Af_k}\not\rightarrow 0$, w\"ahrend
% $f_k \to 0 \ (k \to \infty)$.
% \end{proof}
\begin{proof}
Übung.\qed
\end{proof}
\end{bspenum}
\end{bsp*}


%%--------------------------------------------------------------------
%%-------------Satz----------------------------------------------------
\medskip %Platz
\begin{prop}
\label{prop:4.1}
Sei $\Hc$ ein komplexer Hilbertraum und
$A : D \subset \Hc \to \Hc$ dicht definiert.
Dann gelten
\begin{propenum}
\item $A^{*}$ ist abgeschlossen.
\item
$A$ ist genau dann abschließbar, wenn $D(A^{*})$
dicht ist und dann gilt $\bar{A} = (A^{*})^{*}$.
\item
Falls $A$ abschließbar ist, gilt $\left(\bar{A}\right)^{*} = A^{*}$.
\end{propenum}
\end{prop}
\begin{proof}
\begin{proofenum}
\item Sei $\ph_n\in D(A^*)$ mit $\ph_n\to \ph\in \Hc$ und $A^*\ph_n\to \psi$.
Für alle $\eta\in D(A)$ gilt
\begin{align*}
\lin{\ph,A\eta} = \lim\limits_{n\to\infty}
\lin{\ph_n,A\eta} = \lim\limits_{n\to\infty} \lin{A^*\ph_n,\eta} =
\lin{\psi,\eta}.
\end{align*}
Somit ist $\ph\in D(A^*)$ und $A^*\ph =\psi$.
\item ``$\Leftarrow$'': Sei $D(A^*)\subset \Hc$ dicht und sei $\ph_n\to0$ mit
$A\ph_n\to \psi$. Zu zeigen ist, dass $\psi=0$. Nach Voraussetzung gilt für alle
$\eta\in D(A^*)$,
\begin{align*}
\lin{\eta,\psi} = \lim\limits_{n\to\infty}\lin{\eta,A\ph_n} = 
\lim\limits_{n\to\infty} \lin{A^*\eta,\ph_n} = 0.
\end{align*}
Also ist $\psi\in D(A^*)^\bot = (0)$.

``$\Rightarrow$'': Siehe (Thm. VII.1, \cite{RS95a}).
\item Aus $\bar{A} \supset A$ folgt $(\bar{A})^* \subset A^*$. Es bleibt noch zu
zeigen, dass $(\bar{A})^* \supset A^*$. Sei also $\ph\in D(A^*)$ und $\ph^* =
A^*\ph$, so gilt 
\begin{align*}
\lin{\ph,A\eta} = \lin{\ph^*,\eta}\text{ für alle }\eta\in D(A).\tag{*}
\end{align*}
Ist $\eta\in D(\bar{A})$, dann gibt es eine Folge $(\eta_n)$ in $D(A)$ mit
$\eta_n\to \eta$ und $A\eta_n\to \bar{A}\eta$. Mit (*) folgt somit
\begin{align*}
\lin{\ph,\bar{A}\eta} = \lin{\ph^*,\eta}\text{ für alle }\eta\in D(\bar{A}).
\end{align*}
Folglich ist $\ph\in D((\bar{A})^*)$ und $(\bar{A})^*\ph = \ph^*$. Also ist
$(\bar{A})^*\supset A^*$.\qed
\end{proofenum}
\end{proof}

\begin{defn*}
Der \emph{Kern}\index{Operator!Kern} eines linearen Operators $B : D
\subset \Hc \to \Hc$ ist definiert durch
\begin{align*}
\ker(B) := \setdef{\ph \in D}{B \ph = 0}.\fish
\end{align*}
\end{defn*}

\begin{prop}
\label{prop:4.2}
Sei $A:D(A)\subset \Hc\to \Hc$ dicht definiert. Dann gilt
\begin{align*}
\ker(A^*) = \ran(A)^\bot.
\end{align*}
Insbesondere ist $\ran(A)^\bot\subset D(A^*)$ und aus $\ker(A^*)=(0)$ folgt
$\bar{\ran(A)} = \Hc$.\fish
\end{prop}
\begin{proof}
Es folgt leicht
\begin{align*}
\ph\in \ker(A^*) &\Leftrightarrow
\lin{A^*\ph,\eta} = 0\forall \eta\in D(A)\\
&\Leftrightarrow
\lin{\ph,A\eta} = 0\forall \eta\in D(A)\\
&\Leftrightarrow
\ph\in\ran(A)^\bot.
\end{align*}
Außerdem folgt für $\ker(A^*)=(0)$ nach (FA Kor 3.8, \cite{Fun07}),
\begin{align*}
\Hc = (0)^\bot = \ker(A^*)^\bot
= (\ran(A)^\bot)^\bot = \bar{\ran(A)}.\qed 
\end{align*}
\end{proof}

\subsection{Symmetrie und Selbstadjungiertheit}

\begin{defn*}[Definition / Lemma]
Ein dicht definierter Operator $A:D(A)\subset \Hc\to\Hc$ heißt
\emph{symmetrisch}\index{Operator!symmetrisch}, falls $A\subset A^*$. Der
Operator $A$ heißt \emph{selbstadjungiert}\index{Operator!selbstadjungiert},
falls $A=A^*$.

$A$ ist genau dann symmetrisch, wenn
\begin{align*}
\lin{A\ph,\eta} = \lin{\ph,A\eta},\quad \text{für alle }\ph,\eta \in
D(A).\tag{*}
\end{align*}
$A$ ist genau dann selbstadjungiert, wenn (*) und zusätzlich $D(A)=D(A^*)$
gilt.\fish
\end{defn*}

\begin{bem*}[Bemerkungen.]
\begin{bemenum}
\item Für $A\in\Lc(\Hc)$ sind Symmetrie und Selbstadjungiertheit äquivalent.
\item Ein selbstadjungierter Operator $A$ hat keine symmetrische Erweiterung
$B\supset A$ mit $B\neq A$.
\begin{proof}
Aus $A\subset B\subset B^*$ und $A=A^*$ folgt $A=A^*\supset B^* \supset B\supset
A$ und somit $A=B$.\qed
\end{proof}
\item Falls $A\subset A^*$ und $B$ eine selbstadjungierte Erweiterung von $A$,
dann gilt $A\subset B\subset A^*$.\map
\end{bemenum}
\end{bem*}

\begin{bsp*}
Sei $\Hc=L^2(\R)$, $D=\SS(\R)$ und $A:D\subset \Hc\to \Hc$ definiert durch $f =
-if'$. Dann ist $A$ symmetrisch aber $A\neq A^*$. Es gilt $D(A^*) = H^1(\R)$ und
\begin{align*}
A^*f= \Fc^{-1}p \Fc f,\qquad f\in H^1(\R).
\end{align*}
$A^*$ ist selbstadjungiert, d.h. $(A^*)^*=A^*$.
\begin{proof}
Wir zeigen hier nur die Symmetrie. Die übrigen Eigenschaften weisen wir später
nach. Für alle $f,g\in \SS(\R)$ gilt
\begin{align*}
\lin{f,Ag} &= -i \int_\R \overline{f(x)}g'(x)\dx
= \lim\limits_{R\to\infty} -i \int_{-R}^R \overline{f(x)}g'(x)\dx\\
&= -\overline{f(x)}g(x)\big|_{-R}^R  + i \int_{-R}^R \overline{f'(x)}g(x)\dx\\
&= \int_\R \overline{-if'(x)}g(x)\dx = \lin{Af,g}.
\end{align*}
Somit ist $A\subset A^*$.\qed\boxc
\end{proof}
\end{bsp*}


Wir suchen nun nach handlichen Kriterien, um zu zeigen, dass ein Operator
selbstadjungiert ist. Dabei wird sich folgender Satz als äußert nützlich
erweisen.

\begin{bem*}[Notation.]
$\C_+ := \setd{z\in\C}{\Im z > 0}$, $\C_- = \setdef{z\in\C}{\Im z < 0}$.\map
\end{bem*}

\begin{prop}
\label{prop:4.3}
Sei $A: D(A)\subset \Hc\to \Hc$ symmetrisch.
\begin{propenum}
\item Für alle $\lambda,\mu\in\R$ und alle $\ph\in D(A)$ gilt
\begin{align*}
\norm{(A-\lambda-i\mu)\ph}^2 = \norm{(A-\lambda)\ph}^2 + \mu^2\norm{\ph}^2.
\end{align*}
\item Falls $\ran(A-z) = \Hc$ für ein $z\in\C_\pm$, so ist $\C_\pm\subset
\rho(A)$.\fish
\end{propenum}
\end{prop}
\begin{proof}
\begin{proofenum}
\item Sei $B=A-\lambda$. Dann gilt
\begin{align*}
\norm{(B-i\mu)\ph}^2
&= \lin{(B-i\mu)\ph,(B-i\mu)\ph}\\
&= \lin{B\ph,B\ph} \overline{-i\mu}\lin{\ph,B\ph}
- i\mu\lin{B\ph,\ph} \\
&\qquad\qquad\qquad
+(\overline{-i})(-i)\mu^2\lin{\ph,\ph}\\
&= 
\norm{B\ph}^2+\mu^2\norm{\ph}^2 
\underbrace{+i\mu\lin{\ph,B\ph}
- i\mu\lin{B\ph,\ph}}_{=0}.
\end{align*}
\item Sei $z\in\C_\pm$ und $\ran(A-z) =\Hc$. Nach Teil 1) gilt für
alle $\ph\in D(A)$,
\begin{align*}
\norm{(A-z)\ph} \ge \underbrace{\abs{\Im z}}_{>0}\norm{\ph}.\tag{*}
\end{align*}
D.h. $(A-z) : D(A)\subset \Hc \to \Hc$ ist injektiv , also nach Voraussetzung an
$z$ sogar bijektiv. Mit der Wahl $\ph=(A-z)^{-1}\psi$ folgt aus (*)
\begin{align*}
\norm{(A-z)^{-1}} \le \frac{1}{\abs{\Im z}}.
\end{align*}
Somit ist $z\in\rho(A)$ und nach Theorem \ref{prop:3.2} gilt außerdem
$B(z,\abs{\Im z}) \subset\rho(A)$.

Für jedes $z_1\in B(z,\abs{\Im z})$ folgt aus dem Gezeigten,
dass $B(z_1,\abs{\Im z}) \subset\rho(A)$. Durch Iteration dieses Arguments
schließen wir $\C_\pm\subset \rho(A)$.\qed
\end{proofenum}
\end{proof}

\begin{thm}
\label{prop:4.4}
Sei $A: D(A)\subset \Hc\to \Hc$ symmetrisch. Dann sind äquivalent
\begin{propenum}
\item\label{prop:4.4:1} $A=A^*$.
\item\label{prop:4.4:2} $\sigma(A) \subset \R$.
\item\label{prop:4.4:3} $\ran(A-z_\pm) = \Hc$ für ein $z_+\in \C_+$ und ein
$z_-\in \C_-$.
\item\label{prop:4.4:4} $A$ ist abgeschlossen und $\ker(A^*-z_\pm) = (0)$ für
ein $z_+\in \C_+$ und ein $z_-\in \C_-$.\fish
\end{propenum}
\end{thm}

\begin{bem*}
$\sigma(A)\subset\R$ allein impliziert \textit{nicht}, dass $A$ selbstadjungiert
ist.

Ein einfaches Gegenbeispiel ist $A=\bigl(\begin{smallmatrix}0 & 1\\ 0 &
0\end{smallmatrix}\bigr)$, denn $A$ hat nur den Eigenwert $0\in\R$, ist aber
offensichtlich nicht hermitesch.\map
\end{bem*}

\begin{proof}
``\ref{prop:4.4:1}$\Rightarrow$\ref{prop:4.4:4}'': Aus $A=A^*$ und Satz
\ref{prop:4.1} folgt, dass $A$ abgeschlossen ist. Außerdem gilt
\begin{align*}
\ker (A^*\pm i) = \ker (A\mp i) = (0)
\end{align*}
nach dem eben bewiesenen Satz \ref{prop:4.3}.

``\ref{prop:4.4:4}$\Rightarrow$\ref{prop:4.4:3}'': Aus $\ker(A- z_\pm) = (0)$
und Satz \ref{prop:4.2} folgt $\ran(A-\overline{z_\pm}) \subset \Hc$ ist dicht.
Wir zeigen $\ran(A-\overline{z_\pm}) = \Hc$ mit Hilfe der Abgeschlossenheit von
$A$. Sei $\psi\in \Hc$, so existiert aufgrund der Dichtheit eine Folge $(\ph_n)$
in $D(A)$ mit
\begin{align*}
(A-\overline{z_\pm})\ph_n \to \psi.
\end{align*}
Nach Satz \ref{prop:4.3} gilt
\begin{align*}
\norm{(A-\overline{z_\pm})(\ph_n-\ph_m)} \ge
\abs{\ph_n-\ph_m}\underbrace{\abs{\Im z_\pm}}_{>0}.
\end{align*}
Also ist $(\ph_n)$ Cauchyfolge in $\Hc$. Sei
$\ph=\lim\limits_{n\to\infty}\ph_n$, so folgt, da $A-\overline{z_+}$
abgeschlossen, dass $\ph\in D(A)$ und $(A-\overline{z_\pm})\ph=\psi$.
%Analog zeigt man $\ran(A-\overline{z_-})=\Hc$.

``\ref{prop:4.4:3}$\Rightarrow$\ref{prop:4.4:2}'': Folgt direkt aus Satz
\ref{prop:4.3}.

``\ref{prop:4.4:2}$\Rightarrow$\ref{prop:4.4:1}'': Da $A\subset A^*$ bleibt nur
$D(A^*)\subset D(A)$ zu zeigen. Sei also $\ph\in D(A^*)$.  Da $(A+i): D(A)\to
\Hc$ nach Voraussetzung surjektiv ist, existiert ein $\eta\in D(A)$ mit
$(A^*+i)\ph = (A+i)\eta = (A^*+i)\eta$. Also ist $(A^*+i)(\ph-\eta) = 0$ und
$\ker (A^*+i) = \ran(A-i)^\bot =(0)$, da $i\in\rho(A)$. Das heißt $\ph=\eta \in
D(A)$.\qed
\end{proof}

\subsection{Wesentliche Selbstadjungiertheit}

Sei $A: D(A)\subset \Hc\to\Hc$ ein symmetrischer Operator, so ist $A\subset
A^*$. Da $A^*$ abgeschlossen, ist $A$ abschließbar mit $\bar{A}\subset A^*$.
Falls $\bar{A}$ selbst-adjungiert ist, dann ist $\bar{A}$ die einzige
selbstadjungierte Erweiterung von $A$ (vgl. obige Bemerkung).
Dies motiviert folgende

\begin{defn*}
Ist der Abschluss eines Operators $A:D(A)\subset\Hc\to\Hc$ selbstadjungiert,
dann heißt $A$ \emph{wesentlich selbstadjungiert}\index{Operator!wesentlich
selbstadjungiert}. Sei $A$ abgeschlossen, so heißt jeder Teilraum $D\subset
D(A)$ mit $\bar{A\big|_D} = A$ \emph{definierender Bereich (core)} von
$A$\index{Operator!definierender Bereich}.\fish
\end{defn*}

\begin{thm}
\label{prop:4.5}
Sei $A: D(A)\subset \Hc\to \Hc$ ein symmetrischer Operator. Dann sind
äquivalent:
\begin{propenum}
\item\label{prop:4.5:1} $A$ ist wesentlich selbstadjungiert.
\item\label{prop:4.5:2} $\ran(A\pm i)\subset \Hc$ ist dicht.
\item\label{prop:4.5:3} $\ker(A^*\pm i) = (0)$.\fish
\end{propenum}
\end{thm}
\begin{proof}
``\ref{prop:4.5:1}$\Rightarrow$\ref{prop:4.5:3}'': Nach Voraussetzung
ist $\bar{A}$ selbstadjungiert und $(\bar{A})^* = A^*$ nach Satz
\ref{prop:4.1}. Also ist $\ker(A^*\pm i) = \ker ((\bar{A})^*\pm i) = (0)$ nach
Theorem \ref{prop:4.4}.

``\ref{prop:4.5:3}$\Rightarrow$\ref{prop:4.5:2}'': Folgt direkt aus Satz
\ref{prop:4.2}.

``\ref{prop:4.4:2}$\Rightarrow$\ref{prop:4.4:1}'': Wir zeigen $\ran(\bar{A}\pm
i) = \Hc$, dann folgt die Selbstadjungiertheit aus Theorem \ref{prop:4.4}. Sei
also $\psi\in \Hc$. Da $\bar{\ran(A-i)} = \Hc$ gibt es eine Folge $(\ph_n)$ in
$D(A)$ mit $(A-i)\ph_n \to \psi$. Nach Satz \ref{prop:4.3} ist $(\ph_n)$ eine
Cauchyfolge in $\Hc$. (Siehe Beweis von Theorem \ref{prop:4.4}) Sei
$\ph=\lim\limits_{n\to\infty} \ph_n$, dann gilt $A\ph_n \to \psi + i\ph$. Es
folgt $\ph\in D(\bar{A})$ und $\bar{A}\ph = \psi+i\ph$, d.h. $(\bar{A}-i)\ph =
\psi$.
Analog zeigt man $\ran(\bar{A}+i)=\Hc$.\qed
\end{proof}

Als Anwendung beweisen wir einen wichtigen Störungssatz.

\begin{thm}[Theorem von Kato-Rellich]
\label{prop:4.6}
Sei $A$ wesentlich selbstadjungiert und $B$ symmetrisch mit $D(B)\supset D(A)$.
Gilt weiterhin
\begin{align*}
\norm{B\ph} \le a\norm{A\ph} + b\norm{\ph}
\end{align*}
für alle $\ph\in D(A)$, wobei $a,b\in\R$ und $a<1$, so ist
\begin{align*}
A+B : D(A)\subset \Hc\to \Hc
\end{align*}
wesentlich selbstadjungiert und $D(\bar{A+B}) = D(\bar{A})$.\fish
\end{thm}

Wir werden später sehen, dass $-\Delta$ auf $\SS(\R)$ wesentlich
selbstadjungiert ist und $\abs{x}^{-1}$ die obige Abschätzung erfüllt, folglich
ist der Hamilton des Wasserstoffatoms wesentlich selbstadjungiert.

\begin{bem*}
Ist $A=A^*$ und $B\in\Lc(\Hc)$ symmetrisch, dann ist
\begin{align*}
A+B : D(A)\subset\Hc \to \Hc 
\end{align*}
selbstadjungiert nach Theorem \ref{prop:4.6}. (Wähle dazu $a=0$ und
$b=\norm{B}$). Beschränkte Störungen sind daher weitgehend
uninteressant.\map
\end{bem*}

\begin{proof}
\begin{proofenum}
\item
Wir betrachten zuerst den Fall, dass $A$ selbstadjungiert ist. Nach
Voraussetzung ist auch der Operator
$
A+B:D(A)\subset\Hc\to \Hc
$
symmetrisch. Es genügt daher nach Theorem \ref{prop:4.4} 
zu zeigen, dass $(A+B-i\mu)D(A) =\Hc$ für $\abs{\mu}$ hinreichend groß.

Nach Satz \ref{prop:4.3} gilt
\begin{align*}
\norm{(A-i\mu)\psi}^2 = \norm{A\psi}^2 + \mu^2\norm{\psi}^2,\qquad \psi\in D(A).
\end{align*}
Setzen wir $\psi=(A-i\mu)^{-1}\ph$ für ein $\mu\in\R\setminus\setd{0}$, so folgt
\begin{align*}
\norm{\ph} \ge \norm{A(A-i\mu)^{-1}\ph},\qquad
\norm{\ph} \ge \abs{\mu}\norm{(A-i\mu)^{-1}\ph}\tag{*}
\end{align*}
für alle $\ph\in\Hc$. Sei $\mu\in\R\setminus \setd{0}$, so gilt weiterhin
\begin{align*}
(A+B-i\mu) = (1+B(A-i\mu)^{-1})(A-i\mu),\tag{**}
\end{align*}
wobei $(A-i\mu):D(A)\to\Hc$ bijektiv ist. Ferner gilt nach Voraussetzung und (*)
\begin{align*}
\norm{B(A-i\mu)^{-1}\ph} &\le a \norm{A(A-i\mu)^{-1}\ph} +
b\norm{(A-i\mu)^{-1}\ph}\\
&\le \left(a+\frac{b}{\abs{\mu}}\right)\norm{\ph}.
\end{align*} 
Somit ist $\norm{B(A-i\mu)^{-1}}\le a+\frac{b}{\abs{\mu}} < 1$ für
$\abs{\mu}$ hinreichend groß. Für solche $\mu$ ist
\begin{align*}
(1+B(A-i\mu)^{-1}) : \Hc\to\Hc
\end{align*}
bijektiv (\cite{Fun07}, Thm 3.3.6), folglich ist nach (**) auch $A+B-i\mu$
bijektiv.
\item Sei nun $A$ lediglich wesentlich selbstadjungiert. Sei $\ph\in D(\bar{A})$
und $(\ph_n)_n$ eine Folge in $D(A)$ mit $\ph_n\to \ph$ und $A\ph_n\to
\bar{A}\ph$. Dann ist auch $B\ph_n$ Cauchyfolge, denn
\begin{align*}
\norm{B(\ph_n-\ph_m)} \le a\norm{A(\ph_n-\ph_m)} + b\norm{\ph_n-\ph_m}
\end{align*}
Also ist $\ph\in D(\bar{B})$ und $\ph\in D(\bar{A+B})$, denn $((A+B)\ph_n)$ ist
ebenfalls Cauchyfolge. Außerdem gilt durch Grenzwertübergang
\begin{align*}
\norm{\bar{B}\ph} \le a\norm{\bar{A}\ph} + b\norm{\ph},\qquad \ph\in D(\bar{A}).
\end{align*}
Somit ist $\bar{A}+\bar{B}$ nach dem 1. Teil selbstadjungiert und insbesondere
abgeschlossen. Also gilt
\begin{align*}
\bar{A}+\bar{B}\subset \bar{A+B},
\end{align*}
weil wir eben außerdem gezeigt haben, das $D(\bar{A})\subset D(\bar{A+B})$ folgt
$D(\bar{A+B}) = D(\bar{A})$ und $\bar{A+B} = \bar{A}+\bar{B}$.\qed
\end{proofenum}
\end{proof}

Die (reinen) Zustände eines quantenmechanischen Systems werden durch normierte
Vektoren eines komplexen Hilbertraums $\Hc$ beschrieben. Die Zeitevolution
eines Zustandes, gegeben durch $u\in\Hc$, ist bestimmt durch das
Anfangswertproblem für eine Schrödingergleichung
\begin{align*}
i\frac{\ddd}{\dt}\ph_t = H\ph_t,\qquad \ph_0 = u,\tag{S}
\end{align*}
wobei $H:D\subset\Hc\to\Hc$ im Allgemeinen unbeschränkt ist.

Wir nehmen im Folgenden an, dass $\bar{D}=\Hc$. Eine Lösung von (S) ist eine
differenzierbare\footnote{differenzierbar im Sinne von (iii)} Funktion $\ph:
I\to\Hc$, $I\subset\R$ ein nichtentartetes Intervall mit $0\in I$, so dass
\begin{equivenum}
\item $\ph_t \in D(H)$ für alle $t\in I$.
\item $\ph_0 = u$.
\item $\dfrac{\ddd}{\dt}\ph_t = \lim_{h\to 0} \dfrac{\ph_{t+h}-\ph_t}{h} =
-iH\ph_t$ für alle $t\in I$.
\end{equivenum}
\begin{lem*}
Falls (S) für jedes $u\in D$ eine Lösung $\ph$ besitzt mit $\norm{\ph_t} =
\norm{u}$ für alle $t\in I$, dann ist notwendigerweise $H\subset H^*$. Umgekehrt
folgt aus $H\subset H^*$, dass $\norm{\ph_t}$ erhalten und die Lösung
von (S) eindeutig ist.\fish
\end{lem*}
\begin{proof}
Sei $u\in D$ und $\ph_t$ eine Lösung von (S) mit $\norm{\ph_t} = \norm{u}$. Dann
gilt
\begin{align*}
0 &= \frac{\ddd}{\dt}\norm{\ph_t}^2\big|_{t=0}
= \lin{\dot{\ph}_t,\ph_t}+\lin{\ph_t,\dot{\ph}_t}\big|_{t=0}
=  \lin{-iHu,u} +\lin{u,-iHu}\\ & = i
\left[\underbrace{\lin{Hu,u}-\lin{u,Hu}}_{=0}\right]
\end{align*}
Aus $\lin{Hu,u} = \lin{u,Hu}$ für alle $u\in D$ folgt $H\subset H^*$.

Sind $\ph_t,\psi_t:I\to \Hc$ Lösungen von (S) und ist  $\Phi_t = \ph_t-\psi_t$,
dann gilt
\begin{align*}
i\frac{\ddd}{\dt}\Phi_t = H\Phi_t,\qquad \Phi_0 = 0. 
\end{align*}
Also folgt wegen $H\subset H^*$, dass $\norm{\Phi_t} = \norm{\Phi_0} = 0$ für
alle $t\in I$.\qed
\end{proof}

\begin{prop}
\label{prop:4.7}
Sei $H:D\subset\Hc\to\Hc$ symmetrisch. Falls (S) für jedes $u\in D$ eine globale
Lösung $\ph: \R\to \Hc$ besitzt, ist $H$ wesentlich selbstadjungiert.\fish
\end{prop}
\begin{proof}
Nach Theorem \ref{prop:4.5} genügt es zu zeigen, dass $\ker(H^*\pm i)=(0)$. Sei
also $u\in D$ und $\ph_t$ die Lösung von (S) zu $u$ und für ein $w\in D(H^*)$,
$(H^*+i)w =0$. Dann gilt
\begin{align*}
\frac{\ddd}{\dt}\lin{w,\ph_t}
= \lin{w,-iH\ph_t} = \lin{H^*w,-i\ph_t} = \lin{-iw,-i\ph_t} =
\lin{w,\ph_t}.
\end{align*}
Also gilt $\lin{w,\ph_t} = \lin{w,u}e^t$ und wegen
\begin{align*}
\abs{\lin{w,\ph_t}} \le \norm{w}\norm{\ph_t} = \norm{w}\norm{u}
\end{align*}
ist $\lin{w,u} = 0$ für alle $u\in D$ und da $\bar{D}=H$ folgt $w=0$. Analog
zeigt man, dass $\ker(H^*-i)=(0)$.\qed
\end{proof}

\begin{bem*}[Bemerkungen]
\begin{bemenum}
\item Wir werden sehen, dass $H=H^*$ auch hinrechend ist für die Existenz einer
globalen Lösung
\begin{align*}
\ph: \R\to \Hc
\end{align*}
von (S). Falls $H=H^*\in\Lc(\Hc)$, ist
\begin{align*}
\ph_t = e^{-iHt}\ph_0,\qquad
e^{-iHt} = \sum_{n=1}^\infty \frac{1}{n!} (-iHt)^n\ph_0.
\end{align*}
\item Nicht jeder symmetrische Operator $H$ hat eine selbstadjungierte
Erweiterung. Notwendig und hinreichend für die Existenz einer selbstadjungierten
Erweiterung ist, dass die \emph{Defektindizies}\index{Defektindizes}
\begin{align*}
&h_+ := \dim \ker(H^*-i) = \dim \ran(H^*+i)^\bot,\\
&h_- := \dim \ker(H^*+i) = \dim \ran(H^*-i)^\bot,
\end{align*}
übereinstimmen. (\cite{RS05a}, Kap. X)\map
\end{bemenum}
\end{bem*}
