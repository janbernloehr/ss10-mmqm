\chapter{Exponentieller Abfall von Eigenfunktionen}

Sei $\ph\in L^2(\R)$ eine $C^2$-Lösung der Schrödingergleichung
\begin{align*}
-\ph''  + V\ph = E\ph,
\end{align*}
wobei $V(x)>E$ für $x> R$.

%%%%%%%%%%%%%%%%%%%%%%%%%%%%%%%%%%%%%%%%%%%%%%%%%%%%%%%%---Bild 1---%%%%%%%%%%%%%%%%%%%%%%%%%%%%%%%%%%%%%%%%%%%%%%%%%%%%%%%%%%%%%%%%%%%
%%%%%%%%%%%%%%%%%%%%%%%%%%%%%%%%%%%%%%%%%%%%%%%%%%%%%%%%%%%%%%%%%%%%%%%%%%%%%%%%%%%%%%%%%%%%%%%%%%%%%%%%%%%%%%%%%%%%%%%%%%%%%%%%%%%%%%%
\begin{figure}[!htpb]
\centering
\begin{pspicture}(-1.5,-3.5)(8,2.5) 
 \psline{->}(-1,0)(8.5,0)
 \psline{->}(0,-3)(0,1.8)
 \psline[linestyle=dashed](4.1933,-0.7)(4.1933,1.8)
 \psline[linecolor=accentc](0,-0.4047)(6,-0.4047)

 \psplot[linewidth=1.2pt,plotpoints=200,linecolor=accent,algebraic=true]{0.47}{8}%
 	{(-22.5*x^2+11.25*x^3)*2.71828^(-1.5*x)+0.02}
 \psplot[linewidth=1.2pt,linecolor=accent,algebraic=true]{1.6}{8}% 
 	{-4.25*(2.71828^(-.143*x^2))-0.06}
 	
 \rput(-0.19,-0.38){\color{accentc}$E$}
 \rput(1.9,-2.956){\color{darkgray}$V$}
 \rput(2.1,1){\color{darkgray}$\ph$}
 \rput(6,-0.9){\color{darkgray}$V>E$}
 \rput(6.2,1.43){\color{darkgray}exponentieller Abfall}
 \pscurve[linecolor=gray](5.8,1.2)(5.35,0.8)(5.9,0.75)(5.6,0.4)
\end{pspicture}
\caption{$-\ph^{\prime\prime}+V\ph=E\ph$ mit $\ph\in L^2(\R)$}
\end{figure}
%%%%%%%%%%%%%%%%%%%%%%%%%%%%%%%%%%%%%%%%%%%%%%%%%%%%%%%%%%%%%%%%%%%%%%%%%%%%%%%%%%%%%%%%%%%%%%%%%%%%%%%%%%%%%%%%%%%%%%%%%%%%%%%%%%%%%%%%
Es gilt
\begin{enumerate}
\item[] $V(x)<E \quad\Rightarrow\quad  \ph^{\prime\prime}(x),\ph(x)$ haben verschiedenes Vorzeichen,
\item[] $V(x)>E \quad\Rightarrow\quad  \ph^{\prime\prime}(x),\ph(x)$ haben gleiches Vorzeichen.
\end{enumerate}
Im Bereich $x>R$ ist $V(x)>E$, also muss $\ph$ dort strikt konvex sein wenn $\ph(x)>0$, und strikt konkav wenn $\ph(x)<0$.
%%%%%%%%%%%%%%%%%%%%%%%%%%%%%%%%%%%%%%%%%%%%%---Bild 2---%%%%%%%%%%%%%%%%%%%%%%%%%%%%%%%%%%%%%%%%%%%%%%%%%%%%%%%%%%%%%%%%%%%%%%%%%%%%%%%
%%%%%%%%%%%%%%%%%%%%%%%%%%%%%%%%%%%%%%%%%%%%%%%%%%%%%%%%%%%%%%%%%%%%%%%%%%%%%%%%%%%%%%%%%%%%%%%%%%%%%%%%%%%%%%%%%%%%%%%%%%%%%%%%%%%%%%%%
\begin{figure}[!htpb]
\centering
\begin{pspicture}(-1,-2)(6.4,3)
 \psline[arrowsize=4pt]{->}(0,-1)(0,2)
 
 \pscurve[linecolor=accentb,linewidth=1.2pt](0,1)(1,0.5)(5,2)
 
 \psplot[linewidth=1.2pt,linecolor=accent,algebraic=true]{0}{5}%
 	{2.71828^(-x)}
 \psplot[linewidth=1.2pt,linecolor=accentb,algebraic=true]{1}{4}%
 	{ln(-0.25*(x-5))}
 \psplot[linewidth=1.2pt,linecolor=accentb,algebraic=true]{0}{1}%
	{-0.78125*x^3+2.3125*x^2-2.53125*x+1}
  \psline[arrowsize=4pt]{->}(-2,0)(6,0)
  
  \rput(-0.3,-0.3){\color{darkgray}$R$}
  \rput(5.6,0.3){\color{darkgray}$\ph\in L^2$}
  \rput(5.7,2.1){\color{darkgray}$\ph\not\in L^2$}
\end{pspicture}
\caption{Möglicher Verlauf von $\ph$ für $x>R$.}
\end{figure}
Wegen $\ph\in L^2(\R)$ kommt nur der (exponentielle) Abfall
$\lim_{|x|\to\infty}\ph(x)=0$ in Frage. Wir beweisen ein analoges Resultat im $\R^n$.

\begin{prop}[IMS-Formel]
\label{prop:2.1}
Sei $H=-\Delta : H^2(\R^n)\to H^2(\R^n)$ aufgefasst als linearer Operator und
$f\in C^{\infty}(\R^n\to\R)$ mit $\partial^{\alpha}f\in L^{\infty}(\R^n)$ für $|\alpha|\leq 2$. Dann gilt
\begin{align*}
   fHf = \frac{1}{2}\left(f^2 H + H f^2\right) + |\nabla f|^2.
\end{align*} 
auf $H^2(\R^n)$, wobei $f$, $f^2$ und $\abs{\nabla f}^2$ als
Multiplikationsoperatoren aufzufassen sind.\fish
\end{prop}

\begin{bem*}
Der Satz gilt unverändert auch für
\begin{align*}
H = -\Delta + V,\qquad V: \R^n\to \C,
\end{align*}
wenn $V: H^2(\R^n)\to L^2(\R^n)$.\map
\end{bem*}

\begin{proof}
Nach Satz \ref{prop:1.8} ist $H^2(\R^n)$ invariant unter $f$ und $f^2$
und somit gilt auf $H^2(\R^n)$,
\begin{align*}
 fHf &= f^2H+f[H,f]\tag{1}\\
 fHf &= Hf^2-[H,f]f\tag{2}
\end{align*}
wobei für alle $\ph \in H^2(\R^n)$ gilt
\begin{align*}
[H,f]\ph &= (Hf-fH)\ph\\
  &= -\Delta(f\ph)+f\Delta\ph\\
  &= (-\Delta f)\ph-2(\nabla f)\cdot\nabla\ph.
\end{align*}
Also ist
\begin{align*}
 [H,f]f\ph &= (-\Delta f)f\ph-2(\nabla f)\cdot\nabla(f\ph)\\
    &= (-\Delta f)f\ph-2f(\nabla f)\cdot\nabla\ph-2|\nabla f|^2\ph\\
    &= f[H,f]\ph-2|\nabla f|^2\ph.\tag{3}
\end{align*}
Nach (3) gilt
\begin{align*}
f[H,f] - [H,f]f = 2\abs{\nabla f}^2.\tag{4} 
\end{align*}
Aus (1), (2) und (4) folgt nun die Behauptung nach Addition von (1) und
(2).\qed
\end{proof}

\begin{defn*}
Sei $V:\R^n\to\R$ messbar mit $V\ph\in L^2(\R^n)$ für alle $\ph\in H^2(\R^n)$.
Wir definieren die \emph{Ionisierungsschwelle}\index{Ionisierungsschwelle}
$\Sigma\le \infty$ von $H=-\Delta+V$ durch
\begin{align*}
&\Sigma = \lim\limits_{R\to\infty} \Sigma_R,\\
&\Sigma_R = \inf\setdef{\lin{\ph,H\ph}}{\ph\in H^2(\R^n),\; \norm{\ph} = 1,\;
\ph(x) = 0\text{ für } \abs{x}< R}.\fish
\end{align*}
\end{defn*}

Anschaulich beschreibt die Ionisierungsschwelle die kleinste
Energie des Atoms mit einem Elektron weniger. In den einfachsten Fällen ist
$\Sigma=\lim\limits_{\abs{x}\to\infty} V(x)$, beim Potential $V=x^2$ ist
beispielsweise $\Sigma=\infty$.

\begin{thm}
\label{prop:2.2}
Sei $V:\R^n\to\R$ messbar mit $V\ph\in C^2(\R^n)$ für alle $\ph\in H^2(\R^n)$.
Ist $\ph\in H^2(\R^n)$ und $H\ph = E\ph$, wobei $E< \Sigma$, dann ist
$e^{\beta\abs{x}}\ph(x)\in L^2(\R^n)$ für alle $\beta\in\R$, so dass
$E+\beta^2< \Sigma$.\fish
\end{thm}
\begin{proof}
Sei $\chi\in C_0^\infty(\R^n\to[0,1])$ mit
\begin{align*}
\chi(x) = 
\begin{cases}
1, & \abs{x}\ge 2,\\
0, & \abs{x} \le 1
\end{cases}
\end{align*}
und sei $\chi_R(x) := \chi(R^{-1}x)$ für $R> 0$.

\begin{figure}[!htpb]
\begin{center}
\begin{pspicture}(-2,-1.5)(8,2)
 \psline{->}(-1.5,0)(7,0)
 \psline{->}(0,-1)(0,2)
 \psplot[linewidth=1.18pt,linecolor=accent,algebraic=true]{2}{4}%
 	{0.1875*(x-2)^5-0.9375*(x-2)^4+1.25*(x-2)^3}
 \psline[linewidth=1.18pt,linecolor=accent](-0.75,0)(2,0)
 \psline[linewidth=1.18pt,linecolor=accent](4,1)(6.5,1)
 \rput(2,-0.3){\color{darkgray}$R$}
 \rput(3.9,-0.3){\color{darkgray}$2R$}
 \rput(-0.2,1){\color{darkgray}$1$}
 \rput(5,1.35){\color{darkgray}$\chi_R$}
 \psline(0,1)(-0.08,1)
 \psline(2,0)(2,-0.08)
 \psline(3.9,0)(3.9,-0.08)
\end{pspicture}
\end{center}
\caption{Zur Abschneidefunktion.}
\end{figure}

Für $0<\ep < 1$ definieren wir
\begin{align*}
f(x) = \frac{\beta\abs{x}}{1+\ep\abs{x}}.
\end{align*}
Dann gilt $\abs{f} \le \ep^{-1}\beta$ und $\abs{\nabla f} \le \beta$
für $\abs{x}> 0$. Sei $G=\chi_R e^f$, dann ist $G\in L^\infty\cap
C^\infty(\R^n)$ und $\partial^\alpha G\in L^\infty(\R^n)$ für $\abs{\alpha}\le
2$.

Aus der IMS-Formel folgt nun
\begin{align*}
\lin{G\ph,(H-E)G\ph} &= \lin{\ph, G(H-E)G\ph}
= \frac{1}{2}\lin{\ph,G^2(H-E)\ph} \\ &+ \frac{1}{2}\lin{\ph,(H-E)G^2\ph}
+ \lin{\ph,\abs{\nabla G}^2\ph}
\end{align*}
Da $(H-E)$ ein symmetrischer Operator auf $L^2$ ist, verschwinden die ersten
beiden Terme. Zum dritten Term betrachte
\begin{align*}
\nabla G &= \nabla \chi_R e^f + \chi_R e^f \nabla f 
=  \nabla \chi_R e^f + G f,\\
\Rightarrow\;\abs{\nabla G} &= \nabla \chi_R^2 e^{2f} + 2\nabla \chi_R \nabla f
e^f G + G^2\abs{\nabla f}^2.
\end{align*}
Die ersten beiden Terme sind in der $L^\infty$-Norm durch eine von
$\ep$ unabhängige Konstante $C_R$  beschränkt, denn $\nabla \chi_R$ hat
kompakten Träger. Es folgt
\begin{align*}
\lin{G\ph,(H-E-\abs{\nabla f}^2)G\ph} \le C_R\norm{\ph}^2,\tag{1}
\end{align*}
und andererseits
\begin{align*}
\lin{G\ph,(H-E-\abs{\nabla f}^2)G\ph} &\ge \lin{G\ph,(H-E-\beta^2)G\ph}
\\ &\ge (\Sigma_R-E-\beta^2)\norm{G\ph}^2.\tag{2}
\end{align*}
Da $\Sigma_R\to\Sigma> E+\beta^2$ existiert ein $R> 0$, so dass $\Sigma_R >
E+\beta^2$. Für dieses $R$ folgt aus (1) und (2), dass
\begin{align*}
\norm{G\ph}^2 \le \frac{C_R}{\Sigma_R-E-\beta^2} \norm{\ph}^2.
\end{align*}
Also gilt
\begin{align*}
\int_{\setd{x\ge 2R}} e^{2\beta\abs{x}}\abs{\ph(x)}^2\dx
&=
\int_{\setd{x\ge 2R}} \lim\limits_{\ep \to 0} e^{2f(x)}\abs{\ph(x)}^2\dx\\
&\overset{\text{mon.konv.}}{=}
\lim\limits_{\ep \to 0}
\int_{\setd{x\ge 2R}}  e^{2f(x)}\abs{\ph(x)}^2\dx\\
&\le
\lim\limits_{\ep\to 0}
\norm{G\ph}
\le
\frac{C_R}{\Sigma_R-E-\beta^2} \norm{\ph}^2
< \infty.
\end{align*}
Andererseits ist auch
\begin{align*}
\int_{\setd{x< 2R}} e^{2\beta\abs{x}}\abs{\ph(x)}^2\dx
\le
e^{4\beta R}\norm{\ph}^2 < \infty.\qed
\end{align*}
\end{proof}

\begin{prop}
\label{prop:2.3}
Sei $V:\R^n\to\R$ messbar mit $V\ph\in C^2(\R^n)$ für alle $\ph\in H^2(\R^n)$
und $\ph\in H^2(\R^n)$ mit $H\ph = E\ph$. Sei
\begin{align*}
\sup\limits_{x\in\R^n} \int_{\abs{y-x}\le 1}
\abs{V_-(x)}^p\dy < \infty,
\end{align*}
für ein $p>\frac{n}{2}$ und $n\ge 2$. Dann gilt für alle $x\in\R^n$
\begin{align*}
\esssup\limits_{y:\abs{y-x}\le \frac{1}{2}} \abs{\ph(y)} \le
C\norm{\ph}_{L^2(B_1(x))}.\fish
\end{align*}
\end{prop}
\begin{proof}
Siehe Übungsblatt 3.\qed
\end{proof}

\begin{cor}
\label{prop:2.4}
Sei $V:\R^n\to\R$ messbar mit $V\ph\in C^2(\R^n)$ für alle $\ph\in H^2(\R^n)$
und $\ph\in H^2(\R^n)$ mit $H\ph = E\ph$. Dann gibt es zu jedem $\beta\in\R$
mit $E+\beta^2<\Sigma$ ein $C_\beta \in\R$, so dass
\begin{align*}
\abs{\ph(x)} \le C_\beta e^{-\beta\abs{x}}\quad \text{f.ü.}\fish
\end{align*}
\end{cor}
\begin{proof}
Für jedes $x\in\R^n$ gilt
\begin{align*}
\esssup\limits_{y: \abs{x-y}\le \frac{1}{2}}
e^{\beta\abs{y}}\ph(y)
&\le
e^{\beta \abs{x}}
e^{\beta/2}
\esssup\limits_{y: \abs{x-y}\le \frac{1}{2}}
\ph(y)
\le
e^{\beta \abs{x}}
e^{\beta/2}
C\norm{\ph}_{L^2(B_1(x))}\\
&\le
C\norm{e^{\beta\abs{\cdot}}\ph}_{L^2(B_1(x))}
e^{3/2\beta}
\le
C\norm{e^{\beta\abs{\cdot}}\ph}
e^{3/2\beta}
< \infty.\qed
\end{align*}
\end{proof}

\begin{bem*}
\begin{bemenum}
\item Mit den Methoden des Beweises von Theorem \ref{prop:2.2} lassen sich für
$n$-Teilchensysteme verbesserte, anisotrope Schranken herleiten.
\item Solche Schranken existieren nicht nur für Schrödinger-Operatoren, sondern
auch für Operatoren die ein Teilchen in einem quantisierten Strahlungsfeld
beschreiben,
\begin{align*}
H = (-i\nabla_x  + A(x))^2 + H_f + V,
\end{align*}
wobei $A(x)$ das quantisierte Vektorpotential darstellt und $H_f$ die
Feldenergie.\map
\end{bemenum}
\end{bem*}
% 
% \noindent
% \emph{Bemerkung.} Nach Satz 8.8 ist $V$ relativ kompakt 
% bezüglich $-\Delta/2$ und somit ist $H$ selbstadjungiert mit $D(H)=H^2(\R^n)$ und
% $\sigma_{\rm ess}(H)=[0,\infty)$ nach Theorem 8.12.
% 
% \begin{bew}
% Sei $\chi\in C^{\infty}(\R^n,[0,1])$ mit
% \[\chi(x)=
% \begin{cases}
%  1 & |x|\geq 2,\\
%  0 & |x|\leq 1.
% \end{cases}\]
% Sei $\chi_R(x)=\chi(x/R)$ und $f(x)=\frac{\beta|x|}{1+\eps|x|}$, $\eps>0, R>0$.
% %%%%%%%%%%%%%%%%%%%%%%%%%%%%---Bild 3---%%%%%%%%%%%%%%%%%%%%%%%%%%%%%%%%%%%%%%%%%%%%%%%%%%%%%%%%%%%%%%%%%%%%%%%%%%%%%%%%%%%%%%%%%%%%%%%%%%%%%
% %%%%%%%%%%%%%%%%%%%%%%%%%%%%%%%%%%%%%%%%%%%%%%%%%%%%%%%%%%%%%%%%%%%%%%%%%%%%%%%%%%%%%%%%%%%%%%%%%%%%%%%%%%%%%%%%%%%%%%%%%%%%%%%%%%%%%%%%%%%%%
% \begin{center}
% \begin{pspicture}(-2,-1.5)(8,2)
%  \psline[linewidth=0.5pt,arrowsize=4pt]{->}(-1.5,0)(7,0)
%  \psline[linewidth=0.5pt,arrowsize=4pt]{->}(0,-1)(0,2)
%  \psplot[linewidth=1.2pt,algebraic=true]{2}{4}{0.1875*(x-2)^5-0.9375*(x-2)^4+1.25*(x-2)^3}
%  \psline[linewidth=1.18pt](-0.75,0)(2,0)
%  \psline[linewidth=1.18pt](4,1)(6.5,1)
%  \rput(2,-0.3){$R$}
%  \rput(3.9,-0.3){$2R$}
%  \rput(-0.2,1){$1$}
%  \rput(5,1.35){$\chi_R$}
%  \psline[linewidth=0.5pt](0,1)(-0.08,1)
%  \psline[linewidth=0.5pt](2,0)(2,-0.08)
%  \psline[linewidth=0.5pt](3.9,0)(3.9,-0.08)
% \end{pspicture}
% \end{center}
% %%%%%%%%%%%%%%%%%%%%%%%%%%%%%%%%%%%%%%%%%%%%%%%%%%%%%%%%%%%%%%%%%%%%%%%%%%%%%%%%%%%%%%%%%%%%%%%%%%%%%%%%%%%%%%%%%%%%%%%%%%%%%%%%%%%%%%%%%%%%%
% Es gilt $|f|\leq\beta/\eps$ und für $x\neq0$
% \[
%   |\nabla f(x)|\leq \beta.
% \]
% Sei $G=\chi_Re^f.$ Dann $G\in C^{\infty}(\R^n)\cap L^{\infty}(\R^n)$
% \begin{eqnarray*}
%  \nabla G &=& \nabla\chi_Re^f+\chi_Re^f\nabla f\\
%    &=& \nabla\chi_Re^f+G\nabla f\quad\in L^{\infty}
% \end{eqnarray*}
% und $\partial_i\partial_jG\in L^{\infty}(\R^n)$ für alle $i,j$. Also nach Satz 2
% \begin{eqnarray*}
%  \sprod{G\ph}{(H-E)G\ph} &=& \sprod{\ph}{|\nabla G|^2\ph}\frac{1}{2}\\
%    &=& \sprod{\ph}{(\nabla\chi_Re^f+G\nabla f)^2\ph}\frac{1}{2}\\
%    &=& \frac{1}{2}\sprod{\ph}{\left(|\nabla\chi_R|^2e^{2f}+2\nabla\chi_R\nabla fe^fG+G^2|\nabla f|^2\right)\ph}
% \end{eqnarray*}
% Somit
% \begin{eqnarray}
%  \sprod{G\ph}{\left(H-E-|\nabla f|^2/2\right)G\ph} &\leq&%
%   \sup_x\left(|\nabla\chi_R|e^{2\beta|x|}+2|\nabla\chi_R|\beta e^{2\beta|x|}\right)\|\ph\|^2 \nonumber\\
%      &=& C_R\|\ph\|^2 \label{p:9.3:1}
% \end{eqnarray}
% wobei $C_R<\infty$. Andererseits
% \begin{eqnarray}
%  \sprod{G\ph}{\left(H-E-|\nabla f|^2/2\right)G\ph} &\geq& \sprod{G\ph}{\left(V-E-\beta^2/2\right)G\ph}\nonumber\\
%   &\geq& \left[\left(\inf_{|x|\geq R}V(x)\right)-E-\beta^2/2\right]\|G\ph\|^2 \label{p:9.3:2}\\
%   &=:& \delta_R\|G\ph\|^2.\nonumber
% \end{eqnarray}
% Nach Voraussetzung existiert $R>0$ hinreichend groß, so dass $\delta_R>0$. Halte dieses $R$ fest. 
% Dann folgt aus \eqref{p:9.3:1} und \eqref{p:9.3:2}
% \[
%   \|G\ph\|^2\leq \frac{C_R}{\delta_R}\|\ph\|^2
% \]
% für alle $\eps>0$. Also
% \begin{eqnarray*}
%  \int e^{2\beta|x|}|\ph(x)|^2\:dx &=& \int\lim_{\eps\to0}e^{2f(x)}|\ph(x)|^2\:dx\\
%    &=& \lim_{\eps\to0}\int e^{2f}|\ph|^2\:dx~\leq~\frac{C_R}{\delta_R}\|\ph\|^2~<~\infty.
% \end{eqnarray*}
% \end{bew}
% 
% \noindent{\it Bemerkung.} Unter den Voraussetzungen des Theorems gibt es eine Konstante $C=C(V)$, 
% so dass für alle $x\in\R^n$
% \[
%   \|\ph\|_{L^{\infty}(B_1(x))}\leq C\|\ph\|_{L^2(B_1(x))}
% \]
% (Agmon: Thm. 5.1), somit folgt aus Theorem 3:
% \[
%   |\ph(x)|\leq C_{\beta}e^{-\beta|x|}
% \]
% für $\beta>0$ mit $E+\beta^2/2<0$.
% 
% \subsection*{Regularität der Eigenfunktionen}
% 
% Nach dem Sobolev-Lemma (F.A 5.2.2) ist $H^2(\R^n)\subset C^k(\R^n)$ falls $k<2-n/2$.
% Also sind für $n\leq3$ Eigenfunktionen stetig.
% 
% \begin{thm}
% Sei $H=-\Delta/2+V$ definiert auf $H^2(\R^n)$, $\ph\in H^2(\R^n)$ und
% $H\ph=E\ph$ für $E\in\C$. Falls $\Omega\subset\R^n$ offen ist und $V\restricted\Omega\in C^{\infty}(\Omega)$
% dann ist $\ph\in C^{\infty}(\Omega)$.
% \end{thm}
% 
% \noindent
% \emph{Bemerkung.} Falls $V\in C^{m}(\Omega)$ und $k<2+m-\frac{n}{2}$ dann 
% $\ph\in C^{k}(\Omega)$ (Reed \& Simon Kap. IX.6)
% 
% \begin{bew}
% Wir zeigen, dass $\gamma\ph\in\bigcap_{m\geq2}H^m(\R^n)$ für alle $\gamma\in C_0^{\infty}(\Omega)$.
% Dann folgt $\ph\in C^{\infty}(\Omega)$, denn $\bigcap_{m}H^m(\R^n)\subset C^{\infty}(\R^n)$ nach dem
% Sobolev-Lemma.
% 
% Da $\ph\in H^2(\R^n)$, ist auch $\gamma\ph\in H^2(\R^n)$ für alle $\gamma\in C_0^{\infty}(\Omega)$.
% Sei $m\geq2$ und sei $\gamma\ph\in H^m(\R^n)$ für alle $\gamma\in C_0^{\infty}(\Omega)$. Dann gilt für
% $\gamma\in C_0^{\infty}(\Omega)$
% \begin{eqnarray}
%  \Delta(\gamma\ph) &=& (\Delta\gamma)\ph+2\nabla\gamma\cdot\nabla\ph+\gamma(\Delta\ph)\nonumber\\
%    &=& (\Delta\gamma)\ph+2\nabla\gamma\cdot\nabla\ph+(V-E)\gamma\ph \label{p:9.4:1}
% \end{eqnarray}
% wobei
% \begin{equation}
%   \nabla\gamma\nabla\ph=\div((\nabla\gamma)\ph)-(\Delta\gamma)\ph \label{p:9.4:2}
% \end{equation}
% Nach Annahme sind $(\Delta\gamma)\ph,(V-E)\gamma\ph$ und $(\nabla\gamma)\ph$ in $H^m(\R^n)$.
% Also ist $\div((\nabla\gamma)\ph)\in H^{m-1}(\R^n)$ nach F.A Lemma 5.3.3. Aus \eqref{p:9.4:1} und \eqref{p:9.4:2}
% folgt nun, dass $\Delta(\gamma\ph)\in H^{m-1}(\R^n)$, d.h.
% \[
%   \int \abs{p^2\widehat{\gamma\ph}(p)}^2(1+p^2)^{m-1}\:dp<\infty
% \]
% und somit
% \[
%   \int \abs{\widehat{\gamma\ph}(p)}^2(1+p^2)^{m+1}\:dp<\infty.
% \]
% Also ist $\gamma\ph\in H^{m+1}(\R^n)$ für alle $\gamma\in C_0^{\infty}(\Omega)$.
% \end{bew}
