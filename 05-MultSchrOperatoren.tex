\chapter{Multiplikations- und Schrödingeroperator}

\subsection{Multiplikationsoperatoren}
Ein Schrödingeroperator $-\Delta + V$ wird im einfachsten Fall $(V=0)$ zu
$-\Delta$. Unter Fouriertransformation geht $-\Delta$ über in einen
Multiplikationsoperator.

\begin{defn*}
Eine messbare Funktion $f: \R^n\to \C$ definiert einen
\emph{Multiplikationsoperator}
$
M_f : D_f \subset L^2(\R^n)\to L^2(\R^n)
$
durch
\begin{align*}
D_f := \setdef{\ph\in L^2(\R^n)}{f\ph\in L^2(\R^n)},\quad
(M_f\ph)(x) = f(x)\ph(x).\fish
\end{align*}
\end{defn*}
Man sieht leicht ein, dass $D_f\subset L^2(\R^n)$ dicht ist. 

\begin{prop}
\label{prop:5.1}
Das Spektrum von $M_f$ ist der \emph{wesentliche Wertebereich} von $f$. D.h.
\begin{align*}
\sigma(M_f) := \setdef{z\in \C}{\forall \ep >0 : \abs{f^{-1}(B_\ep(z))}>
0},
\end{align*}
wobei $\abs{\cdot}$ das Lebesgue-Maß  bezeichnet. Ist $f$ reellwertig, dann ist
$M_f$ selbstadjungiert.\fish
\end{prop}

Ist $f$ stetig, dann entspricht der wesentliche Wertebereich dem Wertebereich.

\begin{proof}
Sei $W_f$ der wesentliche Wertebereich von $f$ und $z\in W_f$. Wir zeigen, dass
$z\in \sigma(M_f)$. Zu jedem $k\in\N$ existiert eine messbare Menge
$E_k\subset\R^n$ mit
\begin{align*}
E_k \subset f^{-1}(B_k(z)),\qquad 0 < \abs{E_k}< \infty.
\end{align*}
Wähle $E_k = f^{-1}(B_{1/k}(z))$, falls $\abs{f^{-1}(B_{1/k}(z))} < \infty$,
sonst wähle eine messbare Teilmenge von $f^{-1}(B_{1/k}(z))$. Sei
\begin{align*}
\ph_k = \frac{1}{\sqrt{\abs{E_k}}}\chi_{E_k}.
\end{align*}
Dann ist $\ph_k \in L^2(\R^n)$, $\norm{\ph_k}=1$ und $f\ph_k \in L^2$, denn
$\abs{f\big|_{E_k}-z} < \frac{1}{k}$ und
\begin{align*}
\norm{(z-M_f)\ph_k} &= \left(\int \abs{(z-f(x))\ph_k(x)}^2\dx \right)^{1/2}\\
&=
 \sqrt{\abs{E_k}}^{-1} \left(\int_{E_k} \abs{z-f(x)}^2\dx\right)^{1/2} 
 \le \frac{1}{k}\to 0.
\end{align*}
Somit ist $z\in \sigma(M_f)$ (vgl. Aufgabe 4.4).

Ist andererseits $z\notin W_f$, dann existiert ein $\ep > 0$, so dass
\begin{align*}
\abs{f^{-1}(B_\ep(z))} = 0.
\end{align*}
Definiere nun
\begin{align*}
g(x) =
\begin{cases}
(z-f(x))^{-1},& \abs{z-f(x)} \ge \ep,\\
0, & \text{sonst}.
\end{cases}
\end{align*}
Dann ist $g\in L^\infty(\R^n)$, also $M_g$ ein beschränkter Operator und
$(z-f(x))g(x) = 1$ fast überall. Also $fg\in L^\infty(\R^n)$ und somit
$M_gL^2(\R^n)\subset D_f$. Es folgt
\begin{align*}
\ph\in L^2(\R^n) \Rightarrow
(z-M_f)M_g\ph = (z-f)g\ph = \ph\text{ f.ü.},\\
\ph\in D_f \Rightarrow M_g(z-M_f)\ph = g(z-f)\ph = \ph\text{ f.ü.}.
\end{align*}
Somit ist $z\in \rho(M_f)$ und $M_g=(z-M_f)^{-1}$.

Falls $f$ reellwertig ist, so ist $M_f$ symmetrisch (Übung) und $\sigma(M_f) =
W_f \subset\R$, also ist $M_f$ nach Theorem \ref{prop:4.4}
selbstadjungiert.\qed
\end{proof}

\begin{bsp}
\begin{bspenum}
\item Der Operator $-i\frac{\ddd}{\dx}: H^1(\R)\subset L^2(\R) \to L^2(\R)$
definiert durch
\begin{align*}
-i\frac{\ddd}{\dx} = \Fc^{-1}M_f\Fc,\qquad f(p)=p,
\end{align*}
ist selbstadjungiert auf $H^1$, da $M_f$ selbstadjungiert. Auf $C_0^\infty(\R)$
ist $-i\frac{\ddd}{\dx}$ wesentlich selbstadjungiert und
$\sigma(-i\frac{\ddd}{\dx}) = \sigma(M_f) = \R$.
\item Der Operator $-\Delta : H^2(\R^n)\subset L^2(\R^n)\to L^2(\R^n)$ definiert
durch
\begin{align*}
-\Delta := \Fc^{-1}M_f\Fc,\qquad f(p) = p^2 = \sum_{k=1}^n p_k^2
\end{align*}
ist selbstadjungiert und wesentlich selbstadjungiert auf $C_0^\infty(\R^n)$ und
$\sigma(-\Delta) = [0,\infty)$. Der Beweis erfolgt mit Satz
\ref{prop:1.3}, Satz \ref{prop:5.1} und den Resultaten aus den Übungen.\boxc
\end{bspenum}
\end{bsp}

\subsection{Schrödingeroperatoren}

\begin{defn*}
\emph{Schrödingeroperatoren} sind Operatoren der Form
\begin{align*}
-\Delta + V : D\subset L^2(\R^n)\to L^2(\R^n),\tag{*}
\end{align*}
wobei $V=M_V$ ein Multiplikationsoperator ist.\fish
\end{defn*}

\begin{thm}
\label{prop:5.2}
Sei $V \in L^2(\R^n)  + L^\infty(\R^n)$ reellwertig und $n\le 3$. Dann ist
(*) selbstadjungiert auf $D=H^2(\R^n)$ und wesentlich selbstadjungiert auf
$D=C_0^\infty(\R^n)$.\fish
\end{thm}
\begin{proof}
Da $V$ reellwertig ist, ist $M_V$ ein symmetrischer Multiplikationsoperator und
$-\Delta$ ist selbstadjungiert auf $H^2(\R^n)$ und wesentlich selbstadjungiert
auf $C_0^\infty(\R^n)$. Nach Theorem \ref{prop:1.6} existiert zu jedem $\ep > 0$
ein $C_\ep\in\R$, so dass
\begin{align*}
\norm{V\ph} \le \ep \norm{\Delta \ph} + C_\ep \norm{\ph},\qquad \ph\in
H^2(\R^n).
\end{align*}
Aus dem Satz von Kato-Rellich folgt schließlich, dass (*) selbstadjungiert auf
$H^2(\R^n)$ und wesentlich selbstadjungiert auf $C_0^\infty(\R^n)$ ist.\qed
\end{proof}

Mit denselben Methoden (vgl. \cite{RS05b}) beweist man das folgende
\begin{thm*}[Theorem von Kato (1951)]
Seien $V_k$, $V_{ik}\in L^2(\R^n) + L^\infty(\R^n)$ für $1\le i,k \le N$
reellwertig und $n\le 3$. Dann ist der Operator
\begin{align*}
-\Delta + \sum_{k=1}^N V_k(x_k) + \sum_{i<k} V_{ik}(x_i-x_k)
: D \subset L^2(\R^n)\to L^2(\R^n)
\end{align*}
selbstadjungiert auf $D=H^2(\R^{nN})$ und wesentlich selbstadjungiert auf
$D=C_0^\infty(\R^{nN})$.\fish
\end{thm*}

\begin{cor*}
Der Schrödingeroperator $H: D\subset L^2(\R^{3N})\to L^2(\R^{3N})$ gegeben durch
\begin{align*}
H=\sum_{k=1}^N \left(-\frac{\hbar^2}{2m}\Delta_{x_k}-\frac{Ze^2}{\abs{x_k}}
\right) + \sum_{i<k} \frac{e^2}{\abs{x_i-x_k}}
\end{align*}
ist selbstadjungiert auf $D=H^2(\R^{3N})$ und wesentlich selbstadjungiert auf
$D=C_0^\infty(\R^{3N})$.\fish
\end{cor*}

\begin{bem*}[Bemerkung zu den Einheiten.]
Der Operator aus dem obigen Korollar ist unitär äquivalent zu
\begin{align*}
2\alpha^2mc^2\left[ \sum_{k=1}^N \left(-\Delta_{x_k} -
\frac{Z}{\abs{x_k}}\right) + \sum_{i<k} \frac{1}{\abs{x_i-x_k}}
\right],
\end{align*}
wobei $\alpha= \frac{e^2}{\hbar c} \approx \frac{1}{137}$ die
Feinstrukturkonstante ist.

Die zugehörige unitäre Transformation ist gegeben durch
\begin{align*}
U: L^2(\R^{3N}) \to L^2(\R^{3N}),\qquad (U\ph)(x) := \lambda^{3N/2}\ph(\lambda
x)
\end{align*}
mit geeignetem $\lambda > 0$. Es gilt
\begin{align*}
UHU^{-1} = \sum_{k=1}^N \left(-\frac{\hbar^2}{2m\lambda^2}\Delta_{x_k} -
\frac{Ze^2}{\lambda \abs{x_k}}\right) + \sum_{i< k}
\frac{e^2}{\lambda\abs{x_i-x_k}}.
\end{align*}
Eine einfache Rechnung zeigt, dass
$\frac{\hbar^2}{2m\lambda^2}=\frac{e^2}{\lambda}$, mit der Wahl
\begin{align*}
\lambda = \frac{\hbar^2}{2me^2} = \frac{1}{2}\text{Bohr-Radius}.
\end{align*}
Der gemeinsame Faktor
\begin{align*}
\frac{\hbar^2}{2m\lambda^2} = \frac{e^2}{\lambda} = 2\alpha^2me^2
\end{align*}
ist die 4-fache Rydberg-Energie $E_{\mathrm{Ryd}} =
\frac{1}{2}\alpha^2mc^2$.\map
\end{bem*}